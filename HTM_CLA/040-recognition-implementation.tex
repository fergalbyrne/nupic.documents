\chapter{Pattern Memory Implementation and Pseudocode}
\label{chapter:pattern-memory}

This chapter contains the detailed pseudocode for a first
implementation of the Pattern Memory function. The input to this code
is an array of bottom-up binary inputs from sensory data or the
previous level. The code computes activeColumns(t) --- the list of
columns that win due to the bottom-up input at time t. This list is
then sent as input to the Transition Memory routine described in the
next chapter, i.e., activeColumns(t) is the output of the Pattern Memory routine.

The pseudocode is split into three distinct phases that occur in
sequence:

\begin{description}
\item[Phase 1:] compute the overlap with the current input for each column
\item[Phase 2:] compute the winning columns after inhibition
\item[Phase 3:] update synapse permanence and internal variables
\end{description}

Although Pattern Memory learning is inherently online, you can turn
off learning by simply skipping Phase 3.

The rest of the chapter contains the pseudocode for each of the three
steps. The various data structures and supporting routines used in the
code are defined at the end.

\subsection*{Initialization}

Prior to receiving any inputs, the region is initialized by computing
a list of initial potential synapses for each column. This consists of
a random set of inputs selected from the input space. Each input is
represented by a synapse and assigned a random permanence value. The
random permanence values are chosen with two criteria. First, the
values are chosen to be in a small range around connectedPerm (the
minimum permanence value at which a synapse is considered
``connected''). This enables potential synapses to become connected (or
disconnected) after a small number of training iterations. Second,
each column has a natural center over the input region, and the
permanence values have a bias towards this center (they have higher
values near the center).

\subsection*{Phase 1: Overlap}

Given an input vector, the first phase calculates the overlap of each
column with that vector. The overlap for each column is simply the
number of connected synapses with active inputs, multiplied by its
boost. If this value is below minOverlap, we set the overlap score to
zero.   

\begin{lstlisting}[numbers=left]
for c in columns

  overlap(c) = 0
  for s in connectedSynapses(c)
    overlap(c) = overlap(c) + input(t, s.sourceInput)

  if overlap(c) < minOverlap then
    overlap(c) = 0
  else
    overlap(c) = overlap(c) * boost(c)
\end{lstlisting}

\subsection*{Phase 2: Inhibition}
The second phase calculates which columns remain as winners after the
inhibition step. desiredLocalActivity is a parameter that controls the
number of columns that end up winning. For example, if
desiredLocalActivity is 10, a column will be a winner if its overlap
score is greater than the score of the 10th highest column within its
inhibition radius.

\begin{lstlisting}[numbers=left,firstnumber=11,mathescape]
for c in columns

  minLocalActivity = kthScore(neighbors(c), desiredLocalActivity)
  if overlap(c) > 0 and overlap(c) $\geq$ minLocalActivity then
    activeColumns(t).append(c)

\end{lstlisting}

\subsection*{Phase 3: Learning}
The third phase performs learning; it updates the permanence values of
all synapses as necessary, as well as the boost and inhibition radius.

The main learning rule is implemented in lines 20--26. For winning
columns, if a synapse is active, its permanence value is incremented,
otherwise it is decremented. Permanence values are constrained to be
between 0 and 1.

Lines 28--36 implement boosting. There are two separate boosting
mechanisms in place to help a column learn connections. If a column
does not win often enough (as measured by activeDutyCycle), its
overall boost value is increased (line 30--32). Alternatively, if a
column's connected synapses do not overlap well with any inputs often
enough (as measured by overlapDutyCycle), its permanence values are
boosted (line 34--36). Note: once learning is turned off, boost(c) is
frozen.  Finally, at the end of Phase 3 the inhibition radius is
recomputed (line 38).

\begin{lstlisting}[numbers=left,firstnumber=18]
for c in activeColumns(t)

  for s in potentialSynapses(c)
    if active(s) then
      s.permanence += permanenceInc
      s.permanence = min(1.0, s.permanence)
    else
      s.permanence -= permanenceDec
      s.permanence = max(0.0, s.permanence)

for c in columns:

  minDutyCycle(c) = 0.01 * maxDutyCycle(neighbors(c))
  activeDutyCycle(c) = updateActiveDutyCycle(c)
  boost(c) = boostFunction(activeDutyCycle(c), minDutyCycle(c))

  overlapDutyCycle(c) = updateOverlapDutyCycle(c)
  if overlapDutyCycle(c) < minDutyCycle(c) then
    increasePermanences(c, 0.1*connectedPerm)

inhibitionRadius = averageReceptiveFieldSize()

\end{lstlisting}

\subsection*{Supporting data structures and routines}

The following variables and data structures are used in the pseudocode:

\begin{description}
\item[columns] List of all columns.
\item[input(t,j)] The input to this level at time t. input(t, j) is 1
  if the j'th input is on.
\item[overlap(c)] The Pattern Memory overlap of column c with a
  particular input pattern.
\item[activeColumns(t)] List of column indices that are winners due to
  bottom-up input.
\item[desiredLocalActivity] A parameter controlling the number of
  columns that will be winners after the inhibition step.
\item[inhibitionRadius] Average connected receptive field size of the
  columns.
\item[neighbors(c)] A list of all the columns that are within
  inhibitionRadius of column c.
\item[minOverlap] A minimum number of inputs that must be active for a
  column to be considered during the inhibition step.
\item[boost(c)] The boost value for column c as computed during
  learning---used to increase the overlap value for inactive columns.
\item[synapse] A data structure representing a synapse---contains a
  permanence value and the source input index.
\item[connectedPerm] If the permanence value for a synapse is greater
  than this value, it is said to be connected.
\item[potentialSynapses(c)] The list of potential synapses and their
  permanence values.
\item[connectedSynapses(c)] A subset of potentialSynapses(c) where the
  permanence value is greater than connectedPerm. These are the
  bottom-up inputs that are currently connected to column c.
\item[permanenceInc] Amount permanence values of synapses are
  incremented during learning.
\item[permanenceDec] Amount permanence values of synapses are
  decremented during learning.
\item[activeDutyCycle(c)] A sliding average representing how often
  column c has been active after inhibition (e.g. over the last 1000
  iterations).
\item[overlapDutyCycle(c)] A sliding average representing how often
  column c has had significant overlap (i.e., greater than minOverlap)
  with its inputs (e.g. over the last 1000 iterations).
\item[minDutyCycle(c)] A variable representing the minimum desired
  firing rate for a cell. If a cell's firing rate falls below this
  value, it will be boosted. This value is calculated as 1\% of the
  maximum firing rate of its neighbors.
\end{description}

The following supporting routines are used in the above code.

\begin{description}
\item[kthScore(cols,k)] Given the list of columns, return the k'th
  highest overlap value.

\item[updateActiveDutyCycle(c)] Computes a moving average of how often
  column c has been active after inhibition.

\item[updateOverlapDutyCycle(c)] Computes a moving average of how
  often column c has overlap greater than minOverlap.

\item[averageReceptiveFieldSize()] The radius of the average connected
  receptive field size of all the columns. The connected receptive
  field size of a column includes only the connected synapses (those
  with permanence values $\ge$ connectedPerm). This is used to determine
  the extent of lateral inhibition between columns.

\item[maxDutyCycle(cols)] Returns the maximum active duty cycle of the
  columns in the given list of columns.

\item[increasePermanences(c, s)] Increase the permanence value of
  every synapse in column c by a scale factor s.

\item[boostFunction(c)] Returns the boost value of a column. The boost
  value is a scalar $\ge 1$. If activeDutyCyle(c) is above
  minDutyCycle(c), the boost value is 1. The boost increases linearly
  once the column's activeDutyCyle starts falling below its
  minDutyCycle.
\end{description}
