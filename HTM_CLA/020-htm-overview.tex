\chapter{HTM Overview}
\label{chapter:overview}

Hierarchical Temporal Memory (HTM) is a machine learning technology
that aims to capture the structural and algorithmic properties of the
neocortex.

The neocortex is the seat of intelligent thought in the mammalian
brain. High level vision, hearing, touch, movement, language, and
planning are all performed by the neocortex. Given such a diverse
suite of cognitive functions, you might expect the neocortex to
implement an equally diverse suite of specialized neural
algorithms. This is not the case. The neocortex displays a remarkably
uniform pattern of neural circuitry. The biological evidence suggests
that the neocortex implements a common set of algorithms to perform
many different intelligence functions.

HTM provides a theoretical framework for understanding the neocortex
and its many capabilities. To date we have implemented a small subset
of this theoretical framework. Over time, more and more of the theory
will be implemented. Today we believe we have implemented a sufficient
subset of what the neocortex does to be of commercial and scientific
value.

Programming HTMs is unlike programming traditional computers. With
today's computers, programmers create specific programs to solve
specific problems. By contrast, HTMs are trained through exposure to a
stream of sensory data. The HTM's capabilities are determined largely
by what it has been exposed to.

HTMs can be viewed as a type of neural network. By definition, any
system that tries to model the architectural details of the neocortex
is a neural network. However, on its own, the term ``neural network'' is
not very useful because it has been applied to a large variety of
systems. HTMs model neurons (called cells when referring to HTM),
are arranged in columns, in layers, in regions, and in a
hierarchy. The details matter, and in this regard HTMs are a new form
of neural network.

As the name implies, HTM is fundamentally a memory based system. HTM
networks are trained on lots of time varying data, and rely on storing
a large set of patterns and sequences. The way data is stored and
accessed is logically different from the standard model used by
programmers today. Classic computer memory has a flat organization and
does not have an inherent notion of time. A programmer can implement
any kind of data organization and structure on top of the flat
computer memory. They have control over how and where information is
stored. By contrast, HTM memory is more restrictive. HTM memory has a
hierarchical organization and is inherently time based. Information is
always stored in a distributed fashion. A user of an HTM specifies the
size of the hierarchy and what to train the system on, but the HTM
controls where and how information is stored. Although HTM networks
are substantially different than classic computing, we can use general
purpose computers to model them as long as we incorporate the key
functions of hierarchy, time and sparse distributed representations
(described in detail later). In a process which has already begun, specialized
hardware is being created to generate purpose-built HTM networks.

In this document, we often illustrate HTM properties and principles
using examples drawn from human vision, touch, hearing, language, and
behavior. Such examples are useful because they are intuitive and
easily grasped. However, it is important to keep in mind that HTM
capabilities are general. They can just as easily be exposed to
non-human sensory input streams, such as radar and infrared, or to
purely informational input streams such as financial market data,
weather data, Web traffic patterns, or text. HTMs are learning and
prediction machines that can be applied to many types of problems.

\section*{HTM principles}

In this section, we cover some of the core principles of HTM: why
hierarchical organization is important, how HTM regions are
structured, why data is stored as sparse distributed representations,
and why time-based information is critical.

\subsection*{Hierarchy}

An HTM network consists of regions arranged in a hierarchy. The region
is the main unit of memory and prediction in an HTM, and will be
discussed in detail in the next section. Typically, each HTM region
represents one level in the hierarchy. As you ascend the hierarchy
there is always convergence, multiple elements in a child region
converge onto an element in a parent region. However, due to feedback
connections, information also diverges as you descend the
hierarchy. (A ``region'' and a ``level'' are almost synonymous. We use
the word ``region'' when describing the internal function of a region,
whereas we use the word ``level'' when referring specifically to the
role of the region within the hierarchy.)

\begin{figure}
\resizebox{\textwidth}{!}{\includegraphics{figures/Figure-1_1.pdf}}
\caption{Simplified diagram of four HTM regions arranged in a
  four-level hierarchy, communicating information within levels,
  between levels, and to/from outside the hierarchy.}
\label{figure:region-hierarchy}
\end{figure}

It is possible to combine multiple HTM networks. This kind of
structure makes sense if you have data from more than one source or
sensor. For example, one network might be processing auditory
information and another network might be processing visual
information. There is convergence within each separate network, with
the separate branches converging only towards the top.

\begin{figure}
\resizebox{\textwidth}{!}{\includegraphics{figures/Figure-1_2.pdf}}
\caption{Converging networks from different sensors}
\label{figure:multimodal-convergence}
\end{figure}

The benefit of hierarchical organization is efficiency. It
significantly reduces training time and memory usage because patterns
learned at each level of the hierarchy are reused when combined in
novel ways at higher levels. For an illustration, let's consider
vision. At the lowest level of the hierarchy, your brain stores
information about tiny sections of the visual field such as edges and
corners. An edge is a fundamental component of many objects in the
world. These low-level patterns are recombined at mid-levels into more
complex components such as curves and textures. An arc can be the edge
of an ear, the top of a steering wheel or the rim of a coffee
cup. These mid-level patterns are further combined to represent
high-level object features, such as heads, cars or houses. To learn a
new high level object you don't have to relearn its components.

As another example, consider that when you learn a new word, you don't
need to relearn letters, syllables, or phonemes.

Sharing representations in a hierarchy also leads to generalization of
expected behavior. When you see a new animal, if you see a mouth and
teeth you will predict that the animal eats with his mouth and that it
might bite you. The hierarchy enables a new object in the world to
inherit the known properties of its sub-components.

How much can a single level in an HTM hierarchy learn? Or put another
way, how many levels in the hierarchy are necessary? There is a
tradeoff between how much memory is allocated to each level and how
many levels are needed. Fortunately, HTMs automatically learn the best
possible representations at each level given the statistics of the
input and the amount of resources allocated. If you allocate more
memory to a level, that level will form representations that are
larger and more complex, which in turn means fewer hierarchical levels
may be necessary. If you allocate less memory, a level will form
representations that are smaller and simpler, which in turn means more
hierarchical levels may be needed.

Up to this point we have been describing difficult problems, such as
vision inference (``inference'' is similar to pattern
recognition). But many valuable problems are simpler than vision, and
a single HTM region might prove sufficient. For example, we applied an
HTM to predicting where a person browsing a website is likely to click
next. This problem involved feeding the HTM network streams of web
click data. In this problem there was little or no spatial hierarchy,
the solution mostly required discovering the temporal statistics,
i.e., predicting where the user would click next by recognizing typical
user patterns. The temporal learning algorithms in HTMs are ideal for
such problems.

In summary, hierarchies reduce training time, reduce memory usage, and
introduce a form of generalization. However, many simpler prediction
problems can be solved with a single HTM region.

\subsection*{Regions}

The notion of regions wired in a hierarchy comes from biology. The
neocortex is a large sheet of neural tissue about 2mm
thick. Biologists divide the neocortex into different areas or
``regions'' primarily based on how the regions connect to each
other. Some regions receive input directly from the senses and other
regions receive input only after it has passed through several other
regions. It is the region-to-region connectivity that defines the
hierarchy.

All neocortical regions look similar in their details. They vary in
size and where they are in the hierarchy, but otherwise they are
similar. If you take a slice across the 2mm thickness of a neocortical
region, you will see six layers, five layers of cells and one
non-cellular layer (there are a few exceptions but this is the general
rule). Each layer in a neocortical region has many interconnected
cells arranged in columns. HTM regions also are comprised of a sheet
of highly interconnected cells arranged in columns. ``Layer 3'' in
neocortex is one of the primary feed-forward layers of neurons. The
cells in an HTM region are roughly equivalent to the neurons in Layer
3 in a region of the neocortex.

A fuller description of a HTM region, introduced for the first time in
this revision, includes a model of how Layers 1, 2/3, 4, 5 and 6 combine 
to perform sensorimotor sequence memory, feedback, behavior and hierarchy.
We begin with a detailed description of a single-layer region, and later
extend these concepts to the full model including all the layers.

\begin{figure}
\resizebox{\textwidth}{!}{\includegraphics{figures/Figure-1_3.pdf}}
\caption{A section of an HTM region. HTM regions are comprised of many
  cells. The cells are organized in a two dimensional array of
  columns. This figure shows a small section of a single-layer HTM region with
  four cells per column. Each column connects to a subset of the input
  and each cell connects to other cells in the region (connections not
  shown). Note that this HTM region, including its columnar structure,
  is equivalent to one layer of neurons in a neocortical region.}
\label{figure:one-region}
\end{figure}

Although an HTM region is equivalent to only a portion of a neocortical region, 
it can do inference and prediction on complex data streams and therefore can be useful in many problems.

\subsection*{Sparse Distributed Representations}

Although neurons in the neocortex are highly interconnected,
inhibitory neurons guarantee that only a small percentage of the
neurons are active at one time. Thus, information in the brain is
always represented by a small percentage of active neurons within a
large population of neurons. This kind of encoding is called a
``sparse distributed representation.'' ``Sparse'' means that only a
small percentage of neurons are active at one time. ``Distributed''
means that the activations of many neurons are required in order to
represent something. A single active neuron conveys some meaning but
it must be interpreted within the context of a population of neurons
to convey the full meaning.

HTM regions also use sparse distributed representations. In fact, the
memory mechanisms within an HTM region are dependent on using sparse
distributed representations, and wouldn't work otherwise. The input to
an HTM region is always a distributed representation, but it may not
be sparse, so the first thing an HTM region does is to convert its
input into a sparse distributed representation.

For example, a region might receive 20,000 input bits. The percentage
of input bits that are ``1'' and ``0'' might vary significantly over
time. One time there might be 5,000 ``1'' bits and another time there
might be 9,000 ``1'' bits. The HTM region could convert this input
into an internal representation of 10,000 bits of which 2\%, or 200,
are active at once, regardless of how many of the input bits are
``1.'' As the input to the HTM region varies over time, the internal
representation also will change, but there always will be about 200
bits out of 10,000 active.

It may seem that this process generates a large loss of information as
the number of possible input patterns is much greater than the number
of possible representations in the region. However, both numbers are
incredibly big. The actual inputs seen by a region will be a miniscule
fraction of all possible inputs. Later we will describe how a region
creates a sparse representation from its input. The theoretical loss
of information will not have a practical effect.

\begin{figure}
\resizebox{\textwidth}{!}{\includegraphics{figures/Figure-1_4.pdf}}
\caption{An HTM region showing sparse distributed cell activation}
\label{figure:sparse-activation}
\end{figure}

Sparse distributed representations have several desirable properties
and are integral to the operation of HTMs. They will be touched on
again later.

\subsection*{The role of time}

Time plays a crucial role in learning, inference, and prediction.

Let's start with inference. Without using time, we can infer almost
nothing from our tactile and auditory senses. For example if you are
blindfolded and someone places an apple in your hand, you can identify
what it is after manipulating it for just a second or so. As you move
your fingers over the apple, although the tactile information is
constantly changing, the object itself --- the apple, as well as your
high-level percept for ``apple'' --- stays constant. However, if an
apple was placed on your outstretched palm, and you weren't allowed to
move your hand or fingers, you would have great difficulty identifying
it as an apple rather than a lemon.

The same is true for hearing. A static sound conveys little meaning. A
word like ``apple,'' or the crunching sounds of someone biting into an
apple, can only be recognized from the dozens or hundreds of rapid,
sequential changes over time of the sound spectrum.

Vision, in contrast, is a mixed case. Unlike with touch and hearing,
humans are able to recognize images when they are flashed in front of
them too fast to give the eyes a chance to move. Thus, visual
inference does not always require time-changing inputs. However,
during normal vision we constantly move our eyes, heads and bodies,
and objects in the world move around us too. Our ability to infer
based on quick visual exposure is a special case made possible by the
statistical properties of vision and years of training. The general
case for vision, hearing, and touch is that inference requires
time-changing inputs.

Having covered the general case of inference, and the special case of
vision inference of static images, let's look at learning. In order to
learn, all HTM systems must be exposed to time-changing inputs during
training. Even in vision, where static inference is sometimes
possible, we must see changing images of objects to learn what an
object looks like. For example, imagine a dog is running toward
you. At each instance in time the dog causes a pattern of activity on
the retina in your eye. You perceive these patterns as different views
of the same dog, but mathematically the patterns are entirely
dissimilar. The brain learns that these different patterns mean the
same thing by observing them in sequence. Time is the ``supervisor,''
teaching you which spatial patterns go together.

Note that it isn't sufficient for sensory input merely to change. A
succession of unrelated sensory patterns would only lead to
confusion. The time-changing inputs must come from a common source in
the world. Note also that although we use human senses as examples,
the general case applies to non-human senses as well. If we want to
train an HTM to recognize patterns from a power plant's temperature,
vibration and noise sensors, the HTM will need to be trained on data
from those sensors changing through time.

Typically, an HTM network needs to be trained with lots of data. You
learned to identify dogs by seeing many instances of many breeds of
dogs, not just one single view of one single dog. The job of the HTM
algorithms is to learn the temporal sequences from a stream of input
data, i.e., to build a model of which patterns follow which other
patterns. This job is difficult because it may not know when sequences
start and end, there may be overlapping sequences occurring at the
same time, learning has to occur continuously, and learning has to
occur in the presence of noise.

Learning and recognizing sequences is the basis of forming
predictions. Once an HTM learns what patterns are likely to follow
other patterns, it can predict the likely next pattern(s) given the
current input and immediately past inputs. Prediction is covered in
more detail later.

We now will turn to the four basic functions of HTM: learning,
inference, prediction, and behavior. Every HTM region performs the
first three functions: learning, inference, and prediction. Behavior,
however, is different. We know from biology that most neocortical
regions have a role in creating behavior but we do not believe it is
essential for many interesting applications. Therefore we have not
included behavior in our current implementation of HTM. 

A substantially more complex theory involving the interaction of behavior
and sensation, formation of stable sequences, the roles of all layers in
a region, feedback and hierarchy has been recently developed and is presented
here for the first time.



\subsection*{Learning}

An HTM region learns about its world by finding patterns and then
sequences of patterns in sensory data. The region does not ``know''
what its inputs represent; it works in a purely statistical realm. It
looks for combinations of input bits that occur together often, which
we call spatial patterns. It then looks for how these spatial patterns
appear in sequence over time, which we call temporal patterns or
sequences.

If the input to the region represents environmental sensors on a
building, the region might discover that certain combinations of
temperature and humidity on the north side of the building occur often
and that different combinations occur on the south side of the
building. Then it might learn that sequences of these combinations
occur as each day passes.

If the input to a region represented information related to purchases
within a store, the HTM region might discover that certain types of
articles are purchased on weekends, or that when the weather is cold
certain price ranges are favored in the evening. Then it might learn
that different individuals follow similar sequential patterns in their
purchases.

A single HTM region has limited learning capability. A region
automatically adjusts what it learns based on how much memory it has
and the complexity of the input it receives. The spatial patterns
learned by a region will necessarily become simpler if the memory
allocated to a region is reduced. Or the spatial patterns learned can
become more complex if the allocated memory is increased. If the
learned spatial patterns in a region are simple, then a hierarchy of
regions may be needed to understand complex images. We see this
characteristic in the human vision system where the neocortical region
receiving input from the retina learns spatial patterns for small
parts of the visual space. Only after several levels of hierarchy do
spatial patterns combine and represent most or all of the visual
space.

Like a biological system, the learning algorithms in an HTM region are
capable of ``on-line learning,'' i.e., they continually learn from each
new input. There isn't a need for a learning phase separate from an
inference phase, though inference improves after additional
learning. As the patterns in the input change, the HTM region will
gradually change, too.

After initial training, an HTM can continue to learn or,
alternatively, learning can be disabled after the training
phase. Another option is to turn off learning only at the lowest
levels of the hierarchy but continue to learn at the higher
levels. Once an HTM has learned the basic statistical structure of its
world, most new learning occurs in the upper levels of the
hierarchy. If an HTM is exposed to new patterns that have previously
unseen low-level structure, it will take longer for the HTM to learn
these new patterns. We see this trait in humans. Learning new words in
a language you already know is relatively easy. However, if you try to
learn new words from a foreign language with unfamiliar sounds, you'll
find it much harder because you don't already know the low level
sounds.

Simply discovering patterns is a potentially valuable
capability. Understanding the high-level patterns in market
fluctuations, disease, weather, manufacturing yield, or failures of
complex systems, such as power grids, is valuable in itself. Even so,
learning spatial and temporal patterns is mostly a precursor to
inference and prediction.

\subsection*{Inference}

After an HTM has learned the patterns in its world, it can perform
inference on novel inputs. When an HTM receives input, it will match
it to previously learned spatial and temporal patterns. Successfully
matching new inputs to previously stored sequences is the essence of
inference and pattern matching.

Think about how you recognize a melody. Hearing the first note in a
melody tells you little. The second note narrows down the
possibilities significantly but it may still not be enough. Usually it
takes three, four, or more notes before you recognize the
melody. Inference in an HTM region is similar. It is constantly
looking at a stream of inputs and matching them to previously learned
sequences. An HTM region can find matches from the beginning of
sequences but usually it is more fluid, analogous to how you can
recognize a melody starting from anywhere. Because HTM regions use
distributed representations, the region's use of sequence memory and
inference are more complicated than the melody example implies, but
the example gives a flavor for how it works.

It may not be immediately obvious, but every sensory experience you
have ever had has been novel, yet you easily find familiar patterns in
this novel input. For example, you can understand the word
``breakfast'' spoken by almost anyone, no matter whether they are old
or young, male or female, are speaking quickly or slowly, or have a
strong accent. Even if you had the same person say the same word
``breakfast'' a hundred times, the sound would never stimulate your
cochleae (auditory receptors) in exactly the same way twice.

An HTM region faces the same problem your brain does: inputs may never
repeat exactly. Consequently, just like your brain, an HTM region must
handle novel input during inference and training. One way an HTM
region copes with novel input is through the use of sparse distributed
representations. A key property of sparse distributed representations
is that you only need to match a portion of the pattern to be
confident that the match is significant.

\subsection*{Prediction}

Every region of an HTM stores sequences of patterns. By matching
stored sequences with current input, a region forms a prediction about
what inputs will likely arrive next. HTM regions actually store
transitions between sparse distributed representations. In some
instances the transitions can look like a linear sequence, such as the
notes in a melody, but in the general case many possible future inputs
may be predicted at the same time. An HTM region will make different
predictions based on context that might stretch back far in time. The
majority of memory in an HTM is dedicated to sequence memory, or
storing transitions between spatial patterns.

Following are some key properties of HTM prediction.

\begin{enumerate}
\item {\bf Prediction is continuous.}

Without being conscious of it, you are constantly predicting. HTMs do
the same. When listening to a song, you are predicting the next
note. When walking down the stairs, you are predicting when your foot
will touch the next step. When watching a baseball pitcher throw, you
are predicting that the ball will come near the batter. In an HTM
region, prediction and inference are almost the same thing. Prediction
is not a separate step but integral to the way an HTM region works.

\item {\bf Prediction occurs in every region at every level of the hierarchy.}

If you have a hierarchy of HTM regions, prediction will occur at each
level. Regions will make predictions about the patterns they have
learned. In a language example, lower level regions might predict
possible next phonemes, and higher level regions might predict words
or phrases.

\item {\bf Predictions are context sensitive.}

Predictions are based on what has occurred in the past, as well as
what is occurring now. Thus an input will produce different
predictions based on previous context. An HTM region learns to use as
much prior context as needed, and can keep the context over both short
and long stretches of time. This ability is known as ``variable
order'' memory. For example, think about a memorized speech such as
the Gettysburg Address. To predict the next word, knowing just the
current word is rarely sufficient; the word ``and'' is followed by
``seven'' and later by ``dedicated'' just in the first
sentence. Sometimes, just a little bit of context will help
prediction; knowing ``four score and'' would help predict ``seven.''
Other times, there are repetitive phrases, and one would need to use
the context of a far longer timeframe to know where you are in the
speech, and therefore what comes next.

\item {\bf A prediction tells us if a new input is expected or unexpected.}

Each HTM region is a novelty detector. Because each region predicts
what will occur next, it ``knows'' when something unexpected
happens. HTMs can predict many possible next inputs simultaneously,
not just one. So it may not be able to predict exactly what will
happen next, but if the next input doesn't match any of the
predictions the HTM region will know that an anomaly has occurred.

In a sensorimotor HTM, a region also uses information about motor commands
to assist with prediction. As long as new inputs are correctly predicted, the 
region will be able to form a stable representation of the input. If prediction 
fails, the region will signal this change in the input.

\item {\bf Prediction helps make the system more robust to noise.}

When an HTM predicts what is likely to happen next, the prediction can
bias the system toward inferring what it predicted. For example, if an
HTM were processing spoken language, it would predict what sounds,
words, and ideas are likely to be uttered next. This prediction helps
the system fill in missing data. If an ambiguous sound arrives, the
HTM will interpret the sound based on what it is expecting, thus
helping inference even in the presence of noise.

\item {\bf Prediction leads to stability.}

One of the properties of HTMs is that the outputs of regions become more stable --- that is
slower changing, longer-lasting --- the higher they are in the
hierarchy. This property results from how a region predicts.
In the new sensorimotor theory, higher-level regions communicate a SDR
of a slowly-changing sequence to lower layers, helping them predict the inputs
they will receive in this context. As stated previously, each region also uses 
motor information to help predict transitions in the patterns recognised by the region.
This characteristic mirrors our experience of the
real world, where high level concepts --- such as the name of a song
--- change more slowly than low level concepts --- the actual notes of
the song.

In an HTM region, sequence memory, inference, and prediction are
intimately integrated. They are the core functions of a region.

\end{enumerate}

\subsection*{Behavior}

Our behavior influences what we perceive. As we move our eyes, our
retina receives changing sensory input. Moving our limbs and fingers
causes varying touch sensation to reach the brain. Almost all our
actions change what we sense, and a very large proportion of what we sense 
is due to our own behavior. Sensory input and motor behavior are
intimately entwined.

For decades the prevailing view was that a single region in the
neocortex, the primary motor region, was where motor commands
originated in the neocortex. Over time it was discovered that most or
all regions in the neocortex have a motor output, even low level
sensory regions. It appears that all cortical regions integrate
sensory and motor functions.

We expect that a motor output could be added to each HTM region within
the currently existing framework since generating motor commands is
similar to making predictions. However, all the implementations of
HTMs to date have been purely sensory, without a motor component.

In this revision, we introduce a new framework which explains how 
behavior and sensory/feedforward inputs are integrated and produced in a
fuller HTM model.

\subsection*{Progress toward the implementation of HTM}

We have made substantial progress turning the HTM theoretical
framework into a practical technology. We have implemented and tested
several versions of the HTM cortical learning algorithms and have
found the basic architecture to be sound. As we test the algorithms on
new data sets, we will refine the algorithms and add missing
pieces. We will update this document as we do. The next three chapters
describe the current state of the algorithms.

There are many components of the theory that are not yet implemented,
including attention, feedback between regions, specific timing, and
behavior/sensory-motor integration. These missing components should
fit into the framework already created.
