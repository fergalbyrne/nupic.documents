\chapter{HTM Cortical Learning Algorithms}
\label{chapter:learning}

This chapter describes the learning algorithms at work inside an HTM
region. Chapters~\ref{chapter:pattern-memory} and
\ref{chapter:transition-memory} describe the implementation of the
learning algorithms using pseudocode, whereas this chapter is more
conceptual.

\section*{Terminology}

Before we get started, a note about terminology might be helpful. We
use the language of neuroscience in describing the HTM learning
algorithms. Terms such as cells, synapses, potential synapses,
dendrite segments, and columns are used throughout. This terminology
is logical since the learning algorithms were largely derived by
matching neuroscience details with theoretical needs. However, in the
process of implementing the algorithms we were confronted with
performance issues and therefore once we felt we understood how
something worked we would look for ways to speed processing. This
often involved deviating from a strict adherence to biological details
as long as we could get the same results. If you are new to
neuroscience this won't be a problem. However, if you are familiar
with neuroscience terms, you might find yourself confused as our use
of terms varies from your expectation. The appendixes on biology
discuss the differences and similarities between the HTM learning
algorithms and their neurobiological equivalents in detail. Here we
will mention a few of the deviations that are likely to cause the most
confusion.

\subsection*{Region}

For the moment, we will model each region as a single layer, which receives 
a stream of feedforward inputs, recognises spatial patterns, learns sequences
of these patterns, and makes predictions. In the neocortex, this models the key functions performed 
in Layer 3. We call this initial model the ``Sensory CLA.''

In a later chapter, we will use this basic understanding of recognition, sequence learning, and prediction
to build a more complete sensorimotor temporal memory system. This new theory is called the ``Sensorimotor
CLA.''

\subsection*{Cell states}

HTM cells have two output states, active (firing) and inactive. We have not
found a need for modeling individual action potentials or even scalar
rates of activity beyond the two active states. The use of distributed
representations seems to overcome the need to model scalar activity
rates in cells.

Apart from overt activity, each cell possesses a level of ``potential 
activation'' which corresponds to the amount of depolarisation taking place
during a timestep. This depolarisation arises from feed-forward input,
as well as from lateral input (which represents a prediction), and
these are considered separately. The former is primarily used to perform
feedforward recognition (and thus the region's SDR) while the latter is used
to identify cells as ``predictive''.

\subsection*{Dendrite segments}

HTM cells have a relatively realistic (and therefore complex) dendrite
model. In theory each HTM cell has one proximal dendrite segment and a
dozen or two distal dendrite segments. The proximal dendrite segment
receives feed-forward input and the distal dendrite segments receive
lateral input from nearby cells. A class of inhibitory cells forces
all the cells in a column to respond to similar feed-forward input. 

To simplify, we removed the proximal dendrite segment from each cell and
replaced it with a single shared dendrite segment per column of
cells. The recognition function (described below) operates on the
shared dendrite segment, at the level of columns. The prediction
function operates on distal dendrite segments, at the level of
individual cells within columns. This simplification achieves the same
functionality, though in biology there is no equivalent to a dendrite
segment attached to a column.

Some implementations of HTM have re-introduced the per-cell proximal 
dendrites. The new Sensorimotor CLA theory uses both the per-cell and per-column
models of proximal dendrites (the latter being due to fast inhibitory cells 
surrounding the column and responding to similar inputs).

\subsection*{Synapses}

HTM synapses have binary weights. Biological synapses have varying
weights but they are also partially stochastic, suggesting a
biological neuron cannot rely on precise synaptic weights. The use of
distributed representations in HTMs plus our model of dendrite
operation allows us to assign binary weights to HTM synapses with no
ill effect. To model the forming and un-forming of synapses we use two
additional concepts from neuroscience that you may not be familiar
with. One is the concept of ``potential synapses.'' This represents
all the axons that pass close enough to a dendrite segment that they
could potentially form a synapse. The second is called ``permanence.''
This is a scalar value assigned to each potential synapse. The
permanence of a synapse represents a range of connectedness between an
axon and a dendrite. Biologically, this corresponds to the extent of growth
of dendritic spines, the size and contents of the axonal terminal, and 
the number and characteristics of receptor gates in each synapse.
We simplify this drastically by modelling the range from completely
unconnected, to starting to form a synapse but not connected yet, to a
minimally connected synapse, to a large fully connected synapse. The
permanence of a synapse is a scalar value ranging from $0.0$ to
$1.0$. Learning involves incrementing and decrementing a synapse's
permanence. When a synapse's permanence is above a threshold, it is
connected with a weight of ``1''. When it is below the threshold, it
is unconnected with a weight of ``0.''

\section*{Overview}

Imagine that you are a region of an HTM. Your input consists of
thousands or tens of thousands of bits. These input bits may represent
sensory data or they may come from another region lower in the
hierarchy. They are turning on and off in complex ways. What are you
supposed to do with this input?

We already have discussed the answer in its simplest form. Each HTM
region looks for common patterns in its input and then learns
sequences of those patterns. From its memory of sequences, each region
makes predictions. That high level description makes it sound easy,
but in reality there is a lot going on. Let's break it down a little
further into the following three steps:

\begin{enumerate}
\item Form a sparse distributed representation of the input
\item Form a representation of the input in the context of previous
  inputs
\item Form a prediction based on the current input in the context of
  previous inputs
\end{enumerate}

We will discuss each of these steps in more detail.

\begin{enumerate}
\item {\bf Form a sparse distributed representation of the input}

When you imagine an input to a region, think of it as a large number
of bits. In a brain these would be axons from neurons. At any point in
time some of these input bits will be active (value 1) and others will
be inactive (value 0). The percentage of input bits that are active
vary, say from 0\% to 60\%. The first thing an HTM region does is to
convert this input into a new representation that is sparse. For
example, the input might have 40\% of its bits ``on'' but the new
representation has just 2\% of its bits ``on.''

An HTM region is logically comprised of a set of columns. Each column
is comprised of one or more cells. Columns may be logically arranged
in a 2D array but this is not a requirement. Each column in a region
is connected to a unique subset of the input bits (usually overlapping
with other columns but never exactly the same subset of input
bits). As a result, different input patterns result in different
levels of activation potential in the columns. The columns with the strongest
activation inhibit, or deactivate, the columns with weaker
activation. (The inhibition occurs within a radius that can span from
very local to the entire region.) The sparse representation of the
input is encoded by which columns are active and which are inactive
after inhibition. The inhibition function is defined to achieve a
relatively constant percentage of columns to be active, even when the
number of input bits that are active varies significantly.

\begin{figure}
\resizebox{\textwidth}{!}{\includegraphics{figures/Figure-2_1.pdf}}
\caption{An HTM region consists of columns of cells. Only a small
  portion of a region is shown. Each column of cells receives
  activation from a unique subset of the input. Columns with the
  strongest activation inhibit columns with weaker activation. The
  result is a sparse distributed representation of the input. The
  figure shows active columns in light grey. (When there is no prior
  state, every cell in the active columns will be active, as shown.)}
\label{figure:region-columns}
\end{figure}

Imagine now that the input pattern changes. If only a few input bits
change, some columns will receive a few more or a few less inputs in
the ``on'' state, but the set of active columns will not likely change
much. Thus similar input patterns (ones that have a significant number
of active bits in common) will map to a relatively stable set of
active columns. How stable the encoding is depends greatly on what
inputs each column is connected to. These connections are learned via
a method described later.

All these steps (learning the connections to each column from a subset
of the inputs, determining the level of input to each column, and
using inhibition to select a sparse set of active columns) is referred
to as ``Pattern Memory.''

Another term we've used is ``Spatial Pooling'', because patterns that are
``spatially'' similar (meaning they share a large number of active
bits) are ``pooled'' (meaning they are grouped together in a common
representation).

\item {\bf Form a representation of the input in the context of
  previous inputs}

The next function performed by a region is to convert the columnar
representation of the input into a new representation that includes
state, or context, from the past. The new representation is formed by
activating a subset of the cells within each column, typically only
one cell per column (Figure~\ref{figure:column-partial-activation}).

Consider hearing two spoken sentences, ``I ate a pear'' and ``I have
eight pears.'' The words ``ate'' and ``eight'' are homonyms; they
sound identical. We can be certain that at some point in the brain
there are neurons that respond identically to the spoken words ``ate''
and ``eight.'' After all, identical sounds are entering the
ear. However, we also can be certain that at another point in the
brain the neurons that respond to this input are different, in
different contexts. The representations for the sound ``ate'' will be
different when you hear ``I ate'' vs.\ ``I have eight.'' Imagine that
you have memorized the two sentences ``I ate a pear'' and ``I have
eight pears.'' Hearing ``I ate\dots'' leads to a different prediction
than ``I have eight\dots.'' There must be different internal
representations upon hearing ``I ate'' and ``I have eight.'' This
principle of encoding an input differently in different contexts is a
universal feature of perception and action and is one of the most
important functions of an HTM region. It is hard to overemphasize the
importance of this capability.

Each column in an HTM region consists of multiple cells. All cells in
a column get the same feed-forward input. Each cell in a column can be
active or not active. By selecting different active cells in each
active column, we can represent the exact same input differently in
different contexts. A specific example might help. Say every column
has 10 cells and the representation of every input consists of 100
active columns. If only one cell per column is active at a time, we
have $10^{100}$ ways of representing the exact same input. The same
input will always result in the same 100 columns being active, but in
different contexts different cells in those columns will be
active. Now we can represent the same input in a very large number of
contexts, but how unique will those different representations be?
Nearly all randomly chosen pairs of the $10^{100}$ possible patterns
will overlap by about 10 cells. Thus two representations of a
particular input in different contexts will have about 10 cells in
common and 90 cells that are different, making them easily
distinguishable.

The general rule used by an HTM region is the following. When a column
becomes active, it looks at all the cells in the column. If one or
more cells in the column are already in the predictive state, only
those cells become active. If no cells in the column are in the
predictive state, then all the cells become active. You can think of
it this way, if an input pattern is expected then the system confirms
that expectation by activating only the cells in the predictive
state. If the input pattern is unexpected then the system activates
all cells in the column as if to say ``the input occurred unexpectedly
so all possible interpretations are valid.''

If there is no prior state, and therefore no context and prediction,
all the cells in a column will become active when the column becomes
active. This scenario is similar to hearing the first note in a
song. Without context you usually can't predict what will happen next;
all options are available. If there is prior state but the input does
not match what is expected, all the cells in the active column will
become active. This determination is done on a column by column basis
so a predictive match or mismatch is never an ``all-or-nothing''
event.

\begin{figure}
\resizebox{\textwidth}{!}{\includegraphics{figures/Figure-2_2.pdf}}
\caption{By activating a subset of cells in each column, an HTM region
  can represent the same input in many different contexts. Columns
  only activate predicted cells. Columns with no predicted cells
  activate all the cells in the column. The figure shows some columns
  with one cell active and some columns with all cells active.}
\label{figure:column-partial-activation}
\end{figure}

As mentioned in the terminology section above, HTM cells can be in one
of three states. If a cell is active due to feed-forward input we just
use the term ``active.'' If the cell has high potential due to lateral
connections to other nearby cells we say it is in the ``predictive
state'' (Figure~\ref{figure:activity-types}).

\item {\bf Form a prediction based on the input in the context of
  previous inputs}

The final step for our region is to make a prediction of what is
likely to happen next. The prediction is based on the representation
formed in step 2), which includes context from all previous inputs.

When a region makes a prediction it depolarises (into the predictive
state) all the cells that will likely become active due to future
feed-forward input. Because representations in a region are sparse,
multiple predictions can be made at the same time. For example if 2\%
of the columns are active due to an input, you could expect that ten
different predictions could be made resulting in 20\% of the columns
having a predicted cell. Or, twenty different predictions could be
made resulting in 40\% of the columns having a predicted cell. If each
column had four cells, with one active at a time, then 10\% of the
cells would be in the predictive state.

A future chapter on sparse distributed representations will show that
even though different predictions are merged together, a region can
know with high certainty whether a particular input was predicted or
not.

How does a region make a prediction? When input patterns change over
time, different sets of columns and cells become active in
sequence. When a cell becomes active, it forms connections to a subset
of the cells nearby that were active immediately prior. These
connections can be formed quickly or slowly depending on the learning
rate required by the application. Later, all a cell needs to do is to
look at these connections for coincident activity. If the connections
become active, the cell can expect that it might become active shortly
and enters a predictive state. Thus the feed-forward activation of a
set of cells will lead to the predictive depolarisation of other sets of
cells that typically follow. Think of this as the moment when you
recognize a song and start predicting the next notes.

\begin{figure}
\resizebox{\textwidth}{!}{\includegraphics{figures/Figure-2_3.pdf}}
\caption{At any point in time, some cells in an HTM region will be
  active due to feed-forward input (shown in light gray). Other cells
  that receive lateral input from active cells will be in a predictive
  state (shown in dark gray).}
\label{figure:activity-types}
\end{figure}

In summary, when a new input arrives, it leads to a sparse set of
active columns. One or more of the cells in each column become active,
these in turn cause other cells to enter a predictive state through
learned connections between cells in the region. The cells depolarised
by connections within the region constitute a prediction of what is
likely to happen next. When the next feed-forward input arrives, it
selects another sparse set of active columns. If a newly active column
is unexpected, meaning it was not predicted by any cells, it will
activate all the cells in the columns. If a newly active column has
one or more predicted cells, only those cells will become active. The
output of a region is the pattern of activity of all cells in the region,
including the cells active because of feed-forward input, and in particular those cells
which were predictive of the current input in context.

%As mentioned earlier, predictions are not just for the next time
%step. Predictions in an HTM region can be for several time steps into
% the future. Using melodies as example, an HTM region would not just
% predict the next note in a melody, but might predict the next four
% notes. This leads to a desirable property. The output of a region (the
% union of all the active and predicted cells in a region) changes more
% slowly than the input. Imagine the region is predicting the next four
% notes in a melody. We will represent the melody by the letter sequence
% A,B,C,D,E,F,G. After hearing the first two notes, the region
% recognizes the sequence and starts predicting. It predicts
% C,D,E,F. The ``B'' cells are already active so cells for B,C,D,E,F are
% all in one of the two active states. Now the region hears the next
% note ``C.'' The set of active and predictive cells now represents
% ``C,D,E,F,G.'' Note that the input pattern changed completely going
% from ``B'' to ``C,'' but only 20\% of the cells changed.



\end{enumerate}

% Because the output of an HTM region is a vector representing the
% activity of all the region's cells, the output in this example is five
% times more stable than the input. In a hierarchical arrangement of
% regions, we will see an increase in temporal stability as you ascend
% the hierarchy.

We use the term ``Transition Memory'' to describe the two steps of
adding context to the representation and making cells predictive. 
% By creating
% slowly changing outputs for sequences of patterns, we are in essence
% ``pooling'' together different patterns that follow each other in
% time.

Now we will go into another level of detail. We start with concepts
that are shared by both Pattern and Transition Memory. Then we
discuss concepts and details unique to Pattern Memory followed by
those unique to Transition Memory.

\section*{Shared concepts}

Learning in the Pattern and Transition Memory is
similar. Learning in both cases involves establishing connections, or
synapses, between cells. The Transition Memory learns connections
between cells in the same region. The Pattern Memory learns
feed-forward connections between input bits and columns.

\subsection*{Binary weights}
HTM synapses have only a 0 or 1 effect; their ``weight'' is binary, a
property unlike many neural network models which use scalar variable
values in the range of 0 to 1.

\subsection*{Permanence}
Synapses are forming and unforming constantly during learning. As
mentioned before, we assign a scalar value to each synapse ($0.0$ to
$1.0$) to indicate how permanent the connection is. When a connection is
reinforced, its permanence is increased. Under other conditions, the
permanence is decreased. When the permanence is above a threshold
(e.g., $0.2$), the synapse is considered to be established. If the
permanence is below the threshold, the synapse will have no effect.

\subsection*{Dendrite segments}
Synapses connect to dendrite segments. There are two types of dendrite
segments, proximal and distal.
\begin{itemize}
\item A proximal dendrite segment forms synapses with feed-forward
  inputs. The active synapses on this type of segment are linearly
  summed to determine the feed-forward activation potential of a column.

\item A distal dendrite segment forms synapses with cells within the
  region. Every cell has several distal dendrite segments. If the sum
  of the active synapses on a distal segment exceeds a threshold, then
  the associated cell becomes active in a predicted state. Since there
  are multiple distal dendrite segments per cell, a cell's predictive
  state is the logical OR operation of several constituent threshold
  detectors.
\end{itemize}

\subsection*{Potential Synapses}
As mentioned earlier, each dendrite segment has a list of potential synapses. 
All the potential synapses are given a permanence value and may become functional synapses if their permanence values exceed a threshold.

\subsection*{Learning}
Learning involves incrementing or decrementing the permanence values
of potential synapses on a dendrite segment. The rules used for making
synapses more or less permanent are similar to ``Hebbian'' learning
rules. For example, if a post-synaptic cell is active due to a
dendrite segment receiving input above its threshold, then the
permanence values of the synapses on that segment are
modified. Synapses that are active, and therefore contributed to the
cell being active, have their permanence increased. Synapses that are
inactive, and therefore did not contribute, have their permanence
decreased. The exact conditions under which synapse permanence values
are updated differ in the Pattern and Transition Memories. The details are
described below.

Now we will discuss concepts specific to the Pattern and Transition Memory functions.

\section*{Pattern Memory Concepts}

The most fundamental function of the Pattern Memory is to convert a
region's input into a sparse pattern. This function is important
because the mechanism used to learn sequences and make predictions
requires starting with sparse distributed patterns.

There are several overlapping goals for the Pattern Memory, which
determine how it operates and learns.

\begin{enumerate}
\item {\bf Use all columns - Boosting}

An HTM region has a fixed number of columns that learn to represent
common patterns in the input. One objective is to make sure all the
columns learn to represent something useful regardless of how many
columns you have. We don't want columns that are never active. To
prevent this from happening, we keep track of how often a column is
active relative to its neighbors. If the relative activity of a column
is too low, it boosts its input activity level until it starts to be
part of the winning set of columns. In essence, all columns are
competing with their neighbors to be a participant in representing
input patterns. If a column is not very active, it will become more
aggressive. When it does, other columns will be forced to modify their
input and start representing slightly different input patterns.

\item {\bf Maintain desired density - Inhibition}

A region needs to form a sparse representation of its inputs. Columns
with the most input inhibit their neighbors. There is a radius of
inhibition which is proportional to the size of the receptive fields
of the columns (and therefore can range from small to the size of the
entire region). Within the radius of inhibition, we allow only a
percentage of the columns with the most active input to be
``winners.'' The remainders of the columns are disabled. (A ``radius''
of inhibition implies a 2D arrangement of columns, but the concept can
be adapted to other topologies.)

\item {\bf Avoid trivial patterns}

We want all our columns to represent non-trivial patterns in the
input. This goal can be achieved by setting a minimum threshold of
input for the column to be active. For example, if we set the
threshold to 50, it means that a column must have a least 50 active
synapses on its dendrite segment to be active, guaranteeing a certain
level of complexity to the pattern it represents.

\item {\bf Avoid extra connections}

If we aren't careful, a column could form a large number of valid
synapses. It would then respond strongly to many different unrelated
input patterns. Different subsets of the synapses would respond to
different patterns. To avoid this problem, we decrement the permanence
value of any synapse that isn't currently contributing to a winning
column. By making sure non-contributing synapses are sufficiently 
penalized, we guarantee a column represents a limited number input
patterns, sometimes only one.

\item {\bf Self adjusting receptive fields}

Real brains are highly ``plastic''; regions of the neocortex can learn
to represent entirely different things in reaction to various
changes. If part of the neocortex is damaged, other parts will adjust
to represent what the damaged part used to represent. If a sensory
organ is damaged or changed, the associated part of the neocortex will
adjust to represent something else. The system is self-adjusting.

We want our HTM regions to exhibit the same flexibility. If we
allocate 10,000 columns to a region, it should learn how to best
represent the input with 10,000 columns. If we allocate 20,000
columns, it should learn how best to use that number. If the input
statistics change, the columns should change to best represent the new
reality. In short, the designer of an HTM should be able to allocate
any resources to a region and the region will do the best job it can
of representing the input based on the available columns and input
statistics. The general rule is that with more columns in a region,
each column will represent larger and more detailed patterns in the
input. Typically the columns also will be active less often, yet we
will maintain a relative constant sparsity level.

No new learning rules are required to achieve this highly desirable
goal. By boosting inactive columns, inhibiting neighboring columns to
maintain constant sparsity, establishing minimal thresholds for input,
maintaining a large pool of potential synapses, and adding and
forgetting synapses based on their contribution, the ensemble of
columns will dynamically configure to achieve the desired effect.
\end{enumerate}

\section*{Pattern Memory details}

We can now go through everything the Pattern Memory function does.

\begin{enumerate}
\item Start with an input consisting of a fixed number of bits. These
  bits might represent sensory data or they might come from another
  region lower in the hierarchy.

\item Assign a fixed number of columns to the region receiving this
  input. Each column has an associated dendrite segment. Each dendrite
  segment has a set of potential synapses representing a subset of the
  input bits. Each potential synapse has a permanence value. Based on
  their permanence values, some of the potential synapses will be
  valid.

\item For any given input, determine how many valid synapses on each
  column are connected to active input bits.

\item The number of active synapses is multiplied by a ``boosting''
  factor which is dynamically determined by how often a column is
  active relative to its neighbors.

\item The columns with the highest activations after boosting disable
  all but a fixed percentage of the columns within an inhibition
  radius. The inhibition radius is itself dynamically determined by
  the spread (or ``fan-out'') of input bits. There is now a sparse set
  of active columns.

\item For each of the active columns, we adjust the permanence values
  of all the potential synapses. The permanence values of synapses
  aligned with active input bits are increased. The permanence values
  of synapses aligned with inactive input bits are decreased. The
  changes made to permanence values may change some synapses from
  being valid to not valid, and vice-versa.
\end{enumerate}

\section*{Transition Memory Concepts}

Recall that the Transition Memory learns sequences and makes
predictions. The basic method is that when a cell becomes active, it
forms connections to other cells that were active just prior. Cells
can then predict when they will become active by looking at their
connections. If all the cells do this, collectively they can store and
recall sequences, and they can predict what is likely to happen
next. There is no central storage for a sequence of patterns; instead,
memory is distributed among the individual cells. Because the memory
is distributed, the system is robust to noise and error. Individual
cells can fail, usually with little or no discernible effect.

It is worth noting a few important properties of sparse distributed
representations that the Transition Memory exploits.

Assume we have a hypothetical region that always forms representations
by using 200 active cells out of a total of 10,000 cells (2\% of the
cells are active at any time). How can we remember and recognize a
particular pattern of 200 active cells? A simple way to do this is to
make a list of the 200 active cells we care about. If we see the same
200 cells active again we recognize the pattern. However, what if we
made a list of only 20 of the 200 active cells and ignored the other
180? What would happen? You might think that remembering only 20 cells
would cause lots of errors, that those 20 cells would be active in
many different patterns of 200. But this isn't the case. Because the
patterns are large and sparse (in this example 200 active cells out of
10,000), remembering 20 active cells is almost as good as remembering
all 200. The chance for error in a practical system is exceedingly
small and we have reduced our memory needs considerably.

The cells in an HTM region take advantage of this property. Each of a
cell's dendrite segments has a set of connections to other cells in
the region. A dendrite segment forms these connections as a means of
recognizing the state of the network at some point in time. There may
be hundreds or thousands of active cells nearby but the dendrite
segment only has to connect to 15 or 20 of them. When the dendrite
segment sees 15 of those active cells, it can be fairly certain the
larger pattern is occurring. This technique is called ``sub-sampling''
and is used throughout the HTM algorithms.

Every cell participates in many different distributed patterns and in
many different sequences. A particular cell might be part of dozens or
hundreds of temporal transitions. Therefore every cell has several
dendrite segments, not just one. Ideally a cell would have one
dendrite segment for each pattern of activity it wants to
recognize. Practically though, a dendrite segment can learn
connections for several completely different patterns and still work
well. For example, one segment might learn 20 connections for each of
4 different patterns, for a total of 80 connections. We then set a
threshold so the dendrite segment becomes active when any 15 of its
connections are active. This introduces the possibility for error. It
is possible, by chance, that the dendrite reaches its threshold of 15
active connections by mixing parts of different patterns. However,
this kind of error is very unlikely, again due to the sparseness of
the representations.

Now we can see how a cell with one or two dozen dendrite segments and
a few thousand synapses can recognize hundreds of separate states of
cell activity.

\section*{Transition Memory details}

Here we enumerate the steps performed by the Transition Memory. We start
where the Pattern Memory left off, with a set of active columns
representing the feed-forward input.

\begin{enumerate}
\item For each active column, check for cells in the column that are
  in a predictive state, and activate them. If no cells are in a
  predictive state, activate all the cells in the column. The
  resulting set of active cells is the representation of the input in
  the context of prior input.

\item For every dendrite segment on every cell in the region, count
  how many established synapses are connected to active cells. If the
  number exceeds a threshold, that dendrite segment is marked as
  active. Cells with active dendrite segments are put in the
  predictive state unless they are already active due to feed-forward
  input. Cells with no active dendrites and not active due to
  bottom-up input become or remain inactive. The collection of cells
  now in the predictive state is the prediction of the region.

\item When a dendrite segment becomes active, modify the permanence
  values of all the synapses associated with the segment. For every
  potential synapse on the active dendrite segment, increase the
  permanence of those synapses that are connected to active cells and
  decrement the permanence of those synapses connected to inactive
  cells. These changes to synapse permanence are marked as temporary.
  This modifies the synapses on segments that are already trained
  sufficiently to make the segment active, and thus lead to a
  prediction. 

\item Whenever a cell switches from being inactive to active due to
  feed-forward input, we traverse each potential synapse associated
  with the cell and remove any temporary marks. Thus we update the
  permanence of synapses only if they correctly predicted the
  feed-forward activation of the cell.

\item When a cell switches from either active state to inactive, undo
  any permanence changes marked as temporary for each potential
  synapse on this cell. We don't want to strengthen the permanence of
  synapses that incorrectly predicted the feed-forward activation of a
  cell.
\end{enumerate}

Note that only cells that are active due to feed-forward input
propagate activity {\em within} the region, otherwise predictions
would lead to further predictions. The active cells form the output of a region and
propagate to the {\em next} region in the hierarchy.

\section*{First order versus variable order sequences and prediction}

There is one more major topic to discuss before we end our discussion
on the Pattern and Transition Memory. It may not be of interest to all
readers and it is not needed to understand Chapters 3 and 4.
% TODO: chapter references

What is the effect of having more or fewer cells per column?
Specifically, what happens if we have only one cell per column?

In the example used earlier, we showed that a representation of an
input comprised of 100 active columns with 4 cells per column can be
encoded in $4^{100}$ different ways. Therefore, the same input can
appear in a many contexts without confusion. For example, if input
patterns represent words, then a region can remember many sentences
that use the same words over and over again and not get confused. A
word such as ``dog'' would have a unique representation in different
contexts. This ability permits an HTM region to make what are called
``variable order'' predictions.

A variable order prediction is not based solely on what is currently
happening, but on varying amounts of past context. An HTM region is a
variable order memory.

If we increase to 5 cells per column, the available number of
encodings of any particular input in our example would increase to
$5^{100}$, a huge increase over $4^{100}$. But both these numbers are
so large that for many practical problems the increase in capacity
might not be useful.

However, making the number of cells per column much smaller does make
a big difference.

If we go all the way to one cell per column, we lose the ability to
include context in our representations. An input to a region always
results in the same prediction, regardless of previous activity. With
one cell per column, the memory of an HTM region is a ``first order''
memory; predictions are based only on the current input.

First order prediction is ideally suited for one type of problem that
brains solve: static spatial inference. As stated earlier, a human
exposed to a brief visual image can recognize what the object is even
if the exposure is too short for the eyes to move. With hearing, you
always need to hear a sequence of patterns to recognize what it
is. Vision is usually like that, you usually process a stream of
visual images. But under certain conditions you can recognize an image
with a single exposure.

Temporal and static recognition might appear to require different
inference mechanisms. One requires recognizing sequences of patterns
and making predictions based on variable length context. The other
requires recognizing a static spatial pattern without using temporal
context. An HTM region with multiple cells per column is ideally
suited for recognizing time-based sequences, and an HTM region with
one cell per column is ideally suited to recognizing spatial
patterns. At Numenta, we have performed many experiments using
one-cell-per-column regions applied to vision problems. The details of
these experiments are beyond the scope of this chapter; however we
will cover the important concepts.

If we expose an HTM region to images, the columns in the region learn
to represent common spatial arrangements of pixels. The kind of
patterns learned are similar to what is observed in region V1 in
neocortex (a neocortical region extensively studied in biology),
typically lines and corners at different orientations. By training on
moving images, the HTM region learns transitions of these basic
shapes. For example, a vertical line at one position is often followed
by a vertical line shifted to the left or right. All the commonly
observed transitions of patterns are remembered by the HTM region.

Now what happens if we expose a region to an image of a vertical line
moving to the right? If our region has only one cell per column, it
will predict the line might next appear to the left or to the
right. It can't use the context of knowing where the line was in the
past and therefore know if it is moving left or right. What you find
is that these one-cell-per-column cells behave like ``complex cells''
in the neocortex. The predictive output of such a cell will be active
for a visible line in different positions, regardless of whether the
line is moving left or right or not at all. We have further observed
that a region like this exhibits stability to translation, changes in
scale, etc. while maintaining the ability to distinguish between
different images. This behavior is what is needed for spatial
invariance (recognizing the same pattern in different locations of an
image).

If we now do the same experiment on an HTM region with multiple cells
per column, we find that the cells behave like ``directionally-tuned
complex cells'' in the neocortex. The predictive output of a cell will
be active for a line moving to the left or a line moving to the right,
but not both.

Putting this all together, we make the following hypothesis. The
neocortex has to do both first order and variable order inference and
prediction. There are four or five layers of cells in each region of
the neocortex. The layers differ in several ways but they all have
shared columnar response properties and large horizontal connectivity
within the layer. We speculate that each layer of cells in neocortex
is performing a variation of the HTM inference and learning rules
described in this chapter. The different layers of cells play
different roles. For example it is known from anatomical studies that
layer 6 creates feedback in the hierarchy and layer 5 is involved in
motor behavior. The two primary feed-forward layers of cells are
layers 4 and 3. We speculate that one of the differences between
layers 4 and 3 is that the cells in layer 4 are acting independently,
i.e., one cell per column, whereas the cells in layer 3 are acting as
multiple cells per column. Thus regions in the neocortex near sensory
input have both first order and variable order memory. The first order
sequence memory (roughly corresponding to layer 4 neurons) is useful
in forming representations that are invariant to spatial changes. The
variable order sequence memory (roughly corresponding to layer 3
neurons) is useful for inference and prediction of moving images.

In summary, we hypothesize that the algorithms similar to those
described in this chapter are at work in all layers of neurons in the
neocortex. The layers in the neocortex vary in significant details
which make them play different roles related to feed-forward
vs. feedback, attention, and motor behavior. In regions close to
sensory input, it is useful to have a layer of neurons performing
first order memory as this leads to spatial invariance.

At Numenta, we have experimented with first order (single cell per
column) HTM regions for image recognition problems. We also have
experimented with variable order (multiple cells per column) HTM
regions for recognizing and predicting variable order sequences. In
the future, it would be logical to try to combine these in a single
region and to extend the algorithms to other purposes. However, we
believe many interesting problems can be addressed with the equivalent
of single- layer, multiple-cell-per-column regions, either alone or in
a hierarchy.
