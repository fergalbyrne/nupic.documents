

\chapter{Transition Memory Implementation and Pseudocode}
\label{chapter:transition-memory}
This chapter contains the detailed pseudocode for a first
implementation of the Transition Memory function. The input to this code
is activeColumns(t), as computed by the Pattern Memory. The code
computes the active and predictive state for each cell at the current
timestep, t. The boolean OR of the active and predictive states for
each cell forms the output of the Transition Memory for the next level.

The pseudocode is split into three distinct phases that occur in sequence:
\begin{description}
\item[Phase 1:] compute the active state, activeState(t), for each cell
\item[Phase 2:] compute the predicted state, predictiveState(t), for each cell
\item[Phase 3:] update synapses
\end{description}

\section*{Transition Memory pseudocode: inference alone}

\subsection*{Phase 1}
The first phase calculates the active state for each cell. For each
winning column we determine which cells should become active. If the
bottom-up input was predicted by any cell (i.e., its predictiveState
was 1 due to a sequence segment in the previous time step), then those
cells become active (lines 4--9). If the bottom-up input was unexpected
(i.e., no cells had predictiveState output on), then each cell in the
column becomes active (lines 11--13).

Phase 3 is only required for learning. However, unlike spatial
pooling, Phases 1 and 2 contain some learning-specific operations when
learning is turned on. Since temporal pooling is significantly more
complicated than spatial pooling, we first list the inference-only
version of the Transition Memory, followed by a version that combines
inference and learning. A description of some of the implementation
details, terminology, and supporting routines are at the end of the
chapter, after the pseudocode.

\begin{lstlisting}[numbers=left]
for c in activeColumns(t)

  buPredicted = false
  for i = 0 to cellsPerColumn - 1
    if predictiveState(c, i, t-1) == true then
      s = getActiveSegment(c, i, t-1, activeState)
      if s.sequenceSegment == true then
        buPredicted = true
        activeState(c, i, t) = 1

  if buPredicted == false then
    for i = 0 to cellsPerColumn - 1
      activeState(c, i, t) = 1
\end{lstlisting}

\subsection*{Phase 2}
The second phase calculates the predictive state for each cell. A cell
will turn on its predictiveState if any one of its segments becomes
active, i.e., if enough of its horizontal connections are currently
firing due to feed-forward input.

\begin{lstlisting}[numbers=left,firstnumber=14]
for c, i in cells
  for s in segments(c, i)
    if segmentActive(c, i, s, t) then
      predictiveState(c, i, t) = 1
\end{lstlisting}

\section*{Transition Memory pseudocode: combined inference and learning}

\subsection*{Phase 1}

The first phase calculates the activeState for each cell that is in a
winning column. For those columns, the code further selects one cell
per column as the learning cell (learnState). The logic is as follows:
if the bottom-up input was predicted by any cell (i.e., its
predictiveState output was 1 due to a sequence segment), then those
cells become active (lines 23--27). If that segment became active from
cells chosen with learnState on, this cell is selected as the learning
cell (lines 28--30). If the bottom-up input was not predicted, then
all cells in the become active (lines 32--34). In addition, the best
matching cell is chosen as the learning cell (lines 36--41) and a new
segment is added to that cell.

\begin{lstlisting}[numbers=left,firstnumber=18]
for c in activeColumns(t)

  buPredicted = false
  lcChosen = false
  for i = 0 to cellsPerColumn - 1
    if predictiveState(c, i, t-1) == true then
      s = getActiveSegment(c, i, t-1, activeState)
      if s.sequenceSegment == true then
        buPredicted = true
        activeState(c, i, t) = 1
        if segmentActive(s, t-1, learnState) then
          lcChosen = true
          learnState(c, i, t) = 1

  if buPredicted == false then
    for i = 0 to cellsPerColumn - 1
      activeState(c, i, t) = 1

  if lcChosen == false then
    I,s = getBestMatchingCell(c, t-1)
    learnState(c, i, t) = 1
    sUpdate = getSegmentActiveSynapses (c, i, s, t-1, true)
    sUpdate.sequenceSegment = true
    segmentUpdateList.add(sUpdate)
\end{lstlisting}

\subsection*{Phase 2}

The second phase calculates the predictive state for each cell. A cell
will turn on its predictive state output if one of its segments
becomes active, i.e., if enough of its lateral inputs are currently
active due to feed-forward input. In this case, the cell queues up the
following changes: a) reinforcement of the currently active segment
(lines 47--48), and b) reinforcement of a segment that could have
predicted this activation, i.e., a segment that has a (potentially
weak) match to activity during the previous time step (lines 50--53).

\begin{lstlisting}[numbers=left,firstnumber=42]
for c, i in cells
  for s in segments(c, i)
    if segmentActive(s, t, activeState) then
      predictiveState(c, i, t) = 1

      activeUpdate = getSegmentActiveSynapses(c, i, s, t, false)
      segmentUpdateList.add(activeUpdate)

      predSegment = getBestMatchingSegment(c, i, t-1)
      predUpdate = getSegmentActiveSynapses(c, i, predSegment, t-1, true)
      segmentUpdateList.add(predUpdate)
\end{lstlisting}

\subsection*{Phase 3}

The third and last phase actually carries out learning. In this phase
segment updates that have been queued up are actually implemented once
we get feed-forward input and the cell is chosen as a learning cell
(lines 56--57). Otherwise, if the cell ever stops predicting for any
reason, we negatively reinforce the segments (lines 58--60).

\begin{lstlisting}[numbers=left,firstnumber=54]
for c, i in cells
  if learnState(s, i, t) == 1 then
    adaptSegments (segmentUpdateList(c, i), true)
    segmentUpdateList(c, i).delete()
  else if predictiveState(c, i, t) == 0 and predictiveState(c, i, t-1)==1 then
    adaptSegments (segmentUpdateList(c,i), false)
    segmentUpdateList(c, i).delete()
\end{lstlisting}

\subsection*{Implementation details and terminology}
In this section we describe some of the details of our Transition Memory
implementation and terminology. Each cell is indexed using two
numbers: a column index, c, and a cell index, i. Cells maintain a list
of dendrite segments, where each segment contains a list of synapses
plus a permanence value for each synapse. Changes to a cell's synapses
are marked as temporary until the cell becomes active from
feed-forward input. These temporary changes are maintained in
segmentUpdateList. 

The implementation of potential synapses is different from the
implementation in the Pattern Memory. In the Pattern Memory, the
complete list of potential synapses is represented as an explicit
list. In the Transition Memory, each segment can have its own (possibly
large) list of potential synapses. In practice maintaining a long list
for each segment is computationally expensive and memory
intensive. Therefore in the Transition Memory, we randomly add active
synapses to each segment during learning (controlled by the parameter
newSynapseCount). This optimization has a similar effect to
maintaining the full list of potential synapses, but the list per
segment is far smaller while still maintaining the possibility of
learning new temporal patterns.

The pseudocode also uses a small state machine to keep track of the
cell states at different time steps. We maintain three different
states for each cell. The arrays activeState and predictiveState keep
track of the active and predictive states of each cell at each time
step. The array learnState determines which cell outputs are used
during learning. When an input is unexpected, all the cells in a
particular column become active in the same time step. Only one of
these cells (the cell that best matches the input) has its learnState
turned on. We only add synapses from cells that have learnState set to
one (this avoids overrepresenting a fully active column in dendritic
segments).

The following data structures are used in the Transition Memory pseudocode:

\begin{description}
\item[cell(c,i)] A list of all cells, indexed by i and c.
\item[cellsPerColumn] Number of cells in each column.
\item[activeColumns(t)] List of column indices that are winners due to
  bottom-up input (this is the output of the Pattern Memory).
\item[activeState(c, i, t)] A boolean vector with one number per
  cell. It represents the active state of the column c cell i at time
  t given the current feed-forward input and the past temporal
  context. activeState(c, i, t) is the contribution from column c cell
  i at time t. If 1, the cell has current feed-forward input as well
  as an appropriate temporal context.
\item[predictiveState(c, i, t)] A boolean vector with one number per
  cell. It represents the prediction of the column c cell i at time t,
  given the bottom-up activity of other columns and the past temporal
  context. predictiveState(c, i, t) is the contribution of column c
  cell i at time t. If 1, the cell is predicting feed-forward input in
  the current temporal context.
\item[learnState(c, i, t)] A boolean indicating whether cell i in
  column c is chosen as the cell to learn on.
\item[activationThreshold] A boolean indicating whether cell i in
  column c is chosen as the cell to learn on.
\item[learningRadius] The area around a Transition Memory cell from
  which it can get lateral connections.
\item[initialPerm] Initial permanence value for a synapse.
\item[connectedPerm] If the permanence value for a synapse is greater
  than this value, it is said to be connected.
\item[minThreshold] Minimum segment activity for learning.
\item[newSynapseCount] The maximum number of synapses added to a
  segment during learning.
\item[permanenceInc] Amount permanence values of synapses are
  incremented when activity-based learning occurs.
\item[permanenceDec] Amount permanence values of synapses are
  decremented when activity-based learning occurs.
\item[segmentUpdate] Data structure holding three pieces of
  information required to update a given segment: a) segment index (-1
  if it's a new segment), b) a list of existing active synapses, and
  c) a flag indicating whether this segment should be marked as a
  sequence segment (defaults to false).
\item[segmentUpdateList] A list of segmentUpdate
  structures. segmentUpdateList(c,i) is the list of changes for cell i
  in column c.
\end{description}

The following supporting routines are used in the above code:

\begin{description}
\item[segmentActive(s, t, state)] This routine returns true if the
  number of connected synapses on segment s that are active due to the
  given state at time t is greater than activationThreshold. The
  parameter state can be activeState, or learnState.
\item[getActiveSegment(c, i, t, state)] For the given column c cell i,
  return a segment index such that segmentActive(s,t, state) is
  true. If multiple segments are active, sequence segments are given
  preference. Otherwise, segments with most activity are given
  preference.
\item[getBestMatchingSegment(c, i, t)] For the given column c cell i
  at time t, find the segment with the largest number of active
  synapses. This routine is aggressive in finding the best match. The
  permanence value of synapses is allowed to be below
  connectedPerm. The number of active synapses is allowed to be below
  activationThreshold, but must be above minThreshold. The routine
  returns the segment index. If no segments are found, then an index
  of -1 is returned.
\item[getBestMatchingCell(c)] For the given column, return the cell
  with the best matching segment (as defined above). If no cell has a
  matching segment, then return the cell with the fewest number of
  segments.
\item[getSegmentActiveSynapses(c, i, t, s, newSynapses=false)] Return
  a segmentUpdate data structure containing a list of proposed changes
  to segment s. Let activeSynapses be the list of active synapses
  where the originating cells have their activeState output = 1 at
  time step t. (This list is empty if s = -1 since the segment doesn't
  exist.) newSynapses is an optional argument that defaults to
  false. If newSynapses is true, then newSynapseCount -
  count(activeSynapses) synapses are added to activeSynapses. These
  synapses are randomly chosen from the set of cells that have
  learnState output = 1 at time step t.
\item[adaptSegments(segmentList, positiveReinforcement)] This function
  iterates through a list of segmentUpdate's and reinforces each
  segment. For each segmentUpdate element, the following changes are
  performed. If positiveReinforcement is true then synapses on the
  active list get their permanence counts incremented by
  permanenceInc. All other synapses get their permanence counts
  decremented by permanenceDec. If positiveReinforcement is false,
  then synapses on the active list get their permanence counts
  decremented by permanenceDec. After this step, any synapses in
  segmentUpdate that do yet exist get added with a permanence count of
  initialPerm.
\end{description}

