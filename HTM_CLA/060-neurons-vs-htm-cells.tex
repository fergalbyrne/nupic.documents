\chapter{A Comparison between Biological Neurons and HTM Cells}
\label{appendix:compare-neuron-cell}

\begin{figure}
\resizebox{\textwidth}{!}{\includegraphics{figures/Appendix_A_1.png}}
\end{figure}

The image above shows a picture of a biological neuron on the left, a
simple artificial neuron in the middle, and an HTM neuron or ``cell''
on the right. The purpose of this appendix is to provide a better
understanding of HTM cells and how they work by comparing them to real
neurons and simpler artificial neurons.

Real neurons are tremendously complicated and varied. We will focus on
the most general principles and only those that apply to our
model. Although we ignore many details of real neurons, the cells used
in the HTM cortical learning algorithms are far more realistic than
the artificial neurons used in most neural networks. All the elements
included in HTM cells are necessary for the operation of an HTM
region.

\section*{Biological neurons}
Neurons are the information carrying cells in the brain. The image on
the left above is of a typical excitatory neuron. The visual
appearance of a neuron is dominated by the branching dendrites. All
the excitatory inputs to a neuron are via synapses aligned along the
dendrites. In recent years our knowledge of neurons has advanced
considerably. The biggest change has been in realizing that the
dendrites of a neuron are not just conduits to bring inputs to the
cell body. We now know the dendrites are complex non-linear processing
elements in themselves. The HTM cortical learning algorithms take
advantage of these non-linear properties.

Neurons have several parts.

\subsection*{Cell body}
The cell body is the small volume in the center of the neuron. The
output of the cell, the axon, originates at the cell body. The inputs
to the cell are the synapses aligned along the dendrites which feed to
the cell body.

\subsection*{Proximal Dendrites}
The dendrite branches closest to the cell body are called proximal
dendrites. In the diagram some of the proximal dendrites are marked
with green lines.

Multiple active synapses on proximal dendrites have a roughly linear
additive effect at the cell body. Five active synapses will lead to
roughly five times the depolarization at the cell body compared to one
active synapse. In contrast, if a single synapse is activated
repeatedly by a quick succession of action potentials, the second,
third, and subsequent action potentials have much less effect at the
cell body, than the first.

Therefore, we can say that inputs to the proximal dendrites sum
linearly at the cell body, and that rapid spikes arriving at a single
synapse will have only a slightly larger effect than a single spike.

The feed-forward connections to a region of neocortex preferentially
connect to the proximal dendrites. This has been reported at least for
layer 4 neurons, the primary input layer of neurons in each region.

\subsection*{Distal Dendrites}
The dendrite branches farther from the cell body are called distal
dendrites. In the diagram some of the distal dendrites are marked with
blue lines.

Distal dendrites are thinner than proximal dendrites. They connect to
other dendrites at branches in the dendritic tree and do not connect
directly to the cell body. These differences give distal dendrites
unique electrical and chemical properties. When a single synapse is
activated on a distal dendrite, it has a minimal effect at the cell
body. The depolarization that occurs locally to the synapse weakens by
the time it reaches the cell body. For many years this was viewed as a
mystery. It seemed the distal synapses, which are the majority of
synapses on a neuron, couldn't do much.

We now know that sections of distal dendrites act as semi-independent
processing regions. If enough synapses become active at the same time
within a short distance along the dendrite, they can generate a
dendritic spike that can travel to the cell body with a large
effect. For example, twenty active synapses within 40 $\mu$m of each
other will generate a dendritic spike.

Therefore, we can say that the distal dendrites act like a set of
threshold coincidence detectors.

The synapses formed on distal dendrites are predominantly from other
cells nearby in the region.

The image shows a large dendrite branch extending upwards which is
called the apical dendrite. One theory says that this structure allows
the neuron to locate several distal dendrites in an area where they
can more easily make connections to passing axons. In this
interpretation, the apical dendrite acts as an extension of the cell.

\subsection*{Synapses}
A typical neuron might have several thousand synapses. The large
majority (perhaps 90\%) of these will be on distal dendrites, and the
rest will be on proximal dendrites.

For many years it was assumed that learning involved strengthening and
weakening the effect or ``weight'' of synapses. Although this effect
has been observed, each synapse is somewhat stochastic. When
activated, it will not reliably release a neurotransmitter. Therefore
the algorithms used by the brain cannot depend on precision or
fidelity of individual synapse weights.

Further, we now know that entire synapses form and un-form
rapidly. This flexibility represents a powerful form of learning and
better explains the rapid acquisition of knowledge. A synapse can only
form if an axon and a dendrite are within a certain distance, leading
to the concept of ``potential'' synapses. With these assumptions,
learning occurs largely by forming valid synapses from potential
synapses.

\subsection*{Neuron Output}
The output of a neuron is a spike, or ``action potential,'' which
propagates along the axon. The axon leaves the cell body and almost
always splits in two. One branch travels horizontally making many
connections with other cells nearby. The other branch projects to
other layers of cells or elsewhere in the brain. In the image of the
neuron above, the axon was not visible. We added a line and two arrows
to represent that axon.

Although the actual output of a neuron is always a spike, there are
different views on how to interpret this. The predominant view
(especially in regards to the neocortex) is that the rate of spikes is
what matters. Therefore the output of a cell can be viewed as a scalar
value.

Some neurons also exhibit a ``bursting'' behavior, a short and fast
series of a few spikes that are different than the regular spiking
pattern.

The above description of a neuron is intended to give a brief
introduction to neurons. It focuses on attributes that correspond to
features of HTM cells and leaves out many details. Not all the
features just described are universally accepted. We include them
because they are necessary for our models. What is known about neurons
could easily fill several books, and active research on neurons
continues today.

\section*{Simple artificial neurons}
The middle image at the beginning of this Appendix shows a neuron-like
element used in many classic artificial neural network models. These
artificial neurons have a set of synapses each with a weight. Each
synapse receives a scalar activation, which is multiplied by the
synapse weight. The output of all the synapses is summed in a
non-linear fashion to produce an output of the artificial
neuron. Learning occurs by adjusting the weights of the synapses and
perhaps the non- linear function.

This type of artificial neuron, and variations of it, has proven
useful in many applications as a valuable computational tool. However,
it doesn't capture much of the complexity and processing power of
biological neurons. If we want to understand and model how an ensemble
of real neurons works in the brain we need a more sophisticated neuron
model.

\section*{HTM cells}
In our illustration, the image on the right depicts a cell used in the
HTM cortical learning algorithms. An HTM cell captures many of the
important capabilities of real neurons but also makes several
simplifications.

\subsection*{Proximal Dendrite}
Each HTM cell has a single proximal dendrite. All feed-forward inputs
to the cell are made via synapses (shown as green dots). The activity
of synapses is linearly summed to produce a feed-forward activation
for the cell.

We require that all cells in a column have the same feed-forward
response. In real neurons this would likely be done by a type of
inhibitory cell. In HTMs we simply force all the cells in a column to
share a single proximal dendrite.

To avoid having cells that never win in the competition with
neighboring cells, an HTM cell will boost its feed-forward activation
if it is not winning enough relative to its neighbors. Thus there is a
constant competition between cells. Again, in an HTM we model this as
a competition between columns, not cells. This competition is not
illustrated in the diagram.

Finally, the proximal dendrite has an associated set of potential
synapses which is a subset of all the inputs to a region. As the cell
learns, it increases or decreases the ``permanence'' value of all the
potential synapses on the proximal dendrite. Only those potential
synapses that are above a threshold are valid.

As mentioned earlier, the concept of potential synapses comes from
biology where it refers to axons and dendrites that are close enough
to form a synapse. We extend this concept to a larger set of potential
connections for an HTM cell. Dendrites and axons on biological neurons
can grow and retract as learning occurs and therefore the set of
potential synapses changes with growth. By making the set of potential
synapses on an HTM cell large, we roughly achieve the same result as
axon and dendrite growth. The set of potential synapses is not shown.

The combination of competition between columns, learning from a set of
potential synapses, and boosting underutilized columns gives a region
of HTM neurons a powerful plasticity also seen in brains. An HTM
region will automatically adjust what each column represents (via
changes to the synapses on the proximal dendrites) if the input
changes, or the number of columns increases or decreases.

\subsection*{Distal Dendrites}
Each HTM cell maintains a list of distal dendrite segments. Each
segment acts like a threshold detector. If the number of active
synapses on any segment (shown as blue dots on the earlier diagram) is
above a threshold, the segment becomes active, and the associated cell
enters the predictive state. The predictive state of a cell is the OR
of the activations of its segments.

A dendrite segment remembers the state of the region by forming
connections to cells that were active together at a point in time. The
segment remembers a state that precedes the cell becoming active due
to feed-forward input. Thus the segment is looking for a state that
predicts that its cell will become active. A typical threshold for a
dendrite segment is 15. If 15 valid synapses on a segment are active
at once, the dendrite becomes active. There might be hundreds or
thousands of cells active nearby, but connecting to only 15 is
sufficient to recognize the larger pattern.

Each distal dendrite segment also has an associated set of potential
synapses. The set of potential synapses is a subset of all the cells
in a region. As the segment learns, it increases or decreases the
permanence value of all its potential synapses. Only those potential
synapses that are above a threshold are valid.

In one implementation, we use a fixed number of dendrite segments per
cell. In another implementation, we add and delete segments while
training. Both methods can work. If we have a fixed number of dendrite
segments per cell, it is possible to store several different sets of
synapses on the same segment. For example, say we have 20 valid
synapses on a segment and a threshold of 15. (In general we want the
threshold to be less than the number of synapses to improve noise
immunity.) The segment can now recognize one particular state of the
cells nearby. What would happen if we added another 20 synapses to the
same segment representing an entirely different state of cells nearby?
It introduces the possibility of error because the segment could add 8
active synapses from one pattern and 7 active synapses from the other
and become active incorrectly. We have found experimentally that up to
20 different patterns can be stored on one segment before errors
occur. Therefore an HTM cell with a dozen dendrite segments can
participate in many different predictions.

\subsection*{Synapses}

Synapses on an HTM cell have a binary weight. There is nothing in the
HTM model that precludes scalar synapse weights, but due to the use of
sparse distributed patterns we have not yet had a need to use scalar
weights.

However, synapses on an HTM cell have a scalar value called
``permanence'' which is adjusted during learning. A 0.0 permanence value
represents a potential synapse which is not valid and has not
progressed at all towards becoming a valid synapse. A permanence value
above a threshold (typically 0.2) represents a synapse that has just
connected but could easily be un-connected. A high permanence value,
for example 0.9, represents a synapse that is connected and cannot
easily be un- connected.

The number of valid synapses on the proximal and distal dendrite
segments of an HTM cell is not fixed. It changes as the cell is
exposed to patterns. For example, the number of valid synapses on the
distal dendrites is dependent on the temporal structure of the
data. If there are no persistent temporal patterns in the input to the
region, then all the synapses on distal segments would have low
permanence values and very few synapses would be valid. If there is a
lot of temporal structure in the input stream, then we will find many
valid synapses with high permanence.

\subsection*{Cell Output}

An HTM cell has two different binary outputs: 1) the cell is active
due to feed- forward input (via the proximal dendrite), and 2) the
cell is active due to lateral connections (via the distal dendrite
segments). The former is called the ``active state'' and the latter is
called the ``predictive state.''

In the earlier diagram, the two outputs are represented by the two
lines exiting the square cell body. The left line is the feed-forward
active state, while the right line is the predictive state.

Only the feed-forward active state is connected to other cells in the
region, ensuring that predictions are always based on the current
input (plus context). We don't want to make predictions based on
predictions. If we did, almost all the cells in the region would be in
the predictive state after a few iterations.

The output of the region is a vector representing the state of all the
cells. This vector becomes the input to the next region of the
hierarchy if there is one. 

\section*{Suggested reading}

We are often asked to suggest reading materials to learn more about
neuroscience. The field of neuroscience is so large that a general
introduction requires looking at many different sources. New findings
are published in academic journals which are both hard to read and
hard to get access to if you don't have a university affiliation.

Here are two readily available books that a dedicated reader might
want to look at which are relevant to the topics in this appendix.

\begin{quote}
Stuart, Greg, Spruston, Nelson, H\"ausser, Michael, {\em Dendrites},
second edition (New York: Oxford University Press, 2008)
\end{quote}

This book is a good source on everything about dendrites. Chapter~16
discusses the non-linear properties of dendrite segments used in the
HTM cortical learning algorithms. It is written by Bartlett Mel who
has done much of the thinking in this field.

\begin{quote}
Mountcastle, Vernon B.~{\em Perceptual Neuroscience: The Cerebral
  Cortex} (Cambridge, Mass.: Harvard University Press, 1998)
\end{quote}

This book is a good introduction to everything about the
neocortex. Several of the chapters discuss cell types and their
connections. You can get a good sense of cortical neurons and their
connections, although it is too old to cover the latest knowledge of
dendrite properties.
