\chapter{A Comparison of Layers in the Neocortex and an HTM Region}
\label{appendix:compare-layer-region}

This appendix describes the relationship between an HTM region and a
region of the biological neocortex.

Specifically, the appendix covers how the HTM cortical learning
algorithm, with its columns and cells, relates to the layered and
columnar architecture of the neocortex. Many people are confused by
the concept of ``layers'' in the neocortex and how it relates to an
HTM layer. Hopefully this appendix will resolve this confusion as well
as provide more insight into the biology underlying the HTM cortical
learning algorithm.

\section*{Circuitry of the neocortex}

The human neocortex is a sheet of neural tissue approximately 1,000
cm$^2$ in area and 2 mm thick. To visualize this sheet, think of a
cloth dinner napkin, which is a reasonable approximation of the area
and thickness of the neocortex. The neocortex is divided into dozens
of functional regions, some related to vision, others to audition, and
others to language, etc. Viewed under a microscope, the physical
characteristics of the different regions look remarkably similar.

There are several organizing principles seen in each region throughout
the neocortex.

\begin{figure}
\resizebox{\textwidth}{!}{\includegraphics{figures/Appendix_B_1.png}}
\end{figure}

\subsection*{Layers}
The neocortex is generally said to have six layers. Five of the layers
contain cells and one layer is mostly connections. The layers were
discovered over one hundred years ago with the advent of staining
techniques. The image above (from Cajal) shows a small slice of
neocortex exposed using three different staining methods. The vertical
axis spans the thickness of the neocortex, approximately 2 mm. The
left side of the image indicates the six layers. Layer 1, at the top,
is the non-cellular level. The ``WM'' at the bottom indicates the
beginning of the white matter, where axons from cells travel to other
parts of the neocortex and other parts of the brain.

The right side of the image is a stain that shows only myelinated
axons.  (Myelination is a fatty sheath that covers some but not all
axons.) In this part of the image you can see two of the main
organizing principles of the neocortex, layers and columns. Most axons
split in two immediately after leaving the body of the neuron. One
branch will travel mostly horizontally and the other branch will
travel mostly vertically. The horizontal branch makes a large number
of connections to other cells in the same or nearby layer, thus the
layers become visible in stains such as this. Bear in mind that this
is a drawing of a slice of neocortex. Most of the axons are coming in
and out of the plane of the image so the axons are longer than they
appear in the image. It has been estimated that there are between 2
and 4 kilometers of axons and dendrites in every cubic millimeter of
neocortex.

The middle section of the image is a stain that shows neuron bodies,
but does not show any dendrites or axons. You can see that the size
and density of the neurons also varies by layer. There is only a
little indication of columns in this particular image. You might
notice that there are some neurons in layer 1. The number of layer 1
neurons is so small that the layer is still referred to as a
non-cellular layer. Neuroscientists have estimated that there is
somewhere around 100,000 neurons in a cubic millimeter of neocortex.

The left part of the image is a stain that shows the body, axons, and
dendrites of just a few neurons. You can see that the size of the
dendrite ``arbors'' varies significantly in cells in different
layers. Also visible are some ``apical dendrites'' that rise from the
cell body making connections in other layers. The presence and
destination of apical dendrites is specific to each layer.

In short, the layered and columnar organization of the neocortex
becomes evident when the neural tissue is stained and viewed under a
microscope.

\subsection*{Variations of layers in different regions}
There is variation in the thickness of the layers in different regions
of the neocortex and some disagreement over the number of layers. The
variations depend on what animal is being studied, what region is
being looked at, and who is doing the looking. For example, in the
image above, layer 2 and layer 3 look easily distinguished, but
generally this is not the case. Some scientists report that they
cannot distinguish the two layers in the regions they study, so often
layer 2 and layer 3 are grouped together and called ``layer 2/3.''
Other scientists go the opposite direction, defining sub-layers such
as 3A and 3B.


Layer 4 is most well defined in those neocortical regions which are
closest to the sensory organs. While in some animals (for example
humans and monkeys), layer 4 in the first vision region is clearly
subdivided. In other animals it is not subdivided. Layer 4 mostly
disappears in regions hierarchically far from the sensory organs.

\subsection*{Columns}
The second major organizing principle of the neocortex is
columns. Some columnar organization is visible in stained images, but
most of the evidence for columns is based on how cells respond to
different inputs.

When scientists use probes to see what makes neurons become active,
they find that neurons that are vertically aligned, across different
layers, respond to roughly the same input.

\begin{figure}
\resizebox{\textwidth}{!}{\includegraphics{figures/Appendix_B_2.png}}
\end{figure}

This drawing illustrates some of the response properties of cells in
V1, the first cortical region to process information from the retina.

One of the first discoveries was that most cells in V1 respond to
lines or edges at different orientations at specific areas of the
retina. Cells that are vertically aligned in columns all respond to
edges with the same orientation. If you look carefully, you will see
that the drawing shows a set of small lines at different orientations
arrayed across the top of the section. These lines indicate what line
orientation cells at that location respond to. Cells that are
vertically aligned (within the thin vertical stripes) respond to the
lines of the same orientation.

There are several other columnar properties seen in V1, two of which
are shown in the drawing. There are ``ocular dominance columns'' where
cells respond to similar combinations of left and right eye
influence. And there are ``blobs'' where cells are primarily color
sensitive. The ocular dominance columns are the larger blocks in the
diagram. Each ocular dominance column includes a set of orientation
columns. The ``blobs'' are the dark ovals.

The general rule for neocortex is that several different response
properties are overlaid on one another, such as orientation and ocular
dominance. As you move horizontally across the cortical surface, the
combination of response properties exhibited by cells
changes. However, vertically aligned neurons share the same set of
response properties. This vertical alignment is true in auditory,
visual, and somatosensory areas. There is some debate amongst
neuroscientists whether this is true everywhere in the neocortex but
it appears to be true in most areas if not all.

\subsection*{Mini-columns}
The smallest columnar structure in the neocortex is the
mini-column. Mini-columns are about 30um in diameter and contain
80-100 neurons across all five cellular layers. The entire neocortex
is composed of mini-columns. You can visualize them as tiny pieces of
spaghetti stacked side by side. There are tiny gaps with few cells
between the mini-columns sometimes making them visible in stained
images.

\begin{figure}
\resizebox{\textwidth}{!}{\includegraphics{figures/Appendix_B_3.png}}
\end{figure}

On the left is a stained image that shows neuron cell bodies in part
of a neocortical slice. The vertical structure of mini-columns is
evident in this image. On the right is a conceptual drawing of a
mini-column (from Peters and Yilmez). In reality is skinnier than
this. Note there are multiple neurons in each layer in the column. All
the neurons in a mini-column will respond to similar inputs. For
example, in the drawing of a section of V1 shown previously, a
mini-column will contain cells that respond to lines of a particular
orientation with a particular ocular dominance preference. The cells
in an adjacent mini-column might respond to a slightly different line
orientation or different ocular dominance preference.

Inhibitory neurons play an essential role is defining
mini-columns. They are not visible in the image or drawing but
inhibitory neurons send axons in a straight path between mini-columns
partially giving them their physical separation. The inhibitory
neurons are also believed to help force all the cells in the
mini-column to respond to similar inputs.

The mini-column is the prototype for the column used in the HTM
cortical learning algorithm.

\subsection*{An exception to columnar responses}
There is a one exception to columnar responses that is relevant to the
HTM cortical learning algorithms. Usually scientists find what a cell
responds to by exposing an experimental animal to a simple
stimulus. For example, they might show an animal a single line in a
small part of the visual space to determine the response properties of
cells in V1. When using simple inputs, researchers find that cells
always will respond to the same input. However, if the simple input is
embedded in a video of a natural scene, cells become more selective. A
cell that reliably responds to an isolated vertical line will not
always respond when the vertical line is embedded in a complex moving
image of a natural scene.

In the HTM cortical learning algorithm, all HTM cells in a column
share the same feed-forward response properties, but in a learned
temporal sequence, only one of the cells in an HTM column becomes
active. This mechanism is the means of representing variable order
sequences and is analogous to the property just described for
neurons. A simple input with no context will cause all the cells in a
column to become active. The same input within a learned sequence will
cause just one cell to become active.

We are not suggesting that only one neuron within a mini-column will
be active at once. The HTM cortical learning algorithm suggests that
within a column, all the neurons within a layer would be active for an
unanticipated input and a subset of the neurons would be active for an
anticipated input.

\section*{Why are there layers and columns?}

No one knows for certain why there are layers and why there are
columns in the neocortex. HTM theory, however, proposes an answer. The
HTM cortical learning algorithm shows that a layer of cells organized
in columns can be a high capacity memory of variable order state
transitions. Stated more simply, a layer of cells can learn a lot of
sequences. Columns of cells that share the same feed-forward response
are the key mechanism for learning variable-order transitions.

This hypothesis explains why columns are necessary, but what about the
five layers? If a single cortical layer can learn sequences and make
predictions, why do we see five layers in the neocortex?

We propose that the different layers observed in the neocortex are all
learning sequences using the same basic mechanism but the sequences
learned in each layer are used in different ways. There is a lot we
don't understand about this, but we can describe the general
idea. Before we do, it will be helpful to describe what the neurons in
each layer connect to.

\begin{figure}
\resizebox{\textwidth}{!}{\includegraphics{figures/Appendix_B_4.png}}
\end{figure}

The above diagram illustrates two neocortical regions and the major
connections between them. These connections are seen throughout the
neocortex where two regions project to each other. The box on the left
represents a cortical region that is hierarchically lower than the
region (box) on the right, so feed-forward information goes from left
to right in the diagram. The down arrow projects to other areas of the
brain. Feedback information goes from right to left. Each region is
divided into layers. Layers 2 and 3 are shown together as layer 2/3.

The colored lines represent the output of neurons in the different
layers. These are bundles of axons originating from the neurons in the
layer. Recall that axons immediately split in two. One branch spreads
horizontally within the region, primarily within the same layer. Thus
all the cells in each layer are highly interconnected. The neurons and
horizontal connections are not shown in the diagram.

There are two feed-forward pathways, a direct path shown in orange and
an indirect path shown in green. Layer 4 is the primary feed-forward
input layer and receives input from both feed-forward pathways. Layer
4 projects to layer 3.

Layer 3 is also the origin of the direct feed-forward pathway. So the
direct forward pathway is limited to layer 4 and layer 3.

Some feed-forward connections skip layer 4 and go directly to layer
3. And, as mentioned above, layer 4 disappears in regions far from
sensory input. At that point, the direct forward pathway is just from
layer 3 to layer 3 in the next region.

The second feed-forward pathway (shown in green) originates in layer
5. Layer 3 cells make a connection to layer 5 cells as they pass on
their way to the next region. After exiting the cortical sheet, the
axons from layer 5 cells split again. One branch projects to
sub-cortical areas of the brain that are involved in motor
generation. These axons are believed to be motor commands (shown as
the down facing arrow). The other branch projects to a part of the
brain called the thalamus which acts as a gate. The thalamus either
passes the information onto the next region or blocks it.

Finally, the primary feedback pathway, shown in yellow, starts in
layer 6 and projects to layer 1. Cells in layers 2, 3, and 5 connect
to layer 1 via their apical dendrites (not shown). Layer 6 receives
input from layer 5.

This description is a limited summary of what is known about layer to
layer connections. But it is sufficient to understand our hypothesis
about why there are multiple layers if all the layers are learning
sequences.

\section*{Hypothesis on what the different layers do}

We propose that layers 3, 4 and 5 are all feed-forward layers and are
all learning sequences. Layer 4 is learning first order
sequences. Layer 3 is learning variable order sequences. And layer 5
is learning variable order sequences with timing. Let's look at each
of these in more detail.

\subsubsection*{Layer 4}
It is easy to learn first order sequences using the HTM cortical
learning algorithm. If we don't force the cells in a column to inhibit
each other, that is, the cells in a column don't differentiate in the
context of prior inputs, then first order learning will occur. In the
neocortex this would likely be accomplished by removing an inhibitory
effect between cells in the same column. In our computer models of the
HTM cortical learning algorithm, we just assign one cell per column,
which produces a similar result.

First order sequences are what are needed to form invariant
representations for spatial transformations of an input. In vision,
for example, x-y translation, scale, and rotation are all spatial
transformations. When an HTM region with first order memory is trained
on moving objects, it learns that different spatial patterns are
equivalent. The resulting HTM cells will behave like what are called
``complex cells'' in the neocortex. The HTM cells will stay active (in
the predictive state) over a range of spatial transformations.

At Numenta we have done vision experiments that verify this mechanism
works as expected, and that some spatial invariance is achieved within
each level. The details of these experiments are beyond the scope of
this appendix.

Learning first order sequences in layer 4 is consistent with finding
complex cells in layer 4, and for explaining why layer 4 disappears in
higher regions of neocortex. As you ascend the hierarchy at some point
it will no longer be possible to learn further spatial invariances as
the representations will already be invariant to them.

\subsubsection*{Layer 3}
Layer 3 is closest to the HTM cortical learning algorithm that we
described in Chapter 2. It learns variable order sequences and forms
predictions that are more stable than its input. Layer 3 always
projects to the next region in the hierarchy and therefore leads to
increased temporal stability within the hierarchy. Variable order
sequence memory leads to neurons called ``directionally-tuned complex
cells'' which are first observed in layer 3. Directionally-tuned
complex cells differentiate by temporal context, such as a line moving
left vs. a line moving right.

\subsubsection*{Layer 5}
The final feed-forward layer is layer 5. We propose that layer 5 is
similar to layer 3 with three differences. The first difference is
that layer 5 adds a concept of timing. Layer 3 predicts ``what'' will
happen next, but it doesn't tell you ``when'' it will happen. However,
many tasks require timing such as recognizing spoken words in which
the relative timing between sounds is important. Motor behavior is
another example; coordinated timing between muscle activations is
essential. We propose that layer 5 neurons predict the next state only
after the expected time. There are several biological details that
support this hypothesis. One is that layer 5 is the motor output layer
of the neocortex. Another is that layer 5 receives input from layer 1
that originates in a part of the thalamus (not shown in the
diagram). We propose that this information is how time is encoded and
distributed to many cells via a thalamic input to layer 1 (not shown
in the diagram).

The second difference between layer 3 and layer 5 is that we want
layer 3 to make predictions as a far into the future as possible,
gaining temporal stability. The HTM cortical learning algorithm
described in Chapter 2 does this. In contrast, we only want layer 5 to
predict the next element (at a specific time). We have not modeled
this difference but it would naturally occur if transitions were
always stored with an associated time.

The third difference between layer 3 and layer 5 can be seen in the
diagram. The output of layer 5 always projects to sub-cortical motor
centers, and the feed-forward path is gated by the thalamus. The
output of layer 5 is sometimes passed to the next region and sometimes
it is blocked. We (and others) propose this gating is related to
covert attention (covert attention is when you attend to an input
without motor behavior).

In summary, layer 5 combines specific timing, attention, and motor
behavior. There are many mysteries relating to how these play
together. The point we want to make is that a variation of the HTM
cortical learning algorithm could easily incorporate specific timing
and justify a separate layer in the cortex.

\subsubsection*{Layer 2 and layer 6}
Layer 6 is the origin of axons that feed back to lower regions. Much
less is known about layer 2. As mentioned above, the very existence of
layer 2 as unique from layer 3 is sometimes debated. We won't have
further to say about this question now other than to point out that
layers 2 and 6, like all the other layers, exhibit the pattern of
massive horizontal connections and columnar response properties, so we
propose that they, too, are running a variant of the HTM cortical
learning algorithm.

\subsection*{What does an HTM region correspond to in the neocortex?}
We have implemented the HTM cortical learning algorithm in two
flavors, one with multiple cells per column for variable order memory,
and one with a single cell per column for first order memory. We
believe these two flavors correspond to layer 3 and layer 4 in the
neocortex. We have not attempted to combine these two variants in a
single HTM region.

Although the HTM cortical learning algorithm (with multiple cells per
column) is closest to layer 3 in the neocortex, we have flexibility in
our models that the brain doesn't have. Therefore we can create hybrid
cellular layers that don't correspond to specific neocortical
layers. For example, in our model we know the order in which synapses
are formed on dendrite segments. We can use this information to
extract what is predicted to happen next from the more general
prediction of all the things that will happen in the future. We can
probably add specific timing in the same way. Therefore it should be
possible to create a single layer HTM region that combines the
functions of layer 3 and layer 5.

\section*{Summary}


The HTM cortical learning algorithm embodies what we believe is a
basic building block of neural organization in the neocortex. It shows
how a layer of horizontally-connected neurons learns sequences of
sparse distributed representations. Variations of the HTM cortical
learning algorithm are used in different layers of the neocortex for
related, but different purposes.

We propose that feed-forward input to a neocortical region, whether to
layer 4 or layer 3, projects predominantly to proximal dendrites,
which with the assistance of inhibitory cells, creates a sparse
distributed representation of the input. We propose that cells in
layers 2, 3, 4, 5, and 6 share this sparse distributed
representation. This is accomplished by forcing all cells in a column
that spans the layers to respond to the same feed-forward input.

We propose that layer 4 cells, when they are present, use the HTM
cortical learning algorithm to learn first-order temporal transitions
which make representations that are invariant to spatial
transformations. Layer 3 cells use the HTM cortical learning algorithm
to learn variable-order temporal transitions and form stable
representations that are passed up the cortical hierarchy. Layer 5
cells learn variable-order transitions with timing. We don't have
specific proposals for layer 2 and layer 6. However, due to the
typical horizontal connectivity in these layers it is likely they,
too, are learning some form of sequence memory.
