% TODO: figures and chapters need independent label indexing

\documentclass{report}
\usepackage{array}
\usepackage{listings}
\usepackage{graphics}
\usepackage{setspace}

\setlength{\textwidth}{6.5in}
\setlength{\oddsidemargin}{0in}
\setlength{\evensidemargin}{0in}

\newlength{\drop}

\newcommand*{\titleHTMCLA}{\begingroup%
\drop=0.08\textheight
\vspace*{\drop}
\begin{center}
{\Huge Hierarchical Temporal Memory\par}
{\Large including HTM Cortical Learning Algorithms\par}
\vfill
{\Large Version 0.3, August 24, 2014\par}
{\Large \textcopyright{} Numenta, Inc. 2011\par}
{\Large \textcopyright{} Numenta and the NuPIC Community 2014\par}
\end{center}
\vfill
\endgroup}


\begin{document}
\lstset{language=Python}        % TODO: good choice for pseudocode


% TODO: Numenta logo

\pagestyle{empty}
\titleHTMCLA

Use of Numenta's software and intellectual property, including the
ideas contained in this document, are free for non-commercial research
purposes. For details, see
http://www.numenta.com/about-numenta/licensing.php.

\clearpage

\subsection*{Read This First!}

This is a draft version of this document. There are several things
missing that you should be aware of.

\subsection*{What IS in this document:}
This document describes in detail new algorithms for learning and
prediction developed by Numenta. The new algorithms are described in
sufficient detail that a programmer can understand and implement them
if desired. It starts with an introductory chapter. If you have been
following Numenta and have read some of our past white papers, the
material in the introductory chapter will be familiar. The other
material is new.

There are several topics related to the implementation of these new
algorithms that did not make it into this early draft.

\subsection*{What is NOT in this document:}

\begin{itemize}

\item Although most aspects of the algorithms have been implemented
and tested in software, none of the test results are currently
included.

\item The algorithms are capable of on-line learning. A few details
needed to fully implement on-line learning in some rarer cases are not
described.

\end{itemize}

We are making this document available in its current form because we
think the algorithms will be of interest to others. The missing
components of the document should not impede understanding and
experimenting with the algorithms by motivated researchers. We will
revise this document regularly to reflect our progress.

\tableofcontents

% Table of Contents
% Preface 4
% Chapter 1: HTM Overview 7
% Chapter 2: HTM Cortical Learning Algorithms 19
% Chapter 3: Spatial Pooling Implementation and Pseudocode 34
% Chapter 4: Temporal Pooling Implementation and Pseudocode 39
% Appendix A: A Comparison between Biological Neurons 47 and HTM Cells
% Appendix B: A Comparison of Layers in the Neocortex and 54 an HTM Region
% Glossary 65

\chapter*{Preface}
\addcontentsline{toc}{chapter}{Preface}

There are many things humans find easy to do that computers are
currently unable to do. Tasks such as visual pattern recognition,
understanding spoken language, recognizing and manipulating objects by
touch, and navigating in a complex world are easy for humans. Yet
despite decades of research, we have few viable algorithms for
achieving human-like performance on a computer.

In humans, these capabilities are largely performed by the
neocortex. Hierarchical Temporal Memory (HTM) is a technology modeled
on how the neocortex performs these functions. HTM offers the promise
of building machines that approach or exceed human level performance
for many cognitive tasks.

This document describes HTM technology. Chapter~\ref{chapter:overview}
provides a broad overview of HTM, outlining the importance of
hierarchical organization, sparse distributed representations, and
learning time-based transitions. Chapter~\ref{chapter:learning}
describes the HTM cortical learning algorithms in
detail. Chapters~\ref{chapter:pattern-memory} and ~\ref{chapter:transition-memory}
provide pseudocode for the HTM learning
algorithms divided in two parts called pattern memory and transition
memory. After reading chapters~\ref{chapter:learning} through
\ref{chapter:transition-memory}, experienced software engineers should
be able to reproduce and experiment with the algorithms. Hopefully,
some readers will go further and extend our work.

\subsection*{Intended audience}

This document is intended for a technically educated audience. While
we don't assume prior knowledge of neuroscience, we do assume you can
understand mathematical and computer science concepts. We've written
this document such that it could be used as assigned reading in a
class. Our primary imagined reader is a student in computer science or
cognitive science, or a software developer who is interested in
building artificial cognitive systems that work on the same principles
as the human brain.  Non-technical readers can still benefit from
certain parts of the document, particularly
Chapter~\ref{chapter:overview}: HTM Overview.

\subsection*{Relation to previous documents}

Parts of HTM theory are described in the 2004 book {\em On
  Intelligence}, in white papers published by Numenta, and in peer
reviewed papers written by Numenta employees. We don't assume you've
read any of this prior material, much of which has been incorporated
and updated in this volume. Note that the HTM learning algorithms
described in Chapters 2-4 have not been previously published. The new
algorithms replace our first generation algorithms, called Zeta 1. For
a short time, we called the new algorithms ``Fixed-density Distributed
Representations'', or ``FDR'', but we are no longer using this
terminology. We call the new algorithms the HTM Cortical Learning
Algorithms, or sometimes just the HTM Learning Algorithms.

We encourage you to read {\em On Intelligence}, written by Numenta
co-founder Jeff Hawkins with Sandra Blakeslee. Although the book does
not mention HTM by name, it provides an easy-to-read, non-technical
explanation of HTM theory and the neuroscience behind it. At the time
{\em On Intelligence} was written, we understood the basic
principles underlying HTM but we didn't know how to implement those
principles algorithmically. You can think of this document as
continuing the work started in {\em On Intelligence}.

\subsection*{About Numenta}

Numenta, Inc. (www.numenta.com) was formed in 2005 to develop HTM
technology for both commercial and scientific use. To achieve this
goal we are fully documenting our progress and discoveries. We also
publish our software in a form that other people can use for both
research and commercial development. We have structured our software
to encourage the emergence of an independent, application developer
community. Use of Numenta's software and intellectual property is free
for research purposes. We will generate revenue by selling support,
licensing software, and licensing intellectual property for commercial
deployments. We always will seek to make our developer partners
successful, as well as be successful ourselves.

Numenta is based in Redwood City, California. It is privately funded.

\subsection*{About NuPIC}

In 2013, Numenta created an Open Source project around its Numenta Platform
for Intelligent Computing (NuPIC) software. The project has attracted many contributions
and there are now over 1000 subscribers to thw mailing lists. The project is managed
by Numenta's Matt Taylor. For details, visit http://www.numenta.org

\subsection*{About the authors}

This document is a collaborative effort by the employees of
Numenta and the NuPIC community. The names of the principal authors for each section are
listed in the revision history.

\pagebreak
\subsection*{Revision history}

We note in the table below major changes between versions. Minor
changes such as small clarifications or formatting changes are not
noted.

\vspace{5mm}
\begin{tabular}{|p{0.1\textwidth}|p{0.2\textwidth}|>{\raggedright}p{0.45\textwidth}|>{\raggedright\arraybackslash}p{0.25\textwidth}|}
\hline
Version & Date & Changes & Principal Authors \\
\hline
0.1 & Nov 9, 2010 & 1. Preface, Chapters 1,2,3,4, and Glossary: first release & Jeff~Hawkins, Subutai~Ahmad, Donna~Dubinsky \\
\hline
0.1.1 & Nov 23, 2010 & 1. Chapter 1: the Regions section was edited to clarify terminology, such as levels, columns, and layers & Hawkins \& Dubinsky \\
 & & 2. Appendix A: first release & Hawkins \\
\hline
0.2 & Dec 10, 2010 & 1. Chapter 2: various clarifications & Hawkins \\
 & & 2. Chapter 4: updated line references; code changes in lines 37 and 39 & Ahmad \\
 & & 3. Appendix B: first release & Hawkins \\
\hline
0.2.1 & Sep 12, 2011 & 1. Read This First: Removed reference to 2010 & \\
 & & Preface: Removed Software Release section & \\
\hline
0.3 & Aug 22, 2014 & 1. Preface, Chapters 1,2,3,4, and Glossary: updates, corrections, new theory & 
Hawkins, Ahmad (theory), Fergal~Byrne (theory), Chetan~Surpur (algorithms)\\
\hline
\end{tabular}

\chapter{HTM Overview}
\label{chapter:overview}

Hierarchical Temporal Memory (HTM) is a machine learning technology
that aims to capture the structural and algorithmic properties of the
neocortex.

The neocortex is the seat of intelligent thought in the mammalian
brain. High level vision, hearing, touch, movement, language, and
planning are all performed by the neocortex. Given such a diverse
suite of cognitive functions, you might expect the neocortex to
implement an equally diverse suite of specialized neural
algorithms. This is not the case. The neocortex displays a remarkably
uniform pattern of neural circuitry. The biological evidence suggests
that the neocortex implements a common set of algorithms to perform
many different intelligence functions.

HTM provides a theoretical framework for understanding the neocortex
and its many capabilities. To date we have implemented a small subset
of this theoretical framework. Over time, more and more of the theory
will be implemented. Today we believe we have implemented a sufficient
subset of what the neocortex does to be of commercial and scientific
value.

Programming HTMs is unlike programming traditional computers. With
today's computers, programmers create specific programs to solve
specific problems. By contrast, HTMs are trained through exposure to a
stream of sensory data. The HTM's capabilities are determined largely
by what it has been exposed to.

HTMs can be viewed as a type of neural network. By definition, any
system that tries to model the architectural details of the neocortex
is a neural network. However, on its own, the term ``neural network'' is
not very useful because it has been applied to a large variety of
systems. HTMs model neurons (called cells when referring to HTM),
are arranged in columns, in layers, in regions, and in a
hierarchy. The details matter, and in this regard HTMs are a new form
of neural network.

As the name implies, HTM is fundamentally a memory based system. HTM
networks are trained on lots of time varying data, and rely on storing
a large set of patterns and sequences. The way data is stored and
accessed is logically different from the standard model used by
programmers today. Classic computer memory has a flat organization and
does not have an inherent notion of time. A programmer can implement
any kind of data organization and structure on top of the flat
computer memory. They have control over how and where information is
stored. By contrast, HTM memory is more restrictive. HTM memory has a
hierarchical organization and is inherently time based. Information is
always stored in a distributed fashion. A user of an HTM specifies the
size of the hierarchy and what to train the system on, but the HTM
controls where and how information is stored. Although HTM networks
are substantially different than classic computing, we can use general
purpose computers to model them as long as we incorporate the key
functions of hierarchy, time and sparse distributed representations
(described in detail later). In a process which has already begun, specialized
hardware is being created to generate purpose-built HTM networks.

In this document, we often illustrate HTM properties and principles
using examples drawn from human vision, touch, hearing, language, and
behavior. Such examples are useful because they are intuitive and
easily grasped. However, it is important to keep in mind that HTM
capabilities are general. They can just as easily be exposed to
non-human sensory input streams, such as radar and infrared, or to
purely informational input streams such as financial market data,
weather data, Web traffic patterns, or text. HTMs are learning and
prediction machines that can be applied to many types of problems.

\section*{HTM principles}

In this section, we cover some of the core principles of HTM: why
hierarchical organization is important, how HTM regions are
structured, why data is stored as sparse distributed representations,
and why time-based information is critical.

\subsection*{Hierarchy}

An HTM network consists of regions arranged in a hierarchy. The region
is the main unit of memory and prediction in an HTM, and will be
discussed in detail in the next section. Typically, each HTM region
represents one level in the hierarchy. As you ascend the hierarchy
there is always convergence, multiple elements in a child region
converge onto an element in a parent region. However, due to feedback
connections, information also diverges as you descend the
hierarchy. (A ``region'' and a ``level'' are almost synonymous. We use
the word ``region'' when describing the internal function of a region,
whereas we use the word ``level'' when referring specifically to the
role of the region within the hierarchy.)

\begin{figure}
\resizebox{\textwidth}{!}{\includegraphics{figures/Figure-1_1.pdf}}
\caption{Simplified diagram of four HTM regions arranged in a
  four-level hierarchy, communicating information within levels,
  between levels, and to/from outside the hierarchy.}
\label{figure:region-hierarchy}
\end{figure}

It is possible to combine multiple HTM networks. This kind of
structure makes sense if you have data from more than one source or
sensor. For example, one network might be processing auditory
information and another network might be processing visual
information. There is convergence within each separate network, with
the separate branches converging only towards the top.

\begin{figure}
\resizebox{\textwidth}{!}{\includegraphics{figures/Figure-1_2.pdf}}
\caption{Converging networks from different sensors}
\label{figure:multimodal-convergence}
\end{figure}

The benefit of hierarchical organization is efficiency. It
significantly reduces training time and memory usage because patterns
learned at each level of the hierarchy are reused when combined in
novel ways at higher levels. For an illustration, let's consider
vision. At the lowest level of the hierarchy, your brain stores
information about tiny sections of the visual field such as edges and
corners. An edge is a fundamental component of many objects in the
world. These low-level patterns are recombined at mid-levels into more
complex components such as curves and textures. An arc can be the edge
of an ear, the top of a steering wheel or the rim of a coffee
cup. These mid-level patterns are further combined to represent
high-level object features, such as heads, cars or houses. To learn a
new high level object you don't have to relearn its components.

As another example, consider that when you learn a new word, you don't
need to relearn letters, syllables, or phonemes.

Sharing representations in a hierarchy also leads to generalization of
expected behavior. When you see a new animal, if you see a mouth and
teeth you will predict that the animal eats with his mouth and that it
might bite you. The hierarchy enables a new object in the world to
inherit the known properties of its sub-components.

How much can a single level in an HTM hierarchy learn? Or put another
way, how many levels in the hierarchy are necessary? There is a
tradeoff between how much memory is allocated to each level and how
many levels are needed. Fortunately, HTMs automatically learn the best
possible representations at each level given the statistics of the
input and the amount of resources allocated. If you allocate more
memory to a level, that level will form representations that are
larger and more complex, which in turn means fewer hierarchical levels
may be necessary. If you allocate less memory, a level will form
representations that are smaller and simpler, which in turn means more
hierarchical levels may be needed.

Up to this point we have been describing difficult problems, such as
vision inference (``inference'' is similar to pattern
recognition). But many valuable problems are simpler than vision, and
a single HTM region might prove sufficient. For example, we applied an
HTM to predicting where a person browsing a website is likely to click
next. This problem involved feeding the HTM network streams of web
click data. In this problem there was little or no spatial hierarchy,
the solution mostly required discovering the temporal statistics,
i.e., predicting where the user would click next by recognizing typical
user patterns. The temporal learning algorithms in HTMs are ideal for
such problems.

In summary, hierarchies reduce training time, reduce memory usage, and
introduce a form of generalization. However, many simpler prediction
problems can be solved with a single HTM region.

\subsection*{Regions}

The notion of regions wired in a hierarchy comes from biology. The
neocortex is a large sheet of neural tissue about 2mm
thick. Biologists divide the neocortex into different areas or
``regions'' primarily based on how the regions connect to each
other. Some regions receive input directly from the senses and other
regions receive input only after it has passed through several other
regions. It is the region-to-region connectivity that defines the
hierarchy.

All neocortical regions look similar in their details. They vary in
size and where they are in the hierarchy, but otherwise they are
similar. If you take a slice across the 2mm thickness of a neocortical
region, you will see six layers, five layers of cells and one
non-cellular layer (there are a few exceptions but this is the general
rule). Each layer in a neocortical region has many interconnected
cells arranged in columns. HTM regions also are comprised of a sheet
of highly interconnected cells arranged in columns. ``Layer 3'' in
neocortex is one of the primary feed-forward layers of neurons. The
cells in an HTM region are roughly equivalent to the neurons in Layer
3 in a region of the neocortex.

A fuller description of a HTM region, introduced for the first time in
this revision, includes a model of how Layers 1, 2/3, 4, 5 and 6 combine 
to perform sensorimotor sequence memory, feedback, behavior and hierarchy.
We begin with a detailed description of a single-layer region, and later
extend these concepts to the full model including all the layers.

\begin{figure}
\resizebox{\textwidth}{!}{\includegraphics{figures/Figure-1_3.pdf}}
\caption{A section of an HTM region. HTM regions are comprised of many
  cells. The cells are organized in a two dimensional array of
  columns. This figure shows a small section of a single-layer HTM region with
  four cells per column. Each column connects to a subset of the input
  and each cell connects to other cells in the region (connections not
  shown). Note that this HTM region, including its columnar structure,
  is equivalent to one layer of neurons in a neocortical region.}
\label{figure:one-region}
\end{figure}

Although an HTM region is equivalent to only a portion of a neocortical region, 
it can do inference and prediction on complex data streams and therefore can be useful in many problems.

\subsection*{Sparse Distributed Representations}

Although neurons in the neocortex are highly interconnected,
inhibitory neurons guarantee that only a small percentage of the
neurons are active at one time. Thus, information in the brain is
always represented by a small percentage of active neurons within a
large population of neurons. This kind of encoding is called a
``sparse distributed representation.'' ``Sparse'' means that only a
small percentage of neurons are active at one time. ``Distributed''
means that the activations of many neurons are required in order to
represent something. A single active neuron conveys some meaning but
it must be interpreted within the context of a population of neurons
to convey the full meaning.

HTM regions also use sparse distributed representations. In fact, the
memory mechanisms within an HTM region are dependent on using sparse
distributed representations, and wouldn't work otherwise. The input to
an HTM region is always a distributed representation, but it may not
be sparse, so the first thing an HTM region does is to convert its
input into a sparse distributed representation.

For example, a region might receive 20,000 input bits. The percentage
of input bits that are ``1'' and ``0'' might vary significantly over
time. One time there might be 5,000 ``1'' bits and another time there
might be 9,000 ``1'' bits. The HTM region could convert this input
into an internal representation of 10,000 bits of which 2\%, or 200,
are active at once, regardless of how many of the input bits are
``1.'' As the input to the HTM region varies over time, the internal
representation also will change, but there always will be about 200
bits out of 10,000 active.

It may seem that this process generates a large loss of information as
the number of possible input patterns is much greater than the number
of possible representations in the region. However, both numbers are
incredibly big. The actual inputs seen by a region will be a miniscule
fraction of all possible inputs. Later we will describe how a region
creates a sparse representation from its input. The theoretical loss
of information will not have a practical effect.

\begin{figure}
\resizebox{\textwidth}{!}{\includegraphics{figures/Figure-1_4.pdf}}
\caption{An HTM region showing sparse distributed cell activation}
\label{figure:sparse-activation}
\end{figure}

Sparse distributed representations have several desirable properties
and are integral to the operation of HTMs. They will be touched on
again later.

\subsection*{The role of time}

Time plays a crucial role in learning, inference, and prediction.

Let's start with inference. Without using time, we can infer almost
nothing from our tactile and auditory senses. For example if you are
blindfolded and someone places an apple in your hand, you can identify
what it is after manipulating it for just a second or so. As you move
your fingers over the apple, although the tactile information is
constantly changing, the object itself --- the apple, as well as your
high-level percept for ``apple'' --- stays constant. However, if an
apple was placed on your outstretched palm, and you weren't allowed to
move your hand or fingers, you would have great difficulty identifying
it as an apple rather than a lemon.

The same is true for hearing. A static sound conveys little meaning. A
word like ``apple,'' or the crunching sounds of someone biting into an
apple, can only be recognized from the dozens or hundreds of rapid,
sequential changes over time of the sound spectrum.

Vision, in contrast, is a mixed case. Unlike with touch and hearing,
humans are able to recognize images when they are flashed in front of
them too fast to give the eyes a chance to move. Thus, visual
inference does not always require time-changing inputs. However,
during normal vision we constantly move our eyes, heads and bodies,
and objects in the world move around us too. Our ability to infer
based on quick visual exposure is a special case made possible by the
statistical properties of vision and years of training. The general
case for vision, hearing, and touch is that inference requires
time-changing inputs.

Having covered the general case of inference, and the special case of
vision inference of static images, let's look at learning. In order to
learn, all HTM systems must be exposed to time-changing inputs during
training. Even in vision, where static inference is sometimes
possible, we must see changing images of objects to learn what an
object looks like. For example, imagine a dog is running toward
you. At each instance in time the dog causes a pattern of activity on
the retina in your eye. You perceive these patterns as different views
of the same dog, but mathematically the patterns are entirely
dissimilar. The brain learns that these different patterns mean the
same thing by observing them in sequence. Time is the ``supervisor,''
teaching you which spatial patterns go together.

Note that it isn't sufficient for sensory input merely to change. A
succession of unrelated sensory patterns would only lead to
confusion. The time-changing inputs must come from a common source in
the world. Note also that although we use human senses as examples,
the general case applies to non-human senses as well. If we want to
train an HTM to recognize patterns from a power plant's temperature,
vibration and noise sensors, the HTM will need to be trained on data
from those sensors changing through time.

Typically, an HTM network needs to be trained with lots of data. You
learned to identify dogs by seeing many instances of many breeds of
dogs, not just one single view of one single dog. The job of the HTM
algorithms is to learn the temporal sequences from a stream of input
data, i.e., to build a model of which patterns follow which other
patterns. This job is difficult because it may not know when sequences
start and end, there may be overlapping sequences occurring at the
same time, learning has to occur continuously, and learning has to
occur in the presence of noise.

Learning and recognizing sequences is the basis of forming
predictions. Once an HTM learns what patterns are likely to follow
other patterns, it can predict the likely next pattern(s) given the
current input and immediately past inputs. Prediction is covered in
more detail later.

We now will turn to the four basic functions of HTM: learning,
inference, prediction, and behavior. Every HTM region performs the
first three functions: learning, inference, and prediction. Behavior,
however, is different. We know from biology that most neocortical
regions have a role in creating behavior but we do not believe it is
essential for many interesting applications. Therefore we have not
included behavior in our current implementation of HTM. 

A substantially more complex theory involving the interaction of behavior
and sensation, formation of stable sequences, the roles of all layers in
a region, feedback and hierarchy has been recently developed and is presented
here for the first time.



\subsection*{Learning}

An HTM region learns about its world by finding patterns and then
sequences of patterns in sensory data. The region does not ``know''
what its inputs represent; it works in a purely statistical realm. It
looks for combinations of input bits that occur together often, which
we call spatial patterns. It then looks for how these spatial patterns
appear in sequence over time, which we call temporal patterns or
sequences.

If the input to the region represents environmental sensors on a
building, the region might discover that certain combinations of
temperature and humidity on the north side of the building occur often
and that different combinations occur on the south side of the
building. Then it might learn that sequences of these combinations
occur as each day passes.

If the input to a region represented information related to purchases
within a store, the HTM region might discover that certain types of
articles are purchased on weekends, or that when the weather is cold
certain price ranges are favored in the evening. Then it might learn
that different individuals follow similar sequential patterns in their
purchases.

A single HTM region has limited learning capability. A region
automatically adjusts what it learns based on how much memory it has
and the complexity of the input it receives. The spatial patterns
learned by a region will necessarily become simpler if the memory
allocated to a region is reduced. Or the spatial patterns learned can
become more complex if the allocated memory is increased. If the
learned spatial patterns in a region are simple, then a hierarchy of
regions may be needed to understand complex images. We see this
characteristic in the human vision system where the neocortical region
receiving input from the retina learns spatial patterns for small
parts of the visual space. Only after several levels of hierarchy do
spatial patterns combine and represent most or all of the visual
space.

Like a biological system, the learning algorithms in an HTM region are
capable of ``on-line learning,'' i.e., they continually learn from each
new input. There isn't a need for a learning phase separate from an
inference phase, though inference improves after additional
learning. As the patterns in the input change, the HTM region will
gradually change, too.

After initial training, an HTM can continue to learn or,
alternatively, learning can be disabled after the training
phase. Another option is to turn off learning only at the lowest
levels of the hierarchy but continue to learn at the higher
levels. Once an HTM has learned the basic statistical structure of its
world, most new learning occurs in the upper levels of the
hierarchy. If an HTM is exposed to new patterns that have previously
unseen low-level structure, it will take longer for the HTM to learn
these new patterns. We see this trait in humans. Learning new words in
a language you already know is relatively easy. However, if you try to
learn new words from a foreign language with unfamiliar sounds, you'll
find it much harder because you don't already know the low level
sounds.

Simply discovering patterns is a potentially valuable
capability. Understanding the high-level patterns in market
fluctuations, disease, weather, manufacturing yield, or failures of
complex systems, such as power grids, is valuable in itself. Even so,
learning spatial and temporal patterns is mostly a precursor to
inference and prediction.

\subsection*{Inference}

After an HTM has learned the patterns in its world, it can perform
inference on novel inputs. When an HTM receives input, it will match
it to previously learned spatial and temporal patterns. Successfully
matching new inputs to previously stored sequences is the essence of
inference and pattern matching.

Think about how you recognize a melody. Hearing the first note in a
melody tells you little. The second note narrows down the
possibilities significantly but it may still not be enough. Usually it
takes three, four, or more notes before you recognize the
melody. Inference in an HTM region is similar. It is constantly
looking at a stream of inputs and matching them to previously learned
sequences. An HTM region can find matches from the beginning of
sequences but usually it is more fluid, analogous to how you can
recognize a melody starting from anywhere. Because HTM regions use
distributed representations, the region's use of sequence memory and
inference are more complicated than the melody example implies, but
the example gives a flavor for how it works.

It may not be immediately obvious, but every sensory experience you
have ever had has been novel, yet you easily find familiar patterns in
this novel input. For example, you can understand the word
``breakfast'' spoken by almost anyone, no matter whether they are old
or young, male or female, are speaking quickly or slowly, or have a
strong accent. Even if you had the same person say the same word
``breakfast'' a hundred times, the sound would never stimulate your
cochleae (auditory receptors) in exactly the same way twice.

An HTM region faces the same problem your brain does: inputs may never
repeat exactly. Consequently, just like your brain, an HTM region must
handle novel input during inference and training. One way an HTM
region copes with novel input is through the use of sparse distributed
representations. A key property of sparse distributed representations
is that you only need to match a portion of the pattern to be
confident that the match is significant.

\subsection*{Prediction}

Every region of an HTM stores sequences of patterns. By matching
stored sequences with current input, a region forms a prediction about
what inputs will likely arrive next. HTM regions actually store
transitions between sparse distributed representations. In some
instances the transitions can look like a linear sequence, such as the
notes in a melody, but in the general case many possible future inputs
may be predicted at the same time. An HTM region will make different
predictions based on context that might stretch back far in time. The
majority of memory in an HTM is dedicated to sequence memory, or
storing transitions between spatial patterns.

Following are some key properties of HTM prediction.

\begin{enumerate}
\item {\bf Prediction is continuous.}

Without being conscious of it, you are constantly predicting. HTMs do
the same. When listening to a song, you are predicting the next
note. When walking down the stairs, you are predicting when your foot
will touch the next step. When watching a baseball pitcher throw, you
are predicting that the ball will come near the batter. In an HTM
region, prediction and inference are almost the same thing. Prediction
is not a separate step but integral to the way an HTM region works.

\item {\bf Prediction occurs in every region at every level of the hierarchy.}

If you have a hierarchy of HTM regions, prediction will occur at each
level. Regions will make predictions about the patterns they have
learned. In a language example, lower level regions might predict
possible next phonemes, and higher level regions might predict words
or phrases.

\item {\bf Predictions are context sensitive.}

Predictions are based on what has occurred in the past, as well as
what is occurring now. Thus an input will produce different
predictions based on previous context. An HTM region learns to use as
much prior context as needed, and can keep the context over both short
and long stretches of time. This ability is known as ``variable
order'' memory. For example, think about a memorized speech such as
the Gettysburg Address. To predict the next word, knowing just the
current word is rarely sufficient; the word ``and'' is followed by
``seven'' and later by ``dedicated'' just in the first
sentence. Sometimes, just a little bit of context will help
prediction; knowing ``four score and'' would help predict ``seven.''
Other times, there are repetitive phrases, and one would need to use
the context of a far longer timeframe to know where you are in the
speech, and therefore what comes next.

\item {\bf A prediction tells us if a new input is expected or unexpected.}

Each HTM region is a novelty detector. Because each region predicts
what will occur next, it ``knows'' when something unexpected
happens. HTMs can predict many possible next inputs simultaneously,
not just one. So it may not be able to predict exactly what will
happen next, but if the next input doesn't match any of the
predictions the HTM region will know that an anomaly has occurred.

In a sensorimotor HTM, a region also uses information about motor commands
to assist with prediction. As long as new inputs are correctly predicted, the 
region will be able to form a stable representation of the input. If prediction 
fails, the region will signal this change in the input.

\item {\bf Prediction helps make the system more robust to noise.}

When an HTM predicts what is likely to happen next, the prediction can
bias the system toward inferring what it predicted. For example, if an
HTM were processing spoken language, it would predict what sounds,
words, and ideas are likely to be uttered next. This prediction helps
the system fill in missing data. If an ambiguous sound arrives, the
HTM will interpret the sound based on what it is expecting, thus
helping inference even in the presence of noise.

\item {\bf Prediction leads to stability.}

One of the properties of HTMs is that the outputs of regions become more stable --- that is
slower changing, longer-lasting --- the higher they are in the
hierarchy. This property results from how a region predicts.
In the new sensorimotor theory, higher-level regions communicate a SDR
of a slowly-changing sequence to lower layers, helping them predict the inputs
they will receive in this context. As stated previously, each region also uses 
motor information to help predict transitions in the patterns recognised by the region.
This characteristic mirrors our experience of the
real world, where high level concepts --- such as the name of a song
--- change more slowly than low level concepts --- the actual notes of
the song.

In an HTM region, sequence memory, inference, and prediction are
intimately integrated. They are the core functions of a region.

\end{enumerate}

\subsection*{Behavior}

Our behavior influences what we perceive. As we move our eyes, our
retina receives changing sensory input. Moving our limbs and fingers
causes varying touch sensation to reach the brain. Almost all our
actions change what we sense, and a very large proportion of what we sense 
is due to our own behavior. Sensory input and motor behavior are
intimately entwined.

For decades the prevailing view was that a single region in the
neocortex, the primary motor region, was where motor commands
originated in the neocortex. Over time it was discovered that most or
all regions in the neocortex have a motor output, even low level
sensory regions. It appears that all cortical regions integrate
sensory and motor functions.

We expect that a motor output could be added to each HTM region within
the currently existing framework since generating motor commands is
similar to making predictions. However, all the implementations of
HTMs to date have been purely sensory, without a motor component.

In this revision, we introduce a new framework which explains how 
behavior and sensory/feedforward inputs are integrated and produced in a
fuller HTM model.

\subsection*{Progress toward the implementation of HTM}

We have made substantial progress turning the HTM theoretical
framework into a practical technology. We have implemented and tested
several versions of the HTM cortical learning algorithms and have
found the basic architecture to be sound. As we test the algorithms on
new data sets, we will refine the algorithms and add missing
pieces. We will update this document as we do. The next three chapters
describe the current state of the algorithms.

There are many components of the theory that are not yet implemented,
including attention, feedback between regions, specific timing, and
behavior/sensory-motor integration. These missing components should
fit into the framework already created.

\chapter{HTM Cortical Learning Algorithms}
\label{chapter:learning}

This chapter describes the learning algorithms at work inside an HTM
region. Chapters~\ref{chapter:pattern-memory} and
\ref{chapter:transition-memory} describe the implementation of the
learning algorithms using pseudocode, whereas this chapter is more
conceptual.

\section*{Terminology}

Before we get started, a note about terminology might be helpful. We
use the language of neuroscience in describing the HTM learning
algorithms. Terms such as cells, synapses, potential synapses,
dendrite segments, and columns are used throughout. This terminology
is logical since the learning algorithms were largely derived by
matching neuroscience details with theoretical needs. However, in the
process of implementing the algorithms we were confronted with
performance issues and therefore once we felt we understood how
something worked we would look for ways to speed processing. This
often involved deviating from a strict adherence to biological details
as long as we could get the same results. If you are new to
neuroscience this won't be a problem. However, if you are familiar
with neuroscience terms, you might find yourself confused as our use
of terms varies from your expectation. The appendixes on biology
discuss the differences and similarities between the HTM learning
algorithms and their neurobiological equivalents in detail. Here we
will mention a few of the deviations that are likely to cause the most
confusion.

\subsection*{Region}

For the moment, we will model each region as a single layer, which receives 
a stream of feedforward inputs, recognises spatial patterns, learns sequences
of these patterns, and makes predictions. In the neocortex, this models the key functions performed 
in Layer 3. We call this initial model the ``Sensory CLA.''

In a later chapter, we will use this basic understanding of recognition, sequence learning, and prediction
to build a more complete sensorimotor temporal memory system. This new theory is called the ``Sensorimotor
CLA.''

\subsection*{Cell states}

HTM cells have two output states, active (firing) and inactive. We have not
found a need for modeling individual action potentials or even scalar
rates of activity beyond the two active states. The use of distributed
representations seems to overcome the need to model scalar activity
rates in cells.

Apart from overt activity, each cell possesses a level of ``potential 
activation'' which corresponds to the amount of depolarisation taking place
during a timestep. This depolarisation arises from feed-forward input,
as well as from lateral input (which represents a prediction), and
these are considered separately. The former is primarily used to perform
feedforward recognition (and thus the region's SDR) while the latter is used
to identify cells as ``predictive''.

\subsection*{Dendrite segments}

HTM cells have a relatively realistic (and therefore complex) dendrite
model. In theory each HTM cell has one proximal dendrite segment and a
dozen or two distal dendrite segments. The proximal dendrite segment
receives feed-forward input and the distal dendrite segments receive
lateral input from nearby cells. A class of inhibitory cells forces
all the cells in a column to respond to similar feed-forward input. 

To simplify, we removed the proximal dendrite segment from each cell and
replaced it with a single shared dendrite segment per column of
cells. The recognition function (described below) operates on the
shared dendrite segment, at the level of columns. The prediction
function operates on distal dendrite segments, at the level of
individual cells within columns. This simplification achieves the same
functionality, though in biology there is no equivalent to a dendrite
segment attached to a column.

Some implementations of HTM have re-introduced the per-cell proximal 
dendrites. The new Sensorimotor CLA theory uses both the per-cell and per-column
models of proximal dendrites (the latter being due to fast inhibitory cells 
surrounding the column and responding to similar inputs).

\subsection*{Synapses}

HTM synapses have binary weights. Biological synapses have varying
weights but they are also partially stochastic, suggesting a
biological neuron cannot rely on precise synaptic weights. The use of
distributed representations in HTMs plus our model of dendrite
operation allows us to assign binary weights to HTM synapses with no
ill effect. To model the forming and un-forming of synapses we use two
additional concepts from neuroscience that you may not be familiar
with. One is the concept of ``potential synapses.'' This represents
all the axons that pass close enough to a dendrite segment that they
could potentially form a synapse. The second is called ``permanence.''
This is a scalar value assigned to each potential synapse. The
permanence of a synapse represents a range of connectedness between an
axon and a dendrite. Biologically, this corresponds to the extent of growth
of dendritic spines, the size and contents of the axonal terminal, and 
the number and characteristics of receptor gates in each synapse.
We simplify this drastically by modelling the range from completely
unconnected, to starting to form a synapse but not connected yet, to a
minimally connected synapse, to a large fully connected synapse. The
permanence of a synapse is a scalar value ranging from $0.0$ to
$1.0$. Learning involves incrementing and decrementing a synapse's
permanence. When a synapse's permanence is above a threshold, it is
connected with a weight of ``1''. When it is below the threshold, it
is unconnected with a weight of ``0.''

\section*{Overview}

Imagine that you are a region of an HTM. Your input consists of
thousands or tens of thousands of bits. These input bits may represent
sensory data or they may come from another region lower in the
hierarchy. They are turning on and off in complex ways. What are you
supposed to do with this input?

We already have discussed the answer in its simplest form. Each HTM
region looks for common patterns in its input and then learns
sequences of those patterns. From its memory of sequences, each region
makes predictions. That high level description makes it sound easy,
but in reality there is a lot going on. Let's break it down a little
further into the following three steps:

\begin{enumerate}
\item Form a sparse distributed representation of the input
\item Form a representation of the input in the context of previous
  inputs
\item Form a prediction based on the current input in the context of
  previous inputs
\end{enumerate}

We will discuss each of these steps in more detail.

\begin{enumerate}
\item {\bf Form a sparse distributed representation of the input}

When you imagine an input to a region, think of it as a large number
of bits. In a brain these would be axons from neurons. At any point in
time some of these input bits will be active (value 1) and others will
be inactive (value 0). The percentage of input bits that are active
vary, say from 0\% to 60\%. The first thing an HTM region does is to
convert this input into a new representation that is sparse. For
example, the input might have 40\% of its bits ``on'' but the new
representation has just 2\% of its bits ``on.''

An HTM region is logically comprised of a set of columns. Each column
is comprised of one or more cells. Columns may be logically arranged
in a 2D array but this is not a requirement. Each column in a region
is connected to a unique subset of the input bits (usually overlapping
with other columns but never exactly the same subset of input
bits). As a result, different input patterns result in different
levels of activation potential in the columns. The columns with the strongest
activation inhibit, or deactivate, the columns with weaker
activation. (The inhibition occurs within a radius that can span from
very local to the entire region.) The sparse representation of the
input is encoded by which columns are active and which are inactive
after inhibition. The inhibition function is defined to achieve a
relatively constant percentage of columns to be active, even when the
number of input bits that are active varies significantly.

\begin{figure}
\resizebox{\textwidth}{!}{\includegraphics{figures/Figure-2_1.pdf}}
\caption{An HTM region consists of columns of cells. Only a small
  portion of a region is shown. Each column of cells receives
  activation from a unique subset of the input. Columns with the
  strongest activation inhibit columns with weaker activation. The
  result is a sparse distributed representation of the input. The
  figure shows active columns in light grey. (When there is no prior
  state, every cell in the active columns will be active, as shown.)}
\label{figure:region-columns}
\end{figure}

Imagine now that the input pattern changes. If only a few input bits
change, some columns will receive a few more or a few less inputs in
the ``on'' state, but the set of active columns will not likely change
much. Thus similar input patterns (ones that have a significant number
of active bits in common) will map to a relatively stable set of
active columns. How stable the encoding is depends greatly on what
inputs each column is connected to. These connections are learned via
a method described later.

All these steps (learning the connections to each column from a subset
of the inputs, determining the level of input to each column, and
using inhibition to select a sparse set of active columns) is referred
to as ``Pattern Memory.''

Another term we've used is ``Spatial Pooling'', because patterns that are
``spatially'' similar (meaning they share a large number of active
bits) are ``pooled'' (meaning they are grouped together in a common
representation).

\item {\bf Form a representation of the input in the context of
  previous inputs}

The next function performed by a region is to convert the columnar
representation of the input into a new representation that includes
state, or context, from the past. The new representation is formed by
activating a subset of the cells within each column, typically only
one cell per column (Figure~\ref{figure:column-partial-activation}).

Consider hearing two spoken sentences, ``I ate a pear'' and ``I have
eight pears.'' The words ``ate'' and ``eight'' are homonyms; they
sound identical. We can be certain that at some point in the brain
there are neurons that respond identically to the spoken words ``ate''
and ``eight.'' After all, identical sounds are entering the
ear. However, we also can be certain that at another point in the
brain the neurons that respond to this input are different, in
different contexts. The representations for the sound ``ate'' will be
different when you hear ``I ate'' vs.\ ``I have eight.'' Imagine that
you have memorized the two sentences ``I ate a pear'' and ``I have
eight pears.'' Hearing ``I ate\dots'' leads to a different prediction
than ``I have eight\dots.'' There must be different internal
representations upon hearing ``I ate'' and ``I have eight.'' This
principle of encoding an input differently in different contexts is a
universal feature of perception and action and is one of the most
important functions of an HTM region. It is hard to overemphasize the
importance of this capability.

Each column in an HTM region consists of multiple cells. All cells in
a column get the same feed-forward input. Each cell in a column can be
active or not active. By selecting different active cells in each
active column, we can represent the exact same input differently in
different contexts. A specific example might help. Say every column
has 10 cells and the representation of every input consists of 100
active columns. If only one cell per column is active at a time, we
have $10^{100}$ ways of representing the exact same input. The same
input will always result in the same 100 columns being active, but in
different contexts different cells in those columns will be
active. Now we can represent the same input in a very large number of
contexts, but how unique will those different representations be?
Nearly all randomly chosen pairs of the $10^{100}$ possible patterns
will overlap by about 10 cells. Thus two representations of a
particular input in different contexts will have about 10 cells in
common and 90 cells that are different, making them easily
distinguishable.

The general rule used by an HTM region is the following. When a column
becomes active, it looks at all the cells in the column. If one or
more cells in the column are already in the predictive state, only
those cells become active. If no cells in the column are in the
predictive state, then all the cells become active. You can think of
it this way, if an input pattern is expected then the system confirms
that expectation by activating only the cells in the predictive
state. If the input pattern is unexpected then the system activates
all cells in the column as if to say ``the input occurred unexpectedly
so all possible interpretations are valid.''

If there is no prior state, and therefore no context and prediction,
all the cells in a column will become active when the column becomes
active. This scenario is similar to hearing the first note in a
song. Without context you usually can't predict what will happen next;
all options are available. If there is prior state but the input does
not match what is expected, all the cells in the active column will
become active. This determination is done on a column by column basis
so a predictive match or mismatch is never an ``all-or-nothing''
event.

\begin{figure}
\resizebox{\textwidth}{!}{\includegraphics{figures/Figure-2_2.pdf}}
\caption{By activating a subset of cells in each column, an HTM region
  can represent the same input in many different contexts. Columns
  only activate predicted cells. Columns with no predicted cells
  activate all the cells in the column. The figure shows some columns
  with one cell active and some columns with all cells active.}
\label{figure:column-partial-activation}
\end{figure}

As mentioned in the terminology section above, HTM cells can be in one
of three states. If a cell is active due to feed-forward input we just
use the term ``active.'' If the cell has high potential due to lateral
connections to other nearby cells we say it is in the ``predictive
state'' (Figure~\ref{figure:activity-types}).

\item {\bf Form a prediction based on the input in the context of
  previous inputs}

The final step for our region is to make a prediction of what is
likely to happen next. The prediction is based on the representation
formed in step 2), which includes context from all previous inputs.

When a region makes a prediction it depolarises (into the predictive
state) all the cells that will likely become active due to future
feed-forward input. Because representations in a region are sparse,
multiple predictions can be made at the same time. For example if 2\%
of the columns are active due to an input, you could expect that ten
different predictions could be made resulting in 20\% of the columns
having a predicted cell. Or, twenty different predictions could be
made resulting in 40\% of the columns having a predicted cell. If each
column had four cells, with one active at a time, then 10\% of the
cells would be in the predictive state.

A future chapter on sparse distributed representations will show that
even though different predictions are merged together, a region can
know with high certainty whether a particular input was predicted or
not.

How does a region make a prediction? When input patterns change over
time, different sets of columns and cells become active in
sequence. When a cell becomes active, it forms connections to a subset
of the cells nearby that were active immediately prior. These
connections can be formed quickly or slowly depending on the learning
rate required by the application. Later, all a cell needs to do is to
look at these connections for coincident activity. If the connections
become active, the cell can expect that it might become active shortly
and enters a predictive state. Thus the feed-forward activation of a
set of cells will lead to the predictive depolarisation of other sets of
cells that typically follow. Think of this as the moment when you
recognize a song and start predicting the next notes.

\begin{figure}
\resizebox{\textwidth}{!}{\includegraphics{figures/Figure-2_3.pdf}}
\caption{At any point in time, some cells in an HTM region will be
  active due to feed-forward input (shown in light gray). Other cells
  that receive lateral input from active cells will be in a predictive
  state (shown in dark gray).}
\label{figure:activity-types}
\end{figure}

In summary, when a new input arrives, it leads to a sparse set of
active columns. One or more of the cells in each column become active,
these in turn cause other cells to enter a predictive state through
learned connections between cells in the region. The cells depolarised
by connections within the region constitute a prediction of what is
likely to happen next. When the next feed-forward input arrives, it
selects another sparse set of active columns. If a newly active column
is unexpected, meaning it was not predicted by any cells, it will
activate all the cells in the columns. If a newly active column has
one or more predicted cells, only those cells will become active. The
output of a region is the pattern of activity of all cells in the region,
including the cells active because of feed-forward input, and in particular those cells
which were predictive of the current input in context.

%As mentioned earlier, predictions are not just for the next time
%step. Predictions in an HTM region can be for several time steps into
% the future. Using melodies as example, an HTM region would not just
% predict the next note in a melody, but might predict the next four
% notes. This leads to a desirable property. The output of a region (the
% union of all the active and predicted cells in a region) changes more
% slowly than the input. Imagine the region is predicting the next four
% notes in a melody. We will represent the melody by the letter sequence
% A,B,C,D,E,F,G. After hearing the first two notes, the region
% recognizes the sequence and starts predicting. It predicts
% C,D,E,F. The ``B'' cells are already active so cells for B,C,D,E,F are
% all in one of the two active states. Now the region hears the next
% note ``C.'' The set of active and predictive cells now represents
% ``C,D,E,F,G.'' Note that the input pattern changed completely going
% from ``B'' to ``C,'' but only 20\% of the cells changed.



\end{enumerate}

% Because the output of an HTM region is a vector representing the
% activity of all the region's cells, the output in this example is five
% times more stable than the input. In a hierarchical arrangement of
% regions, we will see an increase in temporal stability as you ascend
% the hierarchy.

We use the term ``Transition Memory'' to describe the two steps of
adding context to the representation and making cells predictive. 
% By creating
% slowly changing outputs for sequences of patterns, we are in essence
% ``pooling'' together different patterns that follow each other in
% time.

Now we will go into another level of detail. We start with concepts
that are shared by both Pattern and Transition Memory. Then we
discuss concepts and details unique to Pattern Memory followed by
those unique to Transition Memory.

\section*{Shared concepts}

Learning in the Pattern and Transition Memory is
similar. Learning in both cases involves establishing connections, or
synapses, between cells. The Transition Memory learns connections
between cells in the same region. The Pattern Memory learns
feed-forward connections between input bits and columns.

\subsection*{Binary weights}
HTM synapses have only a 0 or 1 effect; their ``weight'' is binary, a
property unlike many neural network models which use scalar variable
values in the range of 0 to 1.

\subsection*{Permanence}
Synapses are forming and unforming constantly during learning. As
mentioned before, we assign a scalar value to each synapse ($0.0$ to
$1.0$) to indicate how permanent the connection is. When a connection is
reinforced, its permanence is increased. Under other conditions, the
permanence is decreased. When the permanence is above a threshold
(e.g., $0.2$), the synapse is considered to be established. If the
permanence is below the threshold, the synapse will have no effect.

\subsection*{Dendrite segments}
Synapses connect to dendrite segments. There are two types of dendrite
segments, proximal and distal.
\begin{itemize}
\item A proximal dendrite segment forms synapses with feed-forward
  inputs. The active synapses on this type of segment are linearly
  summed to determine the feed-forward activation potential of a column.

\item A distal dendrite segment forms synapses with cells within the
  region. Every cell has several distal dendrite segments. If the sum
  of the active synapses on a distal segment exceeds a threshold, then
  the associated cell becomes active in a predicted state. Since there
  are multiple distal dendrite segments per cell, a cell's predictive
  state is the logical OR operation of several constituent threshold
  detectors.
\end{itemize}

\subsection*{Potential Synapses}
As mentioned earlier, each dendrite segment has a list of potential synapses. 
All the potential synapses are given a permanence value and may become functional synapses if their permanence values exceed a threshold.

\subsection*{Learning}
Learning involves incrementing or decrementing the permanence values
of potential synapses on a dendrite segment. The rules used for making
synapses more or less permanent are similar to ``Hebbian'' learning
rules. For example, if a post-synaptic cell is active due to a
dendrite segment receiving input above its threshold, then the
permanence values of the synapses on that segment are
modified. Synapses that are active, and therefore contributed to the
cell being active, have their permanence increased. Synapses that are
inactive, and therefore did not contribute, have their permanence
decreased. The exact conditions under which synapse permanence values
are updated differ in the Pattern and Transition Memories. The details are
described below.

Now we will discuss concepts specific to the Pattern and Transition Memory functions.

\section*{Pattern Memory Concepts}

The most fundamental function of the Pattern Memory is to convert a
region's input into a sparse pattern. This function is important
because the mechanism used to learn sequences and make predictions
requires starting with sparse distributed patterns.

There are several overlapping goals for the Pattern Memory, which
determine how it operates and learns.

\begin{enumerate}
\item {\bf Use all columns - Boosting}

An HTM region has a fixed number of columns that learn to represent
common patterns in the input. One objective is to make sure all the
columns learn to represent something useful regardless of how many
columns you have. We don't want columns that are never active. To
prevent this from happening, we keep track of how often a column is
active relative to its neighbors. If the relative activity of a column
is too low, it boosts its input activity level until it starts to be
part of the winning set of columns. In essence, all columns are
competing with their neighbors to be a participant in representing
input patterns. If a column is not very active, it will become more
aggressive. When it does, other columns will be forced to modify their
input and start representing slightly different input patterns.

\item {\bf Maintain desired density - Inhibition}

A region needs to form a sparse representation of its inputs. Columns
with the most input inhibit their neighbors. There is a radius of
inhibition which is proportional to the size of the receptive fields
of the columns (and therefore can range from small to the size of the
entire region). Within the radius of inhibition, we allow only a
percentage of the columns with the most active input to be
``winners.'' The remainders of the columns are disabled. (A ``radius''
of inhibition implies a 2D arrangement of columns, but the concept can
be adapted to other topologies.)

\item {\bf Avoid trivial patterns}

We want all our columns to represent non-trivial patterns in the
input. This goal can be achieved by setting a minimum threshold of
input for the column to be active. For example, if we set the
threshold to 50, it means that a column must have a least 50 active
synapses on its dendrite segment to be active, guaranteeing a certain
level of complexity to the pattern it represents.

\item {\bf Avoid extra connections}

If we aren't careful, a column could form a large number of valid
synapses. It would then respond strongly to many different unrelated
input patterns. Different subsets of the synapses would respond to
different patterns. To avoid this problem, we decrement the permanence
value of any synapse that isn't currently contributing to a winning
column. By making sure non-contributing synapses are sufficiently 
penalized, we guarantee a column represents a limited number input
patterns, sometimes only one.

\item {\bf Self adjusting receptive fields}

Real brains are highly ``plastic''; regions of the neocortex can learn
to represent entirely different things in reaction to various
changes. If part of the neocortex is damaged, other parts will adjust
to represent what the damaged part used to represent. If a sensory
organ is damaged or changed, the associated part of the neocortex will
adjust to represent something else. The system is self-adjusting.

We want our HTM regions to exhibit the same flexibility. If we
allocate 10,000 columns to a region, it should learn how to best
represent the input with 10,000 columns. If we allocate 20,000
columns, it should learn how best to use that number. If the input
statistics change, the columns should change to best represent the new
reality. In short, the designer of an HTM should be able to allocate
any resources to a region and the region will do the best job it can
of representing the input based on the available columns and input
statistics. The general rule is that with more columns in a region,
each column will represent larger and more detailed patterns in the
input. Typically the columns also will be active less often, yet we
will maintain a relative constant sparsity level.

No new learning rules are required to achieve this highly desirable
goal. By boosting inactive columns, inhibiting neighboring columns to
maintain constant sparsity, establishing minimal thresholds for input,
maintaining a large pool of potential synapses, and adding and
forgetting synapses based on their contribution, the ensemble of
columns will dynamically configure to achieve the desired effect.
\end{enumerate}

\section*{Pattern Memory details}

We can now go through everything the Pattern Memory function does.

\begin{enumerate}
\item Start with an input consisting of a fixed number of bits. These
  bits might represent sensory data or they might come from another
  region lower in the hierarchy.

\item Assign a fixed number of columns to the region receiving this
  input. Each column has an associated dendrite segment. Each dendrite
  segment has a set of potential synapses representing a subset of the
  input bits. Each potential synapse has a permanence value. Based on
  their permanence values, some of the potential synapses will be
  valid.

\item For any given input, determine how many valid synapses on each
  column are connected to active input bits.

\item The number of active synapses is multiplied by a ``boosting''
  factor which is dynamically determined by how often a column is
  active relative to its neighbors.

\item The columns with the highest activations after boosting disable
  all but a fixed percentage of the columns within an inhibition
  radius. The inhibition radius is itself dynamically determined by
  the spread (or ``fan-out'') of input bits. There is now a sparse set
  of active columns.

\item For each of the active columns, we adjust the permanence values
  of all the potential synapses. The permanence values of synapses
  aligned with active input bits are increased. The permanence values
  of synapses aligned with inactive input bits are decreased. The
  changes made to permanence values may change some synapses from
  being valid to not valid, and vice-versa.
\end{enumerate}

\section*{Transition Memory Concepts}

Recall that the Transition Memory learns sequences and makes
predictions. The basic method is that when a cell becomes active, it
forms connections to other cells that were active just prior. Cells
can then predict when they will become active by looking at their
connections. If all the cells do this, collectively they can store and
recall sequences, and they can predict what is likely to happen
next. There is no central storage for a sequence of patterns; instead,
memory is distributed among the individual cells. Because the memory
is distributed, the system is robust to noise and error. Individual
cells can fail, usually with little or no discernible effect.

It is worth noting a few important properties of sparse distributed
representations that the Transition Memory exploits.

Assume we have a hypothetical region that always forms representations
by using 200 active cells out of a total of 10,000 cells (2\% of the
cells are active at any time). How can we remember and recognize a
particular pattern of 200 active cells? A simple way to do this is to
make a list of the 200 active cells we care about. If we see the same
200 cells active again we recognize the pattern. However, what if we
made a list of only 20 of the 200 active cells and ignored the other
180? What would happen? You might think that remembering only 20 cells
would cause lots of errors, that those 20 cells would be active in
many different patterns of 200. But this isn't the case. Because the
patterns are large and sparse (in this example 200 active cells out of
10,000), remembering 20 active cells is almost as good as remembering
all 200. The chance for error in a practical system is exceedingly
small and we have reduced our memory needs considerably.

The cells in an HTM region take advantage of this property. Each of a
cell's dendrite segments has a set of connections to other cells in
the region. A dendrite segment forms these connections as a means of
recognizing the state of the network at some point in time. There may
be hundreds or thousands of active cells nearby but the dendrite
segment only has to connect to 15 or 20 of them. When the dendrite
segment sees 15 of those active cells, it can be fairly certain the
larger pattern is occurring. This technique is called ``sub-sampling''
and is used throughout the HTM algorithms.

Every cell participates in many different distributed patterns and in
many different sequences. A particular cell might be part of dozens or
hundreds of temporal transitions. Therefore every cell has several
dendrite segments, not just one. Ideally a cell would have one
dendrite segment for each pattern of activity it wants to
recognize. Practically though, a dendrite segment can learn
connections for several completely different patterns and still work
well. For example, one segment might learn 20 connections for each of
4 different patterns, for a total of 80 connections. We then set a
threshold so the dendrite segment becomes active when any 15 of its
connections are active. This introduces the possibility for error. It
is possible, by chance, that the dendrite reaches its threshold of 15
active connections by mixing parts of different patterns. However,
this kind of error is very unlikely, again due to the sparseness of
the representations.

Now we can see how a cell with one or two dozen dendrite segments and
a few thousand synapses can recognize hundreds of separate states of
cell activity.

\section*{Transition Memory details}

Here we enumerate the steps performed by the Transition Memory. We start
where the Pattern Memory left off, with a set of active columns
representing the feed-forward input.

\begin{enumerate}
\item For each active column, check for cells in the column that are
  in a predictive state, and activate them. If no cells are in a
  predictive state, activate all the cells in the column. The
  resulting set of active cells is the representation of the input in
  the context of prior input.

\item For every dendrite segment on every cell in the region, count
  how many established synapses are connected to active cells. If the
  number exceeds a threshold, that dendrite segment is marked as
  active. Cells with active dendrite segments are put in the
  predictive state unless they are already active due to feed-forward
  input. Cells with no active dendrites and not active due to
  bottom-up input become or remain inactive. The collection of cells
  now in the predictive state is the prediction of the region.

\item When a dendrite segment becomes active, modify the permanence
  values of all the synapses associated with the segment. For every
  potential synapse on the active dendrite segment, increase the
  permanence of those synapses that are connected to active cells and
  decrement the permanence of those synapses connected to inactive
  cells. These changes to synapse permanence are marked as temporary.
  This modifies the synapses on segments that are already trained
  sufficiently to make the segment active, and thus lead to a
  prediction. 

\item Whenever a cell switches from being inactive to active due to
  feed-forward input, we traverse each potential synapse associated
  with the cell and remove any temporary marks. Thus we update the
  permanence of synapses only if they correctly predicted the
  feed-forward activation of the cell.

\item When a cell switches from either active state to inactive, undo
  any permanence changes marked as temporary for each potential
  synapse on this cell. We don't want to strengthen the permanence of
  synapses that incorrectly predicted the feed-forward activation of a
  cell.
\end{enumerate}

Note that only cells that are active due to feed-forward input
propagate activity {\em within} the region, otherwise predictions
would lead to further predictions. The active cells form the output of a region and
propagate to the {\em next} region in the hierarchy.

\section*{First order versus variable order sequences and prediction}

There is one more major topic to discuss before we end our discussion
on the Pattern and Transition Memory. It may not be of interest to all
readers and it is not needed to understand Chapters 3 and 4.
% TODO: chapter references

What is the effect of having more or fewer cells per column?
Specifically, what happens if we have only one cell per column?

In the example used earlier, we showed that a representation of an
input comprised of 100 active columns with 4 cells per column can be
encoded in $4^{100}$ different ways. Therefore, the same input can
appear in a many contexts without confusion. For example, if input
patterns represent words, then a region can remember many sentences
that use the same words over and over again and not get confused. A
word such as ``dog'' would have a unique representation in different
contexts. This ability permits an HTM region to make what are called
``variable order'' predictions.

A variable order prediction is not based solely on what is currently
happening, but on varying amounts of past context. An HTM region is a
variable order memory.

If we increase to 5 cells per column, the available number of
encodings of any particular input in our example would increase to
$5^{100}$, a huge increase over $4^{100}$. But both these numbers are
so large that for many practical problems the increase in capacity
might not be useful.

However, making the number of cells per column much smaller does make
a big difference.

If we go all the way to one cell per column, we lose the ability to
include context in our representations. An input to a region always
results in the same prediction, regardless of previous activity. With
one cell per column, the memory of an HTM region is a ``first order''
memory; predictions are based only on the current input.

First order prediction is ideally suited for one type of problem that
brains solve: static spatial inference. As stated earlier, a human
exposed to a brief visual image can recognize what the object is even
if the exposure is too short for the eyes to move. With hearing, you
always need to hear a sequence of patterns to recognize what it
is. Vision is usually like that, you usually process a stream of
visual images. But under certain conditions you can recognize an image
with a single exposure.

Temporal and static recognition might appear to require different
inference mechanisms. One requires recognizing sequences of patterns
and making predictions based on variable length context. The other
requires recognizing a static spatial pattern without using temporal
context. An HTM region with multiple cells per column is ideally
suited for recognizing time-based sequences, and an HTM region with
one cell per column is ideally suited to recognizing spatial
patterns. At Numenta, we have performed many experiments using
one-cell-per-column regions applied to vision problems. The details of
these experiments are beyond the scope of this chapter; however we
will cover the important concepts.

If we expose an HTM region to images, the columns in the region learn
to represent common spatial arrangements of pixels. The kind of
patterns learned are similar to what is observed in region V1 in
neocortex (a neocortical region extensively studied in biology),
typically lines and corners at different orientations. By training on
moving images, the HTM region learns transitions of these basic
shapes. For example, a vertical line at one position is often followed
by a vertical line shifted to the left or right. All the commonly
observed transitions of patterns are remembered by the HTM region.

Now what happens if we expose a region to an image of a vertical line
moving to the right? If our region has only one cell per column, it
will predict the line might next appear to the left or to the
right. It can't use the context of knowing where the line was in the
past and therefore know if it is moving left or right. What you find
is that these one-cell-per-column cells behave like ``complex cells''
in the neocortex. The predictive output of such a cell will be active
for a visible line in different positions, regardless of whether the
line is moving left or right or not at all. We have further observed
that a region like this exhibits stability to translation, changes in
scale, etc. while maintaining the ability to distinguish between
different images. This behavior is what is needed for spatial
invariance (recognizing the same pattern in different locations of an
image).

If we now do the same experiment on an HTM region with multiple cells
per column, we find that the cells behave like ``directionally-tuned
complex cells'' in the neocortex. The predictive output of a cell will
be active for a line moving to the left or a line moving to the right,
but not both.

Putting this all together, we make the following hypothesis. The
neocortex has to do both first order and variable order inference and
prediction. There are four or five layers of cells in each region of
the neocortex. The layers differ in several ways but they all have
shared columnar response properties and large horizontal connectivity
within the layer. We speculate that each layer of cells in neocortex
is performing a variation of the HTM inference and learning rules
described in this chapter. The different layers of cells play
different roles. For example it is known from anatomical studies that
layer 6 creates feedback in the hierarchy and layer 5 is involved in
motor behavior. The two primary feed-forward layers of cells are
layers 4 and 3. We speculate that one of the differences between
layers 4 and 3 is that the cells in layer 4 are acting independently,
i.e., one cell per column, whereas the cells in layer 3 are acting as
multiple cells per column. Thus regions in the neocortex near sensory
input have both first order and variable order memory. The first order
sequence memory (roughly corresponding to layer 4 neurons) is useful
in forming representations that are invariant to spatial changes. The
variable order sequence memory (roughly corresponding to layer 3
neurons) is useful for inference and prediction of moving images.

In summary, we hypothesize that the algorithms similar to those
described in this chapter are at work in all layers of neurons in the
neocortex. The layers in the neocortex vary in significant details
which make them play different roles related to feed-forward
vs. feedback, attention, and motor behavior. In regions close to
sensory input, it is useful to have a layer of neurons performing
first order memory as this leads to spatial invariance.

At Numenta, we have experimented with first order (single cell per
column) HTM regions for image recognition problems. We also have
experimented with variable order (multiple cells per column) HTM
regions for recognizing and predicting variable order sequences. In
the future, it would be logical to try to combine these in a single
region and to extend the algorithms to other purposes. However, we
believe many interesting problems can be addressed with the equivalent
of single- layer, multiple-cell-per-column regions, either alone or in
a hierarchy.

\chapter{Pattern Memory Implementation and Pseudocode}
\label{chapter:pattern-memory}

This chapter contains the detailed pseudocode for a first
implementation of the spatial pooler function. The input to this code
is an array of bottom-up binary inputs from sensory data or the
previous level. The code computes activeColumns(t) --- the list of
columns that win due to the bottom-up input at time t. This list is
then sent as input to the temporal pooler routine described in the
next chapter, i.e., activeColumns(t) is the output of the spatial
pooling routine.

The pseudocode is split into three distinct phases that occur in
sequence:

\begin{description}
\item[Phase 1:] compute the overlap with the current input for each column
\item[Phase 2:] compute the winning columns after inhibition
\item[Phase 3:] update synapse permanence and internal variables
\end{description}

Although spatial pooler learning is inherently online, you can turn
off learning by simply skipping Phase 3.

The rest of the chapter contains the pseudocode for each of the three
steps. The various data structures and supporting routines used in the
code are defined at the end.

\subsection*{Initialization}

Prior to receiving any inputs, the region is initialized by computing
a list of initial potential synapses for each column. This consists of
a random set of inputs selected from the input space. Each input is
represented by a synapse and assigned a random permanence value. The
random permanence values are chosen with two criteria. First, the
values are chosen to be in a small range around connectedPerm (the
minimum permanence value at which a synapse is considered
``connected''). This enables potential synapses to become connected (or
disconnected) after a small number of training iterations. Second,
each column has a natural center over the input region, and the
permanence values have a bias towards this center (they have higher
values near the center).

\subsection*{Phase 1: Overlap}

Given an input vector, the first phase calculates the overlap of each
column with that vector. The overlap for each column is simply the
number of connected synapses with active inputs, multiplied by its
boost. If this value is below minOverlap, we set the overlap score to
zero.   

\begin{lstlisting}[numbers=left]
for c in columns

  overlap(c) = 0
  for s in connectedSynapses(c)
    overlap(c) = overlap(c) + input(t, s.sourceInput)

  if overlap(c) < minOverlap then
    overlap(c) = 0
  else
    overlap(c) = overlap(c) * boost(c)
\end{lstlisting}

\subsection*{Phase 2: Inhibition}
The second phase calculates which columns remain as winners after the
inhibition step. desiredLocalActivity is a parameter that controls the
number of columns that end up winning. For example, if
desiredLocalActivity is 10, a column will be a winner if its overlap
score is greater than the score of the 10th highest column within its
inhibition radius.

\begin{lstlisting}[numbers=left,firstnumber=11,mathescape]
for c in columns

  minLocalActivity = kthScore(neighbors(c), desiredLocalActivity)
  if overlap(c) > 0 and overlap(c) $\geq$ minLocalActivity then
    activeColumns(t).append(c)

\end{lstlisting}

\subsection*{Phase 3: Learning}
The third phase performs learning; it updates the permanence values of
all synapses as necessary, as well as the boost and inhibition radius.

The main learning rule is implemented in lines 20--26. For winning
columns, if a synapse is active, its permanence value is incremented,
otherwise it is decremented. Permanence values are constrained to be
between 0 and 1.

Lines 28--36 implement boosting. There are two separate boosting
mechanisms in place to help a column learn connections. If a column
does not win often enough (as measured by activeDutyCycle), its
overall boost value is increased (line 30--32). Alternatively, if a
column's connected synapses do not overlap well with any inputs often
enough (as measured by overlapDutyCycle), its permanence values are
boosted (line 34--36). Note: once learning is turned off, boost(c) is
frozen.  Finally, at the end of Phase 3 the inhibition radius is
recomputed (line 38).

\begin{lstlisting}[numbers=left,firstnumber=18]
for c in activeColumns(t)

  for s in potentialSynapses(c)
    if active(s) then
      s.permanence += permanenceInc
      s.permanence = min(1.0, s.permanence)
    else
      s.permanence -= permanenceDec
      s.permanence = max(0.0, s.permanence)

for c in columns:

  minDutyCycle(c) = 0.01 * maxDutyCycle(neighbors(c))
  activeDutyCycle(c) = updateActiveDutyCycle(c)
  boost(c) = boostFunction(activeDutyCycle(c), minDutyCycle(c))

  overlapDutyCycle(c) = updateOverlapDutyCycle(c)
  if overlapDutyCycle(c) < minDutyCycle(c) then
    increasePermanences(c, 0.1*connectedPerm)

inhibitionRadius = averageReceptiveFieldSize()

\end{lstlisting}

\subsection*{Supporting data structures and routines}

The following variables and data structures are used in the pseudocode:

\begin{description}
\item[columns] List of all columns.
\item[input(t,j)] The input to this level at time t. input(t, j) is 1
  if the j'th input is on.
\item[overlap(c)] The spatial pooler overlap of column c with a
  particular input pattern.
\item[activeColumns(t)] List of column indices that are winners due to
  bottom-up input.
\item[desiredLocalActivity] A parameter controlling the number of
  columns that will be winners after the inhibition step.
\item[inhibitionRadius] Average connected receptive field size of the
  columns.
\item[neighbors(c)] A list of all the columns that are within
  inhibitionRadius of column c.
\item[minOverlap] A minimum number of inputs that must be active for a
  column to be considered during the inhibition step.
\item[boost(c)] The boost value for column c as computed during
  learning---used to increase the overlap value for inactive columns.
\item[synapse] A data structure representing a synapse---contains a
  permanence value and the source input index.
\item[connectedPerm] If the permanence value for a synapse is greater
  than this value, it is said to be connected.
\item[potentialSynapses(c)] The list of potential synapses and their
  permanence values.
\item[connectedSynapses(c)] A subset of potentialSynapses(c) where the
  permanence value is greater than connectedPerm. These are the
  bottom-up inputs that are currently connected to column c.
\item[permanenceInc] Amount permanence values of synapses are
  incremented during learning.
\item[permanenceDec] Amount permanence values of synapses are
  decremented during learning.
\item[activeDutyCycle(c)] A sliding average representing how often
  column c has been active after inhibition (e.g. over the last 1000
  iterations).
\item[overlapDutyCycle(c)] A sliding average representing how often
  column c has had significant overlap (i.e., greater than minOverlap)
  with its inputs (e.g. over the last 1000 iterations).
\item[minDutyCycle(c)] A variable representing the minimum desired
  firing rate for a cell. If a cell's firing rate falls below this
  value, it will be boosted. This value is calculated as 1\% of the
  maximum firing rate of its neighbors.
\end{description}

The following supporting routines are used in the above code.

\begin{description}
\item[kthScore(cols,k)] Given the list of columns, return the k'th
  highest overlap value.

\item[updateActiveDutyCycle(c)] Computes a moving average of how often
  column c has been active after inhibition.

\item[updateOverlapDutyCycle(c)] Computes a moving average of how
  often column c has overlap greater than minOverlap.

\item[averageReceptiveFieldSize()] The radius of the average connected
  receptive field size of all the columns. The connected receptive
  field size of a column includes only the connected synapses (those
  with permanence values $\ge$ connectedPerm). This is used to determine
  the extent of lateral inhibition between columns.

\item[maxDutyCycle(cols)] Returns the maximum active duty cycle of the
  columns in the given list of columns.

\item[increasePermanences(c, s)] Increase the permanence value of
  every synapse in column c by a scale factor s.

\item[boostFunction(c)] Returns the boost value of a column. The boost
  value is a scalar $\ge 1$. If activeDutyCyle(c) is above
  minDutyCycle(c), the boost value is 1. The boost increases linearly
  once the column's activeDutyCyle starts falling below its
  minDutyCycle.
\end{description}



\chapter{Transition Memory Implementation and Pseudocode}
\label{chapter:transition-memory}
This chapter contains the detailed pseudocode for a first
implementation of the Transition Memory function. The input to this code
is activeColumns(t), as computed by the Pattern Memory. The code
computes the active and predictive state for each cell at the current
timestep, t. The boolean OR of the active and predictive states for
each cell forms the output of the Transition Memory for the next level.

The pseudocode is split into three distinct phases that occur in sequence:
\begin{description}
\item[Phase 1:] compute the active state, activeState(t), for each cell
\item[Phase 2:] compute the predicted state, predictiveState(t), for each cell
\item[Phase 3:] update synapses
\end{description}

\section*{Transition Memory pseudocode: inference alone}

\subsection*{Phase 1}
The first phase calculates the active state for each cell. For each
winning column we determine which cells should become active. If the
bottom-up input was predicted by any cell (i.e., its predictiveState
was 1 due to a sequence segment in the previous time step), then those
cells become active (lines 4--9). If the bottom-up input was unexpected
(i.e., no cells had predictiveState output on), then each cell in the
column becomes active (lines 11--13).

Phase 3 is only required for learning. However, unlike spatial
pooling, Phases 1 and 2 contain some learning-specific operations when
learning is turned on. Since temporal pooling is significantly more
complicated than spatial pooling, we first list the inference-only
version of the Transition Memory, followed by a version that combines
inference and learning. A description of some of the implementation
details, terminology, and supporting routines are at the end of the
chapter, after the pseudocode.

\begin{lstlisting}[numbers=left]
for c in activeColumns(t)

  buPredicted = false
  for i = 0 to cellsPerColumn - 1
    if predictiveState(c, i, t-1) == true then
      s = getActiveSegment(c, i, t-1, activeState)
      if s.sequenceSegment == true then
        buPredicted = true
        activeState(c, i, t) = 1

  if buPredicted == false then
    for i = 0 to cellsPerColumn - 1
      activeState(c, i, t) = 1
\end{lstlisting}

\subsection*{Phase 2}
The second phase calculates the predictive state for each cell. A cell
will turn on its predictiveState if any one of its segments becomes
active, i.e., if enough of its horizontal connections are currently
firing due to feed-forward input.

\begin{lstlisting}[numbers=left,firstnumber=14]
for c, i in cells
  for s in segments(c, i)
    if segmentActive(c, i, s, t) then
      predictiveState(c, i, t) = 1
\end{lstlisting}

\section*{Transition Memory pseudocode: combined inference and learning}

\subsection*{Phase 1}

The first phase calculates the activeState for each cell that is in a
winning column. For those columns, the code further selects one cell
per column as the learning cell (learnState). The logic is as follows:
if the bottom-up input was predicted by any cell (i.e., its
predictiveState output was 1 due to a sequence segment), then those
cells become active (lines 23--27). If that segment became active from
cells chosen with learnState on, this cell is selected as the learning
cell (lines 28--30). If the bottom-up input was not predicted, then
all cells in the become active (lines 32--34). In addition, the best
matching cell is chosen as the learning cell (lines 36--41) and a new
segment is added to that cell.

\begin{lstlisting}[numbers=left,firstnumber=18]
for c in activeColumns(t)

  buPredicted = false
  lcChosen = false
  for i = 0 to cellsPerColumn - 1
    if predictiveState(c, i, t-1) == true then
      s = getActiveSegment(c, i, t-1, activeState)
      if s.sequenceSegment == true then
        buPredicted = true
        activeState(c, i, t) = 1
        if segmentActive(s, t-1, learnState) then
          lcChosen = true
          learnState(c, i, t) = 1

  if buPredicted == false then
    for i = 0 to cellsPerColumn - 1
      activeState(c, i, t) = 1

  if lcChosen == false then
    I,s = getBestMatchingCell(c, t-1)
    learnState(c, i, t) = 1
    sUpdate = getSegmentActiveSynapses (c, i, s, t-1, true)
    sUpdate.sequenceSegment = true
    segmentUpdateList.add(sUpdate)
\end{lstlisting}

\subsection*{Phase 2}

The second phase calculates the predictive state for each cell. A cell
will turn on its predictive state output if one of its segments
becomes active, i.e., if enough of its lateral inputs are currently
active due to feed-forward input. In this case, the cell queues up the
following changes: a) reinforcement of the currently active segment
(lines 47--48), and b) reinforcement of a segment that could have
predicted this activation, i.e., a segment that has a (potentially
weak) match to activity during the previous time step (lines 50--53).

\begin{lstlisting}[numbers=left,firstnumber=42]
for c, i in cells
  for s in segments(c, i)
    if segmentActive(s, t, activeState) then
      predictiveState(c, i, t) = 1

      activeUpdate = getSegmentActiveSynapses(c, i, s, t, false)
      segmentUpdateList.add(activeUpdate)

      predSegment = getBestMatchingSegment(c, i, t-1)
      predUpdate = getSegmentActiveSynapses(c, i, predSegment, t-1, true)
      segmentUpdateList.add(predUpdate)
\end{lstlisting}

\subsection*{Phase 3}

The third and last phase actually carries out learning. In this phase
segment updates that have been queued up are actually implemented once
we get feed-forward input and the cell is chosen as a learning cell
(lines 56--57). Otherwise, if the cell ever stops predicting for any
reason, we negatively reinforce the segments (lines 58--60).

\begin{lstlisting}[numbers=left,firstnumber=54]
for c, i in cells
  if learnState(s, i, t) == 1 then
    adaptSegments (segmentUpdateList(c, i), true)
    segmentUpdateList(c, i).delete()
  else if predictiveState(c, i, t) == 0 and predictiveState(c, i, t-1)==1 then
    adaptSegments (segmentUpdateList(c,i), false)
    segmentUpdateList(c, i).delete()
\end{lstlisting}

\subsection*{Implementation details and terminology}
In this section we describe some of the details of our Transition Memory
implementation and terminology. Each cell is indexed using two
numbers: a column index, c, and a cell index, i. Cells maintain a list
of dendrite segments, where each segment contains a list of synapses
plus a permanence value for each synapse. Changes to a cell's synapses
are marked as temporary until the cell becomes active from
feed-forward input. These temporary changes are maintained in
segmentUpdateList. 

The implementation of potential synapses is different from the
implementation in the Pattern Memory. In the Pattern Memory, the
complete list of potential synapses is represented as an explicit
list. In the Transition Memory, each segment can have its own (possibly
large) list of potential synapses. In practice maintaining a long list
for each segment is computationally expensive and memory
intensive. Therefore in the Transition Memory, we randomly add active
synapses to each segment during learning (controlled by the parameter
newSynapseCount). This optimization has a similar effect to
maintaining the full list of potential synapses, but the list per
segment is far smaller while still maintaining the possibility of
learning new temporal patterns.

The pseudocode also uses a small state machine to keep track of the
cell states at different time steps. We maintain three different
states for each cell. The arrays activeState and predictiveState keep
track of the active and predictive states of each cell at each time
step. The array learnState determines which cell outputs are used
during learning. When an input is unexpected, all the cells in a
particular column become active in the same time step. Only one of
these cells (the cell that best matches the input) has its learnState
turned on. We only add synapses from cells that have learnState set to
one (this avoids overrepresenting a fully active column in dendritic
segments).

The following data structures are used in the Transition Memory pseudocode:

\begin{description}
\item[cell(c,i)] A list of all cells, indexed by i and c.
\item[cellsPerColumn] Number of cells in each column.
\item[activeColumns(t)] List of column indices that are winners due to
  bottom-up input (this is the output of the Pattern Memory).
\item[activeState(c, i, t)] A boolean vector with one number per
  cell. It represents the active state of the column c cell i at time
  t given the current feed-forward input and the past temporal
  context. activeState(c, i, t) is the contribution from column c cell
  i at time t. If 1, the cell has current feed-forward input as well
  as an appropriate temporal context.
\item[predictiveState(c, i, t)] A boolean vector with one number per
  cell. It represents the prediction of the column c cell i at time t,
  given the bottom-up activity of other columns and the past temporal
  context. predictiveState(c, i, t) is the contribution of column c
  cell i at time t. If 1, the cell is predicting feed-forward input in
  the current temporal context.
\item[learnState(c, i, t)] A boolean indicating whether cell i in
  column c is chosen as the cell to learn on.
\item[activationThreshold] A boolean indicating whether cell i in
  column c is chosen as the cell to learn on.
\item[learningRadius] The area around a Transition Memory cell from
  which it can get lateral connections.
\item[initialPerm] Initial permanence value for a synapse.
\item[connectedPerm] If the permanence value for a synapse is greater
  than this value, it is said to be connected.
\item[minThreshold] Minimum segment activity for learning.
\item[newSynapseCount] The maximum number of synapses added to a
  segment during learning.
\item[permanenceInc] Amount permanence values of synapses are
  incremented when activity-based learning occurs.
\item[permanenceDec] Amount permanence values of synapses are
  decremented when activity-based learning occurs.
\item[segmentUpdate] Data structure holding three pieces of
  information required to update a given segment: a) segment index (-1
  if it's a new segment), b) a list of existing active synapses, and
  c) a flag indicating whether this segment should be marked as a
  sequence segment (defaults to false).
\item[segmentUpdateList] A list of segmentUpdate
  structures. segmentUpdateList(c,i) is the list of changes for cell i
  in column c.
\end{description}

The following supporting routines are used in the above code:

\begin{description}
\item[segmentActive(s, t, state)] This routine returns true if the
  number of connected synapses on segment s that are active due to the
  given state at time t is greater than activationThreshold. The
  parameter state can be activeState, or learnState.
\item[getActiveSegment(c, i, t, state)] For the given column c cell i,
  return a segment index such that segmentActive(s,t, state) is
  true. If multiple segments are active, sequence segments are given
  preference. Otherwise, segments with most activity are given
  preference.
\item[getBestMatchingSegment(c, i, t)] For the given column c cell i
  at time t, find the segment with the largest number of active
  synapses. This routine is aggressive in finding the best match. The
  permanence value of synapses is allowed to be below
  connectedPerm. The number of active synapses is allowed to be below
  activationThreshold, but must be above minThreshold. The routine
  returns the segment index. If no segments are found, then an index
  of -1 is returned.
\item[getBestMatchingCell(c)] For the given column, return the cell
  with the best matching segment (as defined above). If no cell has a
  matching segment, then return the cell with the fewest number of
  segments.
\item[getSegmentActiveSynapses(c, i, t, s, newSynapses=false)] Return
  a segmentUpdate data structure containing a list of proposed changes
  to segment s. Let activeSynapses be the list of active synapses
  where the originating cells have their activeState output = 1 at
  time step t. (This list is empty if s = -1 since the segment doesn't
  exist.) newSynapses is an optional argument that defaults to
  false. If newSynapses is true, then newSynapseCount -
  count(activeSynapses) synapses are added to activeSynapses. These
  synapses are randomly chosen from the set of cells that have
  learnState output = 1 at time step t.
\item[adaptSegments(segmentList, positiveReinforcement)] This function
  iterates through a list of segmentUpdate's and reinforces each
  segment. For each segmentUpdate element, the following changes are
  performed. If positiveReinforcement is true then synapses on the
  active list get their permanence counts incremented by
  permanenceInc. All other synapses get their permanence counts
  decremented by permanenceDec. If positiveReinforcement is false,
  then synapses on the active list get their permanence counts
  decremented by permanenceDec. After this step, any synapses in
  segmentUpdate that do yet exist get added with a permanence count of
  initialPerm.
\end{description}



\appendix

\chapter{A Comparison between Biological Neurons and HTM Cells}
\label{appendix:compare-neuron-cell}

\begin{figure}
\resizebox{\textwidth}{!}{\includegraphics{figures/Appendix_A_1.png}}
\end{figure}

The image above shows a picture of a biological neuron on the left, a
simple artificial neuron in the middle, and an HTM neuron or ``cell''
on the right. The purpose of this appendix is to provide a better
understanding of HTM cells and how they work by comparing them to real
neurons and simpler artificial neurons.

Real neurons are tremendously complicated and varied. We will focus on
the most general principles and only those that apply to our
model. Although we ignore many details of real neurons, the cells used
in the HTM cortical learning algorithms are far more realistic than
the artificial neurons used in most neural networks. All the elements
included in HTM cells are necessary for the operation of an HTM
region.

\section*{Biological neurons}
Neurons are the information carrying cells in the brain. The image on
the left above is of a typical excitatory neuron. The visual
appearance of a neuron is dominated by the branching dendrites. All
the excitatory inputs to a neuron are via synapses aligned along the
dendrites. In recent years our knowledge of neurons has advanced
considerably. The biggest change has been in realizing that the
dendrites of a neuron are not just conduits to bring inputs to the
cell body. We now know the dendrites are complex non-linear processing
elements in themselves. The HTM cortical learning algorithms take
advantage of these non-linear properties.

Neurons have several parts.

\subsection*{Cell body}
The cell body is the small volume in the center of the neuron. The
output of the cell, the axon, originates at the cell body. The inputs
to the cell are the synapses aligned along the dendrites which feed to
the cell body.

\subsection*{Proximal Dendrites}
The dendrite branches closest to the cell body are called proximal
dendrites. In the diagram some of the proximal dendrites are marked
with green lines.

Multiple active synapses on proximal dendrites have a roughly linear
additive effect at the cell body. Five active synapses will lead to
roughly five times the depolarization at the cell body compared to one
active synapse. In contrast, if a single synapse is activated
repeatedly by a quick succession of action potentials, the second,
third, and subsequent action potentials have much less effect at the
cell body, than the first.

Therefore, we can say that inputs to the proximal dendrites sum
linearly at the cell body, and that rapid spikes arriving at a single
synapse will have only a slightly larger effect than a single spike.

The feed-forward connections to a region of neocortex preferentially
connect to the proximal dendrites. This has been reported at least for
layer 4 neurons, the primary input layer of neurons in each region.

\subsection*{Distal Dendrites}
The dendrite branches farther from the cell body are called distal
dendrites. In the diagram some of the distal dendrites are marked with
blue lines.

Distal dendrites are thinner than proximal dendrites. They connect to
other dendrites at branches in the dendritic tree and do not connect
directly to the cell body. These differences give distal dendrites
unique electrical and chemical properties. When a single synapse is
activated on a distal dendrite, it has a minimal effect at the cell
body. The depolarization that occurs locally to the synapse weakens by
the time it reaches the cell body. For many years this was viewed as a
mystery. It seemed the distal synapses, which are the majority of
synapses on a neuron, couldn't do much.

We now know that sections of distal dendrites act as semi-independent
processing regions. If enough synapses become active at the same time
within a short distance along the dendrite, they can generate a
dendritic spike that can travel to the cell body with a large
effect. For example, twenty active synapses within 40 $\mu$m of each
other will generate a dendritic spike.

Therefore, we can say that the distal dendrites act like a set of
threshold coincidence detectors.

The synapses formed on distal dendrites are predominantly from other
cells nearby in the region.

The image shows a large dendrite branch extending upwards which is
called the apical dendrite. One theory says that this structure allows
the neuron to locate several distal dendrites in an area where they
can more easily make connections to passing axons. In this
interpretation, the apical dendrite acts as an extension of the cell.

\subsection*{Synapses}
A typical neuron might have several thousand synapses. The large
majority (perhaps 90\%) of these will be on distal dendrites, and the
rest will be on proximal dendrites.

For many years it was assumed that learning involved strengthening and
weakening the effect or ``weight'' of synapses. Although this effect
has been observed, each synapse is somewhat stochastic. When
activated, it will not reliably release a neurotransmitter. Therefore
the algorithms used by the brain cannot depend on precision or
fidelity of individual synapse weights.

Further, we now know that entire synapses form and un-form
rapidly. This flexibility represents a powerful form of learning and
better explains the rapid acquisition of knowledge. A synapse can only
form if an axon and a dendrite are within a certain distance, leading
to the concept of ``potential'' synapses. With these assumptions,
learning occurs largely by forming valid synapses from potential
synapses.

\subsection*{Neuron Output}
The output of a neuron is a spike, or ``action potential,'' which
propagates along the axon. The axon leaves the cell body and almost
always splits in two. One branch travels horizontally making many
connections with other cells nearby. The other branch projects to
other layers of cells or elsewhere in the brain. In the image of the
neuron above, the axon was not visible. We added a line and two arrows
to represent that axon.

Although the actual output of a neuron is always a spike, there are
different views on how to interpret this. The predominant view
(especially in regards to the neocortex) is that the rate of spikes is
what matters. Therefore the output of a cell can be viewed as a scalar
value.

Some neurons also exhibit a ``bursting'' behavior, a short and fast
series of a few spikes that are different than the regular spiking
pattern.

The above description of a neuron is intended to give a brief
introduction to neurons. It focuses on attributes that correspond to
features of HTM cells and leaves out many details. Not all the
features just described are universally accepted. We include them
because they are necessary for our models. What is known about neurons
could easily fill several books, and active research on neurons
continues today.

\section*{Simple artificial neurons}
The middle image at the beginning of this Appendix shows a neuron-like
element used in many classic artificial neural network models. These
artificial neurons have a set of synapses each with a weight. Each
synapse receives a scalar activation, which is multiplied by the
synapse weight. The output of all the synapses is summed in a
non-linear fashion to produce an output of the artificial
neuron. Learning occurs by adjusting the weights of the synapses and
perhaps the non- linear function.

This type of artificial neuron, and variations of it, has proven
useful in many applications as a valuable computational tool. However,
it doesn't capture much of the complexity and processing power of
biological neurons. If we want to understand and model how an ensemble
of real neurons works in the brain we need a more sophisticated neuron
model.

\section*{HTM cells}
In our illustration, the image on the right depicts a cell used in the
HTM cortical learning algorithms. An HTM cell captures many of the
important capabilities of real neurons but also makes several
simplifications.

\subsection*{Proximal Dendrite}
Each HTM cell has a single proximal dendrite. All feed-forward inputs
to the cell are made via synapses (shown as green dots). The activity
of synapses is linearly summed to produce a feed-forward activation
for the cell.

We require that all cells in a column have the same feed-forward
response. In real neurons this would likely be done by a type of
inhibitory cell. In HTMs we simply force all the cells in a column to
share a single proximal dendrite.

To avoid having cells that never win in the competition with
neighboring cells, an HTM cell will boost its feed-forward activation
if it is not winning enough relative to its neighbors. Thus there is a
constant competition between cells. Again, in an HTM we model this as
a competition between columns, not cells. This competition is not
illustrated in the diagram.

Finally, the proximal dendrite has an associated set of potential
synapses which is a subset of all the inputs to a region. As the cell
learns, it increases or decreases the ``permanence'' value of all the
potential synapses on the proximal dendrite. Only those potential
synapses that are above a threshold are valid.

As mentioned earlier, the concept of potential synapses comes from
biology where it refers to axons and dendrites that are close enough
to form a synapse. We extend this concept to a larger set of potential
connections for an HTM cell. Dendrites and axons on biological neurons
can grow and retract as learning occurs and therefore the set of
potential synapses changes with growth. By making the set of potential
synapses on an HTM cell large, we roughly achieve the same result as
axon and dendrite growth. The set of potential synapses is not shown.

The combination of competition between columns, learning from a set of
potential synapses, and boosting underutilized columns gives a region
of HTM neurons a powerful plasticity also seen in brains. An HTM
region will automatically adjust what each column represents (via
changes to the synapses on the proximal dendrites) if the input
changes, or the number of columns increases or decreases.

\subsection*{Distal Dendrites}
Each HTM cell maintains a list of distal dendrite segments. Each
segment acts like a threshold detector. If the number of active
synapses on any segment (shown as blue dots on the earlier diagram) is
above a threshold, the segment becomes active, and the associated cell
enters the predictive state. The predictive state of a cell is the OR
of the activations of its segments.

A dendrite segment remembers the state of the region by forming
connections to cells that were active together at a point in time. The
segment remembers a state that precedes the cell becoming active due
to feed-forward input. Thus the segment is looking for a state that
predicts that its cell will become active. A typical threshold for a
dendrite segment is 15. If 15 valid synapses on a segment are active
at once, the dendrite becomes active. There might be hundreds or
thousands of cells active nearby, but connecting to only 15 is
sufficient to recognize the larger pattern.

Each distal dendrite segment also has an associated set of potential
synapses. The set of potential synapses is a subset of all the cells
in a region. As the segment learns, it increases or decreases the
permanence value of all its potential synapses. Only those potential
synapses that are above a threshold are valid.

In one implementation, we use a fixed number of dendrite segments per
cell. In another implementation, we add and delete segments while
training. Both methods can work. If we have a fixed number of dendrite
segments per cell, it is possible to store several different sets of
synapses on the same segment. For example, say we have 20 valid
synapses on a segment and a threshold of 15. (In general we want the
threshold to be less than the number of synapses to improve noise
immunity.) The segment can now recognize one particular state of the
cells nearby. What would happen if we added another 20 synapses to the
same segment representing an entirely different state of cells nearby?
It introduces the possibility of error because the segment could add 8
active synapses from one pattern and 7 active synapses from the other
and become active incorrectly. We have found experimentally that up to
20 different patterns can be stored on one segment before errors
occur. Therefore an HTM cell with a dozen dendrite segments can
participate in many different predictions.

\subsection*{Synapses}

Synapses on an HTM cell have a binary weight. There is nothing in the
HTM model that precludes scalar synapse weights, but due to the use of
sparse distributed patterns we have not yet had a need to use scalar
weights.

However, synapses on an HTM cell have a scalar value called
``permanence'' which is adjusted during learning. A 0.0 permanence value
represents a potential synapse which is not valid and has not
progressed at all towards becoming a valid synapse. A permanence value
above a threshold (typically 0.2) represents a synapse that has just
connected but could easily be un-connected. A high permanence value,
for example 0.9, represents a synapse that is connected and cannot
easily be un- connected.

The number of valid synapses on the proximal and distal dendrite
segments of an HTM cell is not fixed. It changes as the cell is
exposed to patterns. For example, the number of valid synapses on the
distal dendrites is dependent on the temporal structure of the
data. If there are no persistent temporal patterns in the input to the
region, then all the synapses on distal segments would have low
permanence values and very few synapses would be valid. If there is a
lot of temporal structure in the input stream, then we will find many
valid synapses with high permanence.

\subsection*{Cell Output}

An HTM cell has two different binary outputs: 1) the cell is active
due to feed- forward input (via the proximal dendrite), and 2) the
cell is active due to lateral connections (via the distal dendrite
segments). The former is called the ``active state'' and the latter is
called the ``predictive state.''

In the earlier diagram, the two outputs are represented by the two
lines exiting the square cell body. The left line is the feed-forward
active state, while the right line is the predictive state.

Only the feed-forward active state is connected to other cells in the
region, ensuring that predictions are always based on the current
input (plus context). We don't want to make predictions based on
predictions. If we did, almost all the cells in the region would be in
the predictive state after a few iterations.

The output of the region is a vector representing the state of all the
cells. This vector becomes the input to the next region of the
hierarchy if there is one. This output is the OR of the active and
predictive states. By combining both active and predictive states, the
output of our region will be more stable (slower changing) than the
input. Such stability is an important property of inference in a
region.

\section*{Suggested reading}

We are often asked to suggest reading materials to learn more about
neuroscience. The field of neuroscience is so large that a general
introduction requires looking at many different sources. New findings
are published in academic journals which are both hard to read and
hard to get access to if you don't have a university affiliation.

Here are two readily available books that a dedicated reader might
want to look at which are relevant to the topics in this appendix.

\begin{quote}
Stuart, Greg, Spruston, Nelson, H\"ausser, Michael, {\em Dendrites},
second edition (New York: Oxford University Press, 2008)
\end{quote}

This book is a good source on everything about dendrites. Chapter~16
discusses the non-linear properties of dendrite segments used in the
HTM cortical learning algorithms. It is written by Bartlett Mel who
has done much of the thinking in this field.

\begin{quote}
Mountcastle, Vernon B.~{\em Perceptual Neuroscience: The Cerebral
  Cortex} (Cambridge, Mass.: Harvard University Press, 1998)
\end{quote}

This book is a good introduction to everything about the
neocortex. Several of the chapters discuss cell types and their
connections. You can get a good sense of cortical neurons and their
connections, although it is too old to cover the latest knowledge of
dendrite properties.

\chapter{A Comparison of Layers in the Neocortex and an HTM Region}
\label{appendix:compare-layer-region}

This appendix describes the relationship between an HTM region and a
region of the biological neocortex.

Specifically, the appendix covers how the HTM cortical learning
algorithm, with its columns and cells, relates to the layered and
columnar architecture of the neocortex. Many people are confused by
the concept of ``layers'' in the neocortex and how it relates to an
HTM layer. Hopefully this appendix will resolve this confusion as well
as provide more insight into the biology underlying the HTM cortical
learning algorithm.

\section*{Circuitry of the neocortex}

The human neocortex is a sheet of neural tissue approximately 1,000
cm$^2$ in area and 2 mm thick. To visualize this sheet, think of a
cloth dinner napkin, which is a reasonable approximation of the area
and thickness of the neocortex. The neocortex is divided into dozens
of functional regions, some related to vision, others to audition, and
others to language, etc. Viewed under a microscope, the physical
characteristics of the different regions look remarkably similar.

There are several organizing principles seen in each region throughout
the neocortex.

\begin{figure}
\resizebox{\textwidth}{!}{\includegraphics{figures/Appendix_B_1.png}}
\end{figure}

\subsection*{Layers}
The neocortex is generally said to have six layers. Five of the layers
contain cells and one layer is mostly connections. The layers were
discovered over one hundred years ago with the advent of staining
techniques. The image above (from Cajal) shows a small slice of
neocortex exposed using three different staining methods. The vertical
axis spans the thickness of the neocortex, approximately 2 mm. The
left side of the image indicates the six layers. Layer 1, at the top,
is the non-cellular level. The ``WM'' at the bottom indicates the
beginning of the white matter, where axons from cells travel to other
parts of the neocortex and other parts of the brain.

The right side of the image is a stain that shows only myelinated
axons.  (Myelination is a fatty sheath that covers some but not all
axons.) In this part of the image you can see two of the main
organizing principles of the neocortex, layers and columns. Most axons
split in two immediately after leaving the body of the neuron. One
branch will travel mostly horizontally and the other branch will
travel mostly vertically. The horizontal branch makes a large number
of connections to other cells in the same or nearby layer, thus the
layers become visible in stains such as this. Bear in mind that this
is a drawing of a slice of neocortex. Most of the axons are coming in
and out of the plane of the image so the axons are longer than they
appear in the image. It has been estimated that there are between 2
and 4 kilometers of axons and dendrites in every cubic millimeter of
neocortex.

The middle section of the image is a stain that shows neuron bodies,
but does not show any dendrites or axons. You can see that the size
and density of the neurons also varies by layer. There is only a
little indication of columns in this particular image. You might
notice that there are some neurons in layer 1. The number of layer 1
neurons is so small that the layer is still referred to as a
non-cellular layer. Neuroscientists have estimated that there is
somewhere around 100,000 neurons in a cubic millimeter of neocortex.

The left part of the image is a stain that shows the body, axons, and
dendrites of just a few neurons. You can see that the size of the
dendrite ``arbors'' varies significantly in cells in different
layers. Also visible are some ``apical dendrites'' that rise from the
cell body making connections in other layers. The presence and
destination of apical dendrites is specific to each layer.

In short, the layered and columnar organization of the neocortex
becomes evident when the neural tissue is stained and viewed under a
microscope.

\subsection*{Variations of layers in different regions}
There is variation in the thickness of the layers in different regions
of the neocortex and some disagreement over the number of layers. The
variations depend on what animal is being studied, what region is
being looked at, and who is doing the looking. For example, in the
image above, layer 2 and layer 3 look easily distinguished, but
generally this is not the case. Some scientists report that they
cannot distinguish the two layers in the regions they study, so often
layer 2 and layer 3 are grouped together and called ``layer 2/3.''
Other scientists go the opposite direction, defining sub-layers such
as 3A and 3B.


Layer 4 is most well defined in those neocortical regions which are
closest to the sensory organs. While in some animals (for example
humans and monkeys), layer 4 in the first vision region is clearly
subdivided. In other animals it is not subdivided. Layer 4 mostly
disappears in regions hierarchically far from the sensory organs.

\subsection*{Columns}
The second major organizing principle of the neocortex is
columns. Some columnar organization is visible in stained images, but
most of the evidence for columns is based on how cells respond to
different inputs.

When scientists use probes to see what makes neurons become active,
they find that neurons that are vertically aligned, across different
layers, respond to roughly the same input.

\begin{figure}
\resizebox{\textwidth}{!}{\includegraphics{figures/Appendix_B_2.png}}
\end{figure}

This drawing illustrates some of the response properties of cells in
V1, the first cortical region to process information from the retina.

One of the first discoveries was that most cells in V1 respond to
lines or edges at different orientations at specific areas of the
retina. Cells that are vertically aligned in columns all respond to
edges with the same orientation. If you look carefully, you will see
that the drawing shows a set of small lines at different orientations
arrayed across the top of the section. These lines indicate what line
orientation cells at that location respond to. Cells that are
vertically aligned (within the thin vertical stripes) respond to the
lines of the same orientation.

There are several other columnar properties seen in V1, two of which
are shown in the drawing. There are ``ocular dominance columns'' where
cells respond to similar combinations of left and right eye
influence. And there are ``blobs'' where cells are primarily color
sensitive. The ocular dominance columns are the larger blocks in the
diagram. Each ocular dominance column includes a set of orientation
columns. The ``blobs'' are the dark ovals.

The general rule for neocortex is that several different response
properties are overlaid on one another, such as orientation and ocular
dominance. As you move horizontally across the cortical surface, the
combination of response properties exhibited by cells
changes. However, vertically aligned neurons share the same set of
response properties. This vertical alignment is true in auditory,
visual, and somatosensory areas. There is some debate amongst
neuroscientists whether this is true everywhere in the neocortex but
it appears to be true in most areas if not all.

\subsection*{Mini-columns}
The smallest columnar structure in the neocortex is the
mini-column. Mini-columns are about 30um in diameter and contain
80-100 neurons across all five cellular layers. The entire neocortex
is composed of mini-columns. You can visualize them as tiny pieces of
spaghetti stacked side by side. There are tiny gaps with few cells
between the mini-columns sometimes making them visible in stained
images.

\begin{figure}
\resizebox{\textwidth}{!}{\includegraphics{figures/Appendix_B_3.png}}
\end{figure}

On the left is a stained image that shows neuron cell bodies in part
of a neocortical slice. The vertical structure of mini-columns is
evident in this image. On the right is a conceptual drawing of a
mini-column (from Peters and Yilmez). In reality is skinnier than
this. Note there are multiple neurons in each layer in the column. All
the neurons in a mini-column will respond to similar inputs. For
example, in the drawing of a section of V1 shown previously, a
mini-column will contain cells that respond to lines of a particular
orientation with a particular ocular dominance preference. The cells
in an adjacent mini-column might respond to a slightly different line
orientation or different ocular dominance preference.

Inhibitory neurons play an essential role is defining
mini-columns. They are not visible in the image or drawing but
inhibitory neurons send axons in a straight path between mini-columns
partially giving them their physical separation. The inhibitory
neurons are also believed to help force all the cells in the
mini-column to respond to similar inputs.

The mini-column is the prototype for the column used in the HTM
cortical learning algorithm.

\subsection*{An exception to columnar responses}
There is a one exception to columnar responses that is relevant to the
HTM cortical learning algorithms. Usually scientists find what a cell
responds to by exposing an experimental animal to a simple
stimulus. For example, they might show an animal a single line in a
small part of the visual space to determine the response properties of
cells in V1. When using simple inputs, researchers find that cells
always will respond to the same input. However, if the simple input is
embedded in a video of a natural scene, cells become more selective. A
cell that reliably responds to an isolated vertical line will not
always respond when the vertical line is embedded in a complex moving
image of a natural scene.

In the HTM cortical learning algorithm, all HTM cells in a column
share the same feed-forward response properties, but in a learned
temporal sequence, only one of the cells in an HTM column becomes
active. This mechanism is the means of representing variable order
sequences and is analogous to the property just described for
neurons. A simple input with no context will cause all the cells in a
column to become active. The same input within a learned sequence will
cause just one cell to become active.

We are not suggesting that only one neuron within a mini-column will
be active at once. The HTM cortical learning algorithm suggests that
within a column, all the neurons within a layer would be active for an
unanticipated input and a subset of the neurons would be active for an
anticipated input.

\section*{Why are there layers and columns?}

No one knows for certain why there are layers and why there are
columns in the neocortex. HTM theory, however, proposes an answer. The
HTM cortical learning algorithm shows that a layer of cells organized
in columns can be a high capacity memory of variable order state
transitions. Stated more simply, a layer of cells can learn a lot of
sequences. Columns of cells that share the same feed-forward response
are the key mechanism for learning variable-order transitions.

This hypothesis explains why columns are necessary, but what about the
five layers? If a single cortical layer can learn sequences and make
predictions, why do we see five layers in the neocortex?

We propose that the different layers observed in the neocortex are all
learning sequences using the same basic mechanism but the sequences
learned in each layer are used in different ways. There is a lot we
don't understand about this, but we can describe the general
idea. Before we do, it will be helpful to describe what the neurons in
each layer connect to.

\begin{figure}
\resizebox{\textwidth}{!}{\includegraphics{figures/Appendix_B_4.png}}
\end{figure}

The above diagram illustrates two neocortical regions and the major
connections between them. These connections are seen throughout the
neocortex where two regions project to each other. The box on the left
represents a cortical region that is hierarchically lower than the
region (box) on the right, so feed-forward information goes from left
to right in the diagram. The down arrow projects to other areas of the
brain. Feedback information goes from right to left. Each region is
divided into layers. Layers 2 and 3 are shown together as layer 2/3.

The colored lines represent the output of neurons in the different
layers. These are bundles of axons originating from the neurons in the
layer. Recall that axons immediately split in two. One branch spreads
horizontally within the region, primarily within the same layer. Thus
all the cells in each layer are highly interconnected. The neurons and
horizontal connections are not shown in the diagram.

There are two feed-forward pathways, a direct path shown in orange and
an indirect path shown in green. Layer 4 is the primary feed-forward
input layer and receives input from both feed-forward pathways. Layer
4 projects to layer 3.

Layer 3 is also the origin of the direct feed-forward pathway. So the
direct forward pathway is limited to layer 4 and layer 3.

Some feed-forward connections skip layer 4 and go directly to layer
3. And, as mentioned above, layer 4 disappears in regions far from
sensory input. At that point, the direct forward pathway is just from
layer 3 to layer 3 in the next region.

The second feed-forward pathway (shown in green) originates in layer
5. Layer 3 cells make a connection to layer 5 cells as they pass on
their way to the next region. After exiting the cortical sheet, the
axons from layer 5 cells split again. One branch projects to
sub-cortical areas of the brain that are involved in motor
generation. These axons are believed to be motor commands (shown as
the down facing arrow). The other branch projects to a part of the
brain called the thalamus which acts as a gate. The thalamus either
passes the information onto the next region or blocks it.

Finally, the primary feedback pathway, shown in yellow, starts in
layer 6 and projects to layer 1. Cells in layers 2, 3, and 5 connect
to layer 1 via their apical dendrites (not shown). Layer 6 receives
input from layer 5.

This description is a limited summary of what is known about layer to
layer connections. But it is sufficient to understand our hypothesis
about why there are multiple layers if all the layers are learning
sequences.

\section*{Hypothesis on what the different layers do}

We propose that layers 3, 4 and 5 are all feed-forward layers and are
all learning sequences. Layer 4 is learning first order
sequences. Layer 3 is learning variable order sequences. And layer 5
is learning variable order sequences with timing. Let's look at each
of these in more detail.

\subsubsection*{Layer 4}
It is easy to learn first order sequences using the HTM cortical
learning algorithm. If we don't force the cells in a column to inhibit
each other, that is, the cells in a column don't differentiate in the
context of prior inputs, then first order learning will occur. In the
neocortex this would likely be accomplished by removing an inhibitory
effect between cells in the same column. In our computer models of the
HTM cortical learning algorithm, we just assign one cell per column,
which produces a similar result.

First order sequences are what are needed to form invariant
representations for spatial transformations of an input. In vision,
for example, x-y translation, scale, and rotation are all spatial
transformations. When an HTM region with first order memory is trained
on moving objects, it learns that different spatial patterns are
equivalent. The resulting HTM cells will behave like what are called
``complex cells'' in the neocortex. The HTM cells will stay active (in
the predictive state) over a range of spatial transformations.

At Numenta we have done vision experiments that verify this mechanism
works as expected, and that some spatial invariance is achieved within
each level. The details of these experiments are beyond the scope of
this appendix.

Learning first order sequences in layer 4 is consistent with finding
complex cells in layer 4, and for explaining why layer 4 disappears in
higher regions of neocortex. As you ascend the hierarchy at some point
it will no longer be possible to learn further spatial invariances as
the representations will already be invariant to them.

\subsubsection*{Layer 3}
Layer 3 is closest to the HTM cortical learning algorithm that we
described in Chapter 2. It learns variable order sequences and forms
predictions that are more stable than its input. Layer 3 always
projects to the next region in the hierarchy and therefore leads to
increased temporal stability within the hierarchy. Variable order
sequence memory leads to neurons called ``directionally-tuned complex
cells'' which are first observed in layer 3. Directionally-tuned
complex cells differentiate by temporal context, such as a line moving
left vs. a line moving right.

\subsubsection*{Layer 5}
The final feed-forward layer is layer 5. We propose that layer 5 is
similar to layer 3 with three differences. The first difference is
that layer 5 adds a concept of timing. Layer 3 predicts ``what'' will
happen next, but it doesn't tell you ``when'' it will happen. However,
many tasks require timing such as recognizing spoken words in which
the relative timing between sounds is important. Motor behavior is
another example; coordinated timing between muscle activations is
essential. We propose that layer 5 neurons predict the next state only
after the expected time. There are several biological details that
support this hypothesis. One is that layer 5 is the motor output layer
of the neocortex. Another is that layer 5 receives input from layer 1
that originates in a part of the thalamus (not shown in the
diagram). We propose that this information is how time is encoded and
distributed to many cells via a thalamic input to layer 1 (not shown
in the diagram).

The second difference between layer 3 and layer 5 is that we want
layer 3 to make predictions as a far into the future as possible,
gaining temporal stability. The HTM cortical learning algorithm
described in Chapter 2 does this. In contrast, we only want layer 5 to
predict the next element (at a specific time). We have not modeled
this difference but it would naturally occur if transitions were
always stored with an associated time.

The third difference between layer 3 and layer 5 can be seen in the
diagram. The output of layer 5 always projects to sub-cortical motor
centers, and the feed-forward path is gated by the thalamus. The
output of layer 5 is sometimes passed to the next region and sometimes
it is blocked. We (and others) propose this gating is related to
covert attention (covert attention is when you attend to an input
without motor behavior).

In summary, layer 5 combines specific timing, attention, and motor
behavior. There are many mysteries relating to how these play
together. The point we want to make is that a variation of the HTM
cortical learning algorithm could easily incorporate specific timing
and justify a separate layer in the cortex.

\subsubsection*{Layer 2 and layer 6}
Layer 6 is the origin of axons that feed back to lower regions. Much
less is known about layer 2. As mentioned above, the very existence of
layer 2 as unique from layer 3 is sometimes debated. We won't have
further to say about this question now other than to point out that
layers 2 and 6, like all the other layers, exhibit the pattern of
massive horizontal connections and columnar response properties, so we
propose that they, too, are running a variant of the HTM cortical
learning algorithm.

\subsection*{What does an HTM region correspond to in the neocortex?}
We have implemented the HTM cortical learning algorithm in two
flavors, one with multiple cells per column for variable order memory,
and one with a single cell per column for first order memory. We
believe these two flavors correspond to layer 3 and layer 4 in the
neocortex. We have not attempted to combine these two variants in a
single HTM region.

Although the HTM cortical learning algorithm (with multiple cells per
column) is closest to layer 3 in the neocortex, we have flexibility in
our models that the brain doesn't have. Therefore we can create hybrid
cellular layers that don't correspond to specific neocortical
layers. For example, in our model we know the order in which synapses
are formed on dendrite segments. We can use this information to
extract what is predicted to happen next from the more general
prediction of all the things that will happen in the future. We can
probably add specific timing in the same way. Therefore it should be
possible to create a single layer HTM region that combines the
functions of layer 3 and layer 5.

\section*{Summary}


The HTM cortical learning algorithm embodies what we believe is a
basic building block of neural organization in the neocortex. It shows
how a layer of horizontally-connected neurons learns sequences of
sparse distributed representations. Variations of the HTM cortical
learning algorithm are used in different layers of the neocortex for
related, but different purposes.

We propose that feed-forward input to a neocortical region, whether to
layer 4 or layer 3, projects predominantly to proximal dendrites,
which with the assistance of inhibitory cells, creates a sparse
distributed representation of the input. We propose that cells in
layers 2, 3, 4, 5, and 6 share this sparse distributed
representation. This is accomplished by forcing all cells in a column
that spans the layers to respond to the same feed-forward input.

We propose that layer 4 cells, when they are present, use the HTM
cortical learning algorithm to learn first-order temporal transitions
which make representations that are invariant to spatial
transformations. Layer 3 cells use the HTM cortical learning algorithm
to learn variable-order temporal transitions and form stable
representations that are passed up the cortical hierarchy. Layer 5
cells learn variable-order transitions with timing. We don't have
specific proposals for layer 2 and layer 6. However, due to the
typical horizontal connectivity in these layers it is likely they,
too, are learning some form of sequence memory.


\chapter*{Glossary}
\addcontentsline{toc}{chapter}{Glossary}
\label{glossary}

\begin{description}
\item[Active State]{a state in which Cells are active due to
  Feed-Forward input}

\item[Bottom-Up]{synonym to Feed-Forward}

\item[Cells]{HTM equivalent of a Neuron\\{\em Cells are organized into
    columns in HTM regions.}}

\item[Coincident Activity]{two or more Cells are active at the same
  time}

\item[Column]{a group of one or more Cells that function as a unit in
  an HTM Region\\{\em Cells within a column represent the same
    feed-forward input, but in different contexts.}}

\item[Dendrite Segment]{a unit of integration of Synapses associated
  with Cells and Columns\\{\em HTMs have two different types of
    dendrite segments. One is associated with lateral connections to a
    cell. When the number of active synapses on the dendrite segment
    exceeds a threshold, the associated cell enters the predictive
    state. The other is associated with feed-forward connections to a
    column. The number of active synapses is summed to generate the
    feed-forward activation of a column.}}

\item[Desired Density]{desired percentage of Columns active due to
  Feed-Forward input to a Region\\{\em The percentage only applies
    within a radius that varies based on the fan-out of feed-forward
    inputs. It is ``desired'' because the percentage varies some based
    on the particular input.}}

\item[Feed-Forward]{moving in a direction away from an input, or from
  a lower Level to a higher Level in a Hierarchy (sometimes called
  Bottom-Up)}

\item[Feedback]{moving in a direction towards an input, or from a
  higher Level to a lower level in a Hierarchy (sometimes called
  Top-Down)}

\item[First Order Prediction]{a prediction based only on the current
  input and not on the prior inputs --- compare to Variable Order
  Prediction}

\item[Hierarchical Temporal Memory (HTM]{a technology that replicates
  some of the structural and algorithmic functions of the neocortex}

\item[Hierarchy]{a network of connected elements where the connections
  between the elements are uniquely identified as Feed-Forward or
  Feedback}

\item[HTM Cortical Learning Algorithms]{the suite of functions for
  Spatial Pooling, Temporal Pooling, and learning and forgetting that
  comprise an HTM Region, also referred to as HTM Learning Algorithms}

\item[HTM Network]{a Hierarchy of HTM Regions}

\item[HTM Region]{the main unit of memory and Prediction in an
  HTM\\{\em An HTM region is comprised of a layer of highly
    interconnected cells arranged in columns. An HTM region today has
    a single layer of cells, whereas in the neocortex (and ultimately
    in HTM), a region will have multiple layers of cells. When
    referred to in the context of its position in a hierarchy, a
    region may be referred to as a level.}}

\item[Inference]{recognizing a spatial and temporal input pattern as
  similar to previously learned patterns}

\item[Inhibition Radius]{defines the area around a Column that it
  actively inhibits}

\item[Lateral Connections]{connections between Cells within the same
  Region}

\item[Level]{an HTM Region in the context of the Hierarchy}

\item[Neuron]{an information processing Cell in the brain\\{\em In
    this document, we use the word neuron specifically when referring
    to biological cells, and ``cell'' when referring to the HTM unit
    of computation.}}

\item[Permanence]{a scalar value which indicates the connection state
  of a Potential Synapse\\{\em A permanence value below a threshold
    indicates the synapse is not formed. A permanence value above the
    threshold indicates the synapse is valid. Learning in an HTM
    region is accomplished by modifying permanence values of potential
    synapses.}}

\item[Potential Synapse]{the subset of all Cells that could
  potentially form Synapses with a particular Dendrite Segment\\{\em
    Only a subset of potential synapses will be valid synapses at any
    time based on their permanence value.}}

\item[Prediction]{activating Cells (into a predictive state) that will
  likely become active in the near future due to Feed-Forward
  input\\{\em An HTM region often predicts many possible future inputs
    at the same time.}}

\item[Receptive Field]{the set of inputs to which a Column or Cell is
  connected\\{\em If the input to an HTM region is organized as a 2D
    array of bits, then the receptive field can be expressed as a
    radius within the input space.}}

\item[Sensor]{a source of inputs for an HTM Network}

\item[Sparse Distributed Representation]{representation comprised of
  many bits in which a small percentage are active and where no single
  bit is sufficient to convey meaning}

\item[Spatial Pooling]{the process of forming a sparse distributed
  representation of an input\\{\em One of the properties of spatial
    pooling is that overlapping input patterns map to the same sparse
    distributed representation.}}

\item[Sub-Sampling]{recognizing a large distributed pattern by
  matching only a small subset of the active bits in the large
  pattern}

\item[Synapse]{connection between Cells formed while learning}

\item[Temporal Pooling]{the process of forming a representation of a
  sequence of input patterns where the resulting representation is
  more stable than the input}

\item[Top-Down]{synonym for Feedback}

\item[Variable Order Prediction]{a prediction based on varying amounts
  of prior context --- compare to First Order Prediction\\{\em It is
    called ``variable'' because the memory to maintain prior context
    is allocated as needed. Thus a memory system that implements
    variable order prediction can use context going way back in time
    without requiring exponential amounts of memory.}}

\end{description}



\end{document}

