
\chapter*{Glossary}
\addcontentsline{toc}{chapter}{Glossary}
\label{glossary}

\begin{description}
\item[Active State]{a state in which Cells are active due to
  Feed-Forward input}

\item[Bottom-Up]{synonym to Feed-Forward}

\item[Cells]{HTM equivalent of a Neuron\\{\em Cells are organized into
    columns in HTM regions.}}

\item[Coincident Activity]{two or more Cells are active at the same
  time}

\item[Column]{a group of one or more Cells that function as a unit in
  an HTM Region\\{\em Cells within a column represent the same
    feed-forward input, but in different contexts.}}

\item[Dendrite Segment]{a unit of integration of Synapses associated
  with Cells and Columns\\{\em HTMs have two different types of
    dendrite segments. One is associated with lateral connections to a
    cell. When the number of active synapses on the dendrite segment
    exceeds a threshold, the associated cell enters the predictive
    state. The other is associated with feed-forward connections to a
    column. The number of active synapses is summed to generate the
    feed-forward activation of a column.}}

\item[Desired Density]{desired percentage of Columns active due to
  Feed-Forward input to a Region\\{\em The percentage only applies
    within a radius that varies based on the fan-out of feed-forward
    inputs. It is ``desired'' because the percentage varies some based
    on the particular input.}}

\item[Feed-Forward]{moving in a direction away from an input, or from
  a lower Level to a higher Level in a Hierarchy (sometimes called
  Bottom-Up)}

\item[Feedback]{moving in a direction towards an input, or from a
  higher Level to a lower level in a Hierarchy (sometimes called
  Top-Down)}

\item[First Order Prediction]{a prediction based only on the current
  input and not on the prior inputs --- compare to Variable Order
  Prediction}

\item[Hierarchical Temporal Memory (HTM]{a technology that replicates
  some of the structural and algorithmic functions of the neocortex}

\item[Hierarchy]{a network of connected elements where the connections
  between the elements are uniquely identified as Feed-Forward or
  Feedback}

\item[HTM Cortical Learning Algorithms]{the suite of functions for
  Spatial Pooling, Temporal Pooling, and learning and forgetting that
  comprise an HTM Region, also referred to as HTM Learning Algorithms}

\item[HTM Network]{a Hierarchy of HTM Regions}

\item[HTM Region]{the main unit of memory and Prediction in an
  HTM\\{\em An HTM region is comprised of a layer of highly
    interconnected cells arranged in columns. An HTM region today has
    a single layer of cells, whereas in the neocortex (and ultimately
    in HTM), a region will have multiple layers of cells. When
    referred to in the context of its position in a hierarchy, a
    region may be referred to as a level.}}

\item[Inference]{recognizing a spatial and temporal input pattern as
  similar to previously learned patterns}

\item[Inhibition Radius]{defines the area around a Column that it
  actively inhibits}

\item[Lateral Connections]{connections between Cells within the same
  Region}

\item[Level]{an HTM Region in the context of the Hierarchy}

\item[Neuron]{an information processing Cell in the brain\\{\em In
    this document, we use the word neuron specifically when referring
    to biological cells, and ``cell'' when referring to the HTM unit
    of computation.}}

\item[Permanence]{a scalar value which indicates the connection state
  of a Potential Synapse\\{\em A permanence value below a threshold
    indicates the synapse is not formed. A permanence value above the
    threshold indicates the synapse is valid. Learning in an HTM
    region is accomplished by modifying permanence values of potential
    synapses.}}

\item[Potential Synapse]{the subset of all Cells that could
  potentially form Synapses with a particular Dendrite Segment\\{\em
    Only a subset of potential synapses will be valid synapses at any
    time based on their permanence value.}}

\item[Prediction]{activating Cells (into a predictive state) that will
  likely become active in the near future due to Feed-Forward
  input\\{\em An HTM region often predicts many possible future inputs
    at the same time.}}

\item[Receptive Field]{the set of inputs to which a Column or Cell is
  connected\\{\em If the input to an HTM region is organized as a 2D
    array of bits, then the receptive field can be expressed as a
    radius within the input space.}}

\item[Sensor]{a source of inputs for an HTM Network}

\item[Sparse Distributed Representation]{representation comprised of
  many bits in which a small percentage are active and where no single
  bit is sufficient to convey meaning}

\item[Spatial Pooling]{the process of forming a sparse distributed
  representation of an input\\{\em One of the properties of spatial
    pooling is that overlapping input patterns map to the same sparse
    distributed representation.}}

\item[Sub-Sampling]{recognizing a large distributed pattern by
  matching only a small subset of the active bits in the large
  pattern}

\item[Synapse]{connection between Cells formed while learning}

\item[Temporal Pooling]{the process of forming a representation of a
  sequence of input patterns where the resulting representation is
  more stable than the input}

\item[Top-Down]{synonym for Feedback}

\item[Variable Order Prediction]{a prediction based on varying amounts
  of prior context --- compare to First Order Prediction\\{\em It is
    called ``variable'' because the memory to maintain prior context
    is allocated as needed. Thus a memory system that implements
    variable order prediction can use context going way back in time
    without requiring exponential amounts of memory.}}

\end{description}